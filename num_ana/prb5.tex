\documentclass[a4paper]{article}

\usepackage[T1]{fontenc}
\usepackage[utf8]{inputenc}
\usepackage{mlmodern}

%\usepackage{ngerman}	% Sprachanpassung Deutsch

\usepackage{graphicx}
\usepackage{geometry}
\geometry{a4paper, top=15mm}

\usepackage{subcaption}
\usepackage[shortlabels]{enumitem}
\usepackage{amssymb}
\usepackage{amsthm}
\usepackage{amsmath}
\usepackage{mathtools}
\usepackage{braket}
\usepackage{bbm}
\usepackage{graphicx}
\usepackage{float}
\usepackage{yhmath}
\usepackage{tikz}
\usepackage{scratch}
\usetikzlibrary{patterns,decorations.pathmorphing,positioning}
\usetikzlibrary{calc,decorations.markings}

\usepackage[backend=biber, sorting=none]{biblatex}
\addbibresource{cite.bib}

\usepackage[framemethod=TikZ]{mdframed}

\tikzstyle{titlered} =
    [draw=black, thick, fill=white,%
        text=black, rectangle,
        right, minimum height=.7cm]


\usepackage[colorlinks=true,naturalnames=true,plainpages=false,pdfpagelabels=true]{hyperref}
\usepackage[parfill]{parskip}
\usepackage{lipsum}

\usepackage{tcolorbox}
\tcbuselibrary{skins,breakable}

\pagestyle{myheadings}

\colorlet{colexam}{black}
\newcounter{definition}
\newtcolorbox[use counter=definition]{mydef}[1]{
    empty,
    title={\textbf{Definition~\thetcbcounter}~~(\textit{#1})},
    attach boxed title to top left,
    fontupper=\sl,
    boxed title style={
        empty,
        size=minimal,
        bottomrule=1pt,
        top=1pt,
        left skip=0cm,
        overlay=
            {\draw[colexam,line width=1pt]([yshift=-0.4cm]frame.north
        west)--([yshift=-0.4cm]frame.north east);}},
            coltitle=colexam,
            fonttitle=\normalfont,
            before=\par\medskip\noindent,
            parbox=false,
            boxsep=-1pt,
            left=0.75cm,
            right=3mm,
            top=4pt,
            breakable,
            pad at break*=0mm,
            vfill before first,
            overlay unbroken={
                \draw[colexam,line width=1pt]
                ([xshift=0.6cm, yshift=-0.5pt]frame.south
                west)--([xshift=0.6cm,yshift=-1pt]frame.north west)
                --([xshift=0.6cm]frame.south west)--([xshift=-13cm]frame.south east); },
            overlay first={
                \draw[colexam,line width=1pt]
                ([xshift=0.6cm, yshift=-0.5pt]frame.south
                west)--([xshift=0.6cm,yshift=-1pt]frame.north west)
                --([xshift=0.6cm]frame.south west); },
            overlay last={
                \draw[colexam,line width=1pt]
                ([xshift=0.6cm, yshift=-0.5pt]frame.south
                west)--([xshift=0.6cm,yshift=-1pt]frame.north west)
                --([xshift=0.6cm]frame.south west)--([xshift=-13cm]frame.south east); }
}
\newcounter{theorem}
\newtcolorbox[use counter=theorem]{theorem}{
    empty,
    title={Theorem ~\thetcbcounter},
    attach boxed title to top left,
    fontupper=\sl,
    boxed title style={
        empty,
        size=minimal,
        bottomrule=1pt,
        top=1pt,
        left skip=0cm,
        overlay=
            {\draw[colexam,line width=1pt]([yshift=-0.4cm]frame.north
        west)--([yshift=-0.4cm]frame.north east);}},
            coltitle=colexam,
            fonttitle=\bfseries,
            before=\par\medskip\noindent,
            parbox=false,
            boxsep=-1pt,
            left=0.75cm,
            right=3mm,
            top=4pt,
            breakable,
            pad at break*=0mm,
            vfill before first,
            overlay unbroken={
                \draw[colexam,line width=1pt]
                ([xshift=0.6cm, yshift=-0.5pt]frame.south
                west)--([xshift=0.6cm,yshift=-1pt]frame.north west)
                --([xshift=0.6cm]frame.south west)--([xshift=-13cm]frame.south east); },
            overlay first={
                \draw[colexam,line width=1pt]
                ([xshift=0.6cm, yshift=-0.5pt]frame.south
                west)--([xshift=0.6cm,yshift=-1pt]frame.north west)
                --([xshift=0.6cm]frame.south west); },
            overlay last={
                \draw[colexam,line width=1pt]
                ([xshift=0.6cm, yshift=-0.5pt]frame.south
                west)--([xshift=0.6cm,yshift=-1pt]frame.north west)
                --([xshift=0.6cm]frame.south west)--([xshift=-13cm]frame.south east); }
}
\newcounter{lemma}
\newtcolorbox[use counter=lemma]{lemma}{
    empty,
    title={Lemma~\thetcbcounter},
    attach boxed title to top left,
    fontupper=\sl,
    boxed title style={
        empty,
        size=minimal,
        bottomrule=1pt,
        top=1pt,
        left skip=0cm,
        overlay=
            {\draw[colexam,line width=1pt]([yshift=-0.4cm]frame.north
        west)--([yshift=-0.4cm]frame.north east);}},
            coltitle=colexam,
            fonttitle=\bfseries,
            before=\par\medskip\noindent,
            parbox=false,
            boxsep=-1pt,
            left=0.75cm,
            right=3mm,
            top=4pt,
            breakable,
            pad at break*=0mm,
            vfill before first,
            overlay unbroken={
                \draw[colexam,line width=1pt]
                ([xshift=0.6cm, yshift=-0.5pt]frame.south
                west)--([xshift=0.6cm,yshift=-1pt]frame.north west)
                --([xshift=0.6cm]frame.south west)--([xshift=-13cm]frame.south east); },
            overlay first={
                \draw[colexam,line width=1pt]
                ([xshift=0.6cm, yshift=-0.5pt]frame.south
                west)--([xshift=0.6cm,yshift=-1pt]frame.north west)
                --([xshift=0.6cm]frame.south west); },
            overlay last={
                \draw[colexam,line width=1pt]
                ([xshift=0.6cm, yshift=-0.5pt]frame.south
                west)--([xshift=0.6cm,yshift=-1pt]frame.north west)
                --([xshift=0.6cm]frame.south west)--([xshift=-13cm]frame.south east); }
}

\newcommand{\eps}{\varepsilon}
\usepackage[OT2,T1]{fontenc}
\DeclareSymbolFont{cyrletters}{OT2}{wncyr}{m}{n}
\DeclareMathSymbol{\Sha}{\mathalpha}{cyrletters}{"58}

\markright{Popović\hfill Seminar\hfill}


\title{University of Vienna\\
\vspace{1cm}Seminar:\\Joint RICAM Seminar\\
\vspace{0.5cm}
Summary of talk by Otmar Scherzer
}
\author{Milutin Popovic}


\begin{document}
\maketitle
\tableofcontents
\section{Sheet 5}
\subsection{Problem 1}
Let $A \in \mathbb{R}^{n \times n}$ be an SPD matrix with an additive
decomposition $A = D + L + L^T$, where $D$ consists of the diagonal entries
of $A$, $L$ and $L^T$ of the lower diagonal and upper-diagonal entries of $A$
respectively. For $\omega \in(0,2)$ the SSOR preconditioner is defined as
follows
\begin{align}
    C_\omega = \frac{1}{2-\omega}\left( \frac{1}{\omega} D+L\right)
    \left( \frac{1}{\omega} D \right)^{-1}\left( \frac{1}{\omega}D + L^T
    \right).
\end{align}
We can rewrite $C_\omega = KK^T$, where $K$ is an invertible lower-triangular
matrix by a simple splitting of the diagonal entries $D =
D^{\frac{1}{2}}D^{\frac{1}{2}}$. We get
\begin{align}
    C_\omega &= \frac{1}{2-\omega}\left( \frac{1}{\omega}D +L \right)\omega
    D^{-1}\left( \frac{1}{\omega}D + L^T \right)  \\
    &= \frac{1}{2-\omega}\left( D^{\frac{1}{2}} + \omega
    LD^{-\frac{1}{2}}\left( D^{\frac{1}{2}}+\omega LD^{-\frac{1}{2}} \right)
\right)^T = KK^T,
\end{align}
where
\begin{align}
    K := \frac{1}{\sqrt{1-\omega} }\left( D^{\frac{1}{2}} + \omega LD^{-
    \frac{1}{2}} \right)
\end{align}
The Matrix $C_\omega$ is a good approximation for the inverse of $A$ because
\begin{align}
    H^{\text{SSOR}}_\omega = I - C^{-1}_\omega A\\
    \rho(H^{\text{SSOR}}_\omega) < 1
\end{align}
for a right choice of $\omega$. Furthermore the general idea is that we
multiply the system $Ax=b$ with a preconditioning matrix $P \in
\text{GL}_n\left(\mathbb{R}\right)$ with inevitability, then we get the
system
\begin{align}
    \underbrace{PA}_{\approx I}x&= Pb\\
\end{align}
The thing is that $\text{cond}_2(A) = v(n)$ bound by the curse of
dimensionality and $\text{cond}_2 = s \ll v(n)$ not dependent and thereby $P$
would be an optimal preconditioner.
\subsection{Exercise 2}
Let $m, n \in \mathbb{N}$, $I \in \mathbb{R}^{m\times m}$ the identity in
$\mathbb{R}^{m\times m}$ and $Q$ be the banded matrix
\begin{align}
    Q =
    \begin{pmatrix}
        4  & - 1 &&  &   \\
        -1 &  4& -1&  &   \\
           &\ddots& \ddots & \ddots & \\
           &   &   -1 & 4 & -1\\
           &   &    & & 4 \\
    \end{pmatrix}
\end{align}
The eigenvalues of the matrix $Q$ lie in $\sigma(A) \subset [4-2, 4+2]$ by
the Gershorin disk theorem. Since no eigenvalue is $0$, then $Q$ is
invertible.
Now consider the matrix $A \in \mathbb{R}^{nm \times nm}$
\begin{align}
    Q =
    \begin{pmatrix}
        Q  & - I &  &  & \\
        -I &  Q& -I&    & \\
           &\ddots& \ddots & \ddots & \\
           &     & -I & Q & -I\\
           &     &  & & Q \\
    \end{pmatrix}
\end{align}
We can consider the separation wrt. addition $A = D + L + L^T$ (Like in
Exercise 1). The Jacobi-Method iteration matrix is $J = - D^{-1}(L + L^T)$,
where $L$ is the lower triangular with \textbf{-1 or 0 entries}. Further more
the Gershorin theorem states that $\sigma(A) < 1$. All in all the matrix $I -
J$ is by the geometric(Neumann) series
\begin{align}
    (I-J)^{-1} = \sum_{n=0}^{\infty}J^k
\end{align}
and we have the identity
\begin{align}
    J = I - D^{-1} A \quad \Rightarrow \quad (I-J)^{-1} = DA^{-1}.
\end{align}
Thereby the sum transforms to
\begin{align}
    A^{-1} = D^{-1}\sum_{n=0}^{\infty} J^k.
\end{align}
The entries of $D$ are all $4$ and thereby non-negative, the matrix is also
invertible. The matrix $L$ has only -1 or 0 entries which get compensated
with the minus sing in $J = -D^{-1}(L +L^T) = D^{-1}(-L - L^T)$, thereby all
entries of $J^k$ are positive for all $k$. Finally we arrive at the
conclusion, that all entries of $A^{-1}$ are non-negative and $A$ is a
$M$-matrix or `(inverse) monotone' matrix.
\subsection{Exercise 3}
Let $A \in \mathbb{R}^{n \times n}$ be an SPD matrix and $b \in
\mathbb{R}^{n}$ be a right hand side of a linear system. Suppose we apply the
CG method for solving $Ax = b$. The $k$-th iterate $x_k$ of the CG method
then satisfies the $A$-norm optimality condition
\begin{align}
    \|x_k - x\|_A = \min_{y\in x_0 + B_k} \|y - x\|_A,
\end{align}
where
\begin{align}
    B_k = \text{span}\left\{ p_0,\ldots,p_{k-1} \right\} = \text{span}\left\{
    r_0, Ar_0, \ldots,A^{k-1}r_0\right\}
\end{align}
is the Krylov space. The search directions $p_k$ form an $A$-orthogonal
system.
\newline
Now if the spectrum of $A$, $\sigma(A) = [a, b] \subset (0, \infty)$ then
for any polynomial $p \in \mathbb{P}_k^{0, 1}:= \left\{p \in \mathbb{P}: p(0)
= 1\right\} $ we have that
\begin{align}
    \|x_k - x\|_A \le \left( \sup_{t\in[a,b]} \mid p(t) \mid \right)
    \|x_0 - x\|_A.
\end{align}
To show this we have that for all $y \in x_0 + B_k$ the representation
\begin{align}
    y = \sum_{j=0}^{k-1} c_j A^j r_0 = x_0 + q_y(A)r_0
\end{align}
for suitable $c_j$'s and a polynomial $q_y \in \mathbb{P}_{k-1}$. Now
\begin{align}
    y - x &= x_0 + x - q_y(A) r_0 = x_0 - x + q_y(A) \left( b - Ax_0
    \right)\\
          &= x_0 - x + q_y(A) \left(Ax - Ax_0 \right)\\
          &=\underbrace{\left(I - q_y(A)A \right)}_{=: p_y(A) \in
          \mathbb{P}_k^{0, 1}} (x-x_0).
\end{align}
With this information we may consider the norm
\begin{align}
    \|x_k - x\|_A \leq \|p_y(A)(x- x_0)\|_A \qquad \forall y \in x_0+B_k.
\end{align}
Now we use the fact that $A$ is SPD thereby there is an orthogonal matrix $Q$
diagonalizing $A = Q^T \Lambda Q$, with $\Lambda = \text{diag}(\lambda_1, \ldots,
\lambda_n)$ consisting of eigenvalues of $A$, then
\begin{align}
    A^k = Q^T\Lambda Q \cdots Q^T \Lambda Q = Q^T \Lambda^k Q
\end{align}
With this we can transform the polynomial $p_y(A)$ and the geometric
(Neumann) series
\begin{align}
    p_y(A) = \sum_{j=0}^{\infty} c_j A^j = Q^T \left(\sum_{j=0}^{\infty} c_j
        \Lambda^j\right)
\end{align}
The norm becomes then
\begin{align}
    \|p(A) (x-x_0)\|^2_a
    &= \langle Ap(A) (x-x_0), p(A)(x-x_0)\rangle =\\
    &= \langle Q^T\Lambda Q Q^Tp(\Lambda)Q (x-x_0),
    Q^Tp(\Lambda)Q(x-x_0)\rangle =\\
    &= \langle Q^T\Lambda p(\Lambda)Q (x-x_0),
    Q^Tp(\Lambda)Q(x-x_0)\rangle =\\
    &= \langle \Lambda p(\Lambda)Q (x-x_0),
    p(\Lambda)Q(x-x_0)\rangle =\\
    &= \langle \Lambda^{\frac{1}{2}} p(\Lambda)Q (x-x_0),
    \Lambda^{\frac{1}{2}} p(\Lambda)Q(x-x_0)\rangle =\\
    &= \|\Lambda^{\frac{1}{2}}p(\Lambda)Q(x-x_0)\|_2\\
    &= \|p(\Lambda)\Lambda^{\frac{1}{2}}Q(x-x_0)\|_2\\
    &\leq \|p(\Lambda)\|_2 \|\Lambda^{\frac{1}{2}}Q(x-x_0)\|_2.
\end{align}
The Norm of the polynomial is the maximal eigenvalue thereby
\begin{align}
    \|p(\Lambda\|_2 = \max_{\lambda \in \sigma(A)}  \mid p(\lambda) \mid \leq
    \sup_{t\in[a,b]}  \mid p(t)\mid,
\end{align}
we can do the supremum boundary because $\lambda \in [a, b]$. As for the
second part
\begin{align}
    \|\Lambda^{\frac{1}{2}}Q(x-x_0)\|_2^2 &= (x-x_0)^T Q^T
    \Lambda^{\frac{1}{2}}\Lambda^{\frac{1}{2}}Q(x-x_0)\\
    &= (x-x_0)^T A (x-x_0) \\
    &= \|x-x_0\|_A^2.
\end{align}
And finally we get the result
\begin{align}
    \|x_k - x\| \le \sup_{t\in[a,b]}  \mid p(t)  \mid \|x-x_0\|_A^2.
\end{align}
The last approximation can be done because $\sup_{t\in[a,b]} \mid p(t) \mid$
holds for \textbf{all} $p \in \mathbb{P}_k^{0,1}$ thereby we can bound by an
infimum over all the polynomials in $\mathbb{P}_k^{0,1}$ and we get
\begin{align}
    \sup_{t\in[a,b]}  \mid p(t)  \mid \le \inf_{p \in \mathbb{P}_k^{0,1}}
    \|p\|_{C([0,1])}.
\end{align}
\subsubsection{Exercise 4}
We can do subsequently the as in the last exercise wit the GMRES method. So
we let $A \in \mathbb{R}^{n \times n}$ be an SPD and $b \in \mathbb{R}^{n}$
be the right hand side of the linear system. The iterates of $x_k$  of the CG
method satisfy the $A^{-1}$-norm optimality
\begin{align}
    \|Ax_k - b\|_{A^{-1}} = \min_{y\in x_0 + C_k} \|Ay - b\|_{A^{-1}},
\end{align}
with $C_k = \text{span}\left\{ p_0, Ap_0, \ldots , A^{k-1}p_0 \right\}$. The
`generalized minimal residual', short GMRES method, instead, formally
constructs a sequence of iterates $x_k^G$ by
\begin{align}
    \|Ax^G_2 - b \|_2 = \min_{y \in x_0 + C_k}\|Ay - b\|_2.
\end{align}
The GMRES method allows for an error inequality similar to the one observed
in the CG method
\begin{align}
    \|Ax_k^G - b\|_2 \le \inf_{p \in \mathbb{P}_k^{0,1}} \|p(A)\|_2 \|Ax_0
    -b\|_2.
\end{align}
To show this we start off by minimizing over a $z \in C_k$
\begin{align}
    \|Ax_k^G - b\|_2 = \min_{y\in x_k + C_k} \|Ay - b\|_2 = \min_{z \in
    C_k}\|Az + Ax_0 -b\|_2.
\end{align}
Then for all $z \in C_k$, there exists a $\pi_k \in \mathbb{P}_{k-1}$ such that
\begin{align}
    z = \pi_k(A) p_0 = \pi_k(A) r_0,
\end{align}
Then the minimization can be bounded
\begin{align}
    \min_{z \in C_k} \|Az + Ax_0 - b\|_2 &=
    \min_{\pi_k \in \mathbb{P}_k}\|A\pi_k(A)r_0 + Ax_0 -b\|_2\\
        &\le \|A\pi_k(A) (b-Ax_0) + Ax_0 +b\|_2\\
        &= \|(Ax_0 -b)\underbrace{(I-A\pi_k(A)}_{=:p \in
        \mathbb{P}_k^{0,1}}\|_2\\
        &= \|(Ax_0-b)p(A)\|\\
        &\le \|p(A)\|_2 \|Ax_0 -b\|_2.
\end{align}
Following the same argumentation as in Exercise 3 we get for the norm of the
polynomial
\begin{align}
    \|p(A)\|_2  &= \max_{\lambda \in \sigma(A)}  \mid p(\lambda) \mid\\
    &\le \sup_{t\in [a,b]}  \mid p(t) \mid\\
    &\le \inf_{p \in \mathbb{P}_k^{0,1}} \|p\|_{C\left([a,b]  \right) }.
\end{align}



\end{document}


