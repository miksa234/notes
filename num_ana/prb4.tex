\documentclass[a4paper]{article}

\usepackage[T1]{fontenc}
\usepackage[utf8]{inputenc}
\usepackage{mlmodern}

%\usepackage{ngerman}	% Sprachanpassung Deutsch

\usepackage{graphicx}
\usepackage{geometry}
\geometry{a4paper, top=15mm}

\usepackage{subcaption}
\usepackage[shortlabels]{enumitem}
\usepackage{amssymb}
\usepackage{amsthm}
\usepackage{mathtools}
\usepackage{braket}
\usepackage{bbm}
\usepackage{graphicx}
\usepackage{float}
\usepackage{yhmath}
\usepackage{tikz}
\usetikzlibrary{patterns,decorations.pathmorphing,positioning}
\usetikzlibrary{calc,decorations.markings}

%\usepackage[backend=biber, sorting=none]{biblatex}
%\addbibresource{uni.bib}

\usepackage[framemethod=TikZ]{mdframed}

\tikzstyle{titlered} =
    [draw=black, thick, fill=white,%
        text=black, rectangle,
        right, minimum height=.7cm]


\usepackage[colorlinks=true,naturalnames=true,plainpages=false,pdfpagelabels=true]{hyperref}
\usepackage[parfill]{parskip}
\usepackage{lipsum}

\usepackage{tcolorbox}
\tcbuselibrary{skins,breakable}

\pagestyle{myheadings}

\newcommand{\eps}{\varepsilon}
\usepackage[OT2,T1]{fontenc}
\DeclareSymbolFont{cyrletters}{OT2}{wncyr}{m}{n}
\DeclareMathSymbol{\Sha}{\mathalpha}{cyrletters}{"58}

\markright{Popović\hfill Numerical Analysis\hfill}


\title{University of Vienna\\ Faculty of Mathematics\\
\vspace{1cm}Numerical Analysis Problems
}
\author{Milutin Popovic}


\begin{document}
\maketitle
\tableofcontents
\section{Sheet 4}
\subsection{Problem 1}
Consider a linear system of equations $Ax = b$, where
\begin{align}
    A =
    \begin{pmatrix}
        2 & -1 & 0 \\
        -1 & 2 & -1\\
        0 & -1 & 2
    \end{pmatrix}, \qquad
    b =
    \begin{pmatrix}
        4 \\ 0 \\ 0
    \end{pmatrix},
\end{align}
we carry out iterations of the CG method by hand until we reach the
solution with an initial guess $x_0 = \begin{pmatrix} 0 & 0 & 0
\end{pmatrix}^T$. For the sake of completeness the CG method has the following
iteration at the $k$-th step
\begin{align}
    \alpha_k &= \frac{r_k^Tr_k}{p_k^TAp_k}\\
    x_{k+1} &=  x_k + \alpha_k p_k\\
    r_{k+1} &= r_k - \alpha_k A p_k \\
    \beta_{k} &= \frac{r_{k+1}^Tr_{k+1}}{r_{k}^T r_k}\\
    p_{k+1} &= r_{k+1} + \beta_{k}p_k \\
\end{align}
For $k=0$ we have
\begin{align}
    r_0 &= b - Ax_0 = b,\\
    p_0 &= r_0 = b = \begin{pmatrix} 4 & 0 & 0 \end{pmatrix}^T.
\end{align}
For k=1 we have
\begin{align}
    \alpha_0 &= \frac{1}{2}, \quad x_1=\begin{pmatrix} 2 \\ 0 \\0
        \end{pmatrix}, \quad r_1 = \begin{pmatrix} 0 \\ 2 \\0
    \end{pmatrix},\\
    \beta_0 &= \frac{1}{4},\quad p_1 = \begin{pmatrix} 1\\2\\0
    \end{pmatrix}.
\end{align}
For k=2 we have
\begin{align}
    \alpha_1 &= \frac{2}{3}, \quad x_2=\frac{1}{3}\begin{pmatrix} 8 \\ 4 \\0
        \end{pmatrix}, \quad r_2 = \frac{1}{3}\begin{pmatrix} 0 \\ 0 \\4
    \end{pmatrix},\\
    \beta_1 &= \frac{4}{9},\quad p_2 = \frac{1}{9}\begin{pmatrix} 4\\8\\12
    \end{pmatrix}.
\end{align}
For k=3 we have
\begin{align}
    \alpha_2 &= \frac{3}{4}, \quad x_3=\begin{pmatrix} 1 \\ 2 \\3
        \end{pmatrix}, \quad r_3 = \begin{pmatrix} 0 \\ 0 \\0
    \end{pmatrix},\\
    \beta_2 &= 0,\quad p_3 = \begin{pmatrix} 0\\0\\0
    \end{pmatrix}.
\end{align}
Since $r_3 = \textbf{0}$ we can stop here, and $x_3 = x$ is the unique
solution. The Krylov space of $\mathcal{K}_k(A, b)$ is defined for $k=3$ as
\begin{align}
    \mathcal{K}_3(A,b) = \text{span}\left\{b, Ab, A^2b \right\} = \text{span} \left\{ \begin{pmatrix} 0\\0\\4 \end{pmatrix},
             \begin{pmatrix} 8\\-4\\0 \end{pmatrix},
         \begin{pmatrix}26\\-16\\4\end{pmatrix}\right\}
\end{align}
the rank of the span of $\mathcal{K}_k(A,b)$ is full thereby the
$\dim(\mathcal{K}_k(A,b)) = 3$. Furthermore the residuals $r_0,\ldots,
r_{k-1}$ form an orthogonal basis for $\mathcal{K}_k(A,b)$. This can be
verified by checking that `key' elements in $\mathcal{K}_k(A, b)$ can be
expressed as a linear combination of $r_0, r_1, r_2$.
\begin{align}
    b = 3\cdot r_2,\quad Ab = 2r_0-2r_1,\quad A^2b=6r_0-8r_1+3r_2.
\end{align}
\subsection{Exercise 3, 4}
Not important see notes is not easy
\end{document}
