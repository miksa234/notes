\documentclass[a4paper]{article}

\usepackage[T1]{fontenc}
\usepackage[utf8]{inputenc}
\usepackage{mlmodern}

%\usepackage{ngerman}	% Sprachanpassung Deutsch

\usepackage{graphicx}
\usepackage{geometry}
\geometry{a4paper, top=15mm}

\usepackage{subcaption}
\usepackage[shortlabels]{enumitem}
\usepackage{amssymb}
\usepackage{amsthm}
\usepackage{amsmath}
\usepackage{mathtools}
\usepackage{braket}
\usepackage{bbm}
\usepackage{graphicx}
\usepackage{float}
\usepackage{yhmath}
\usepackage{tikz}
\usepackage{scratch}
\usetikzlibrary{patterns,decorations.pathmorphing,positioning}
\usetikzlibrary{calc,decorations.markings}

\usepackage[backend=biber, sorting=none]{biblatex}
\addbibresource{cite.bib}

\usepackage[framemethod=TikZ]{mdframed}

\tikzstyle{titlered} =
    [draw=black, thick, fill=white,%
        text=black, rectangle,
        right, minimum height=.7cm]


\usepackage[colorlinks=true,naturalnames=true,plainpages=false,pdfpagelabels=true]{hyperref}
\usepackage[parfill]{parskip}
\usepackage{lipsum}

\usepackage{tcolorbox}
\tcbuselibrary{skins,breakable}

\pagestyle{myheadings}

\colorlet{colexam}{black}
\newcounter{definition}
\newtcolorbox[use counter=definition]{mydef}[1]{
    empty,
    title={\textbf{Definition~\thetcbcounter}~~(\textit{#1})},
    attach boxed title to top left,
    fontupper=\sl,
    boxed title style={
        empty,
        size=minimal,
        bottomrule=1pt,
        top=1pt,
        left skip=0cm,
        overlay=
            {\draw[colexam,line width=1pt]([yshift=-0.4cm]frame.north
        west)--([yshift=-0.4cm]frame.north east);}},
            coltitle=colexam,
            fonttitle=\normalfont,
            before=\par\medskip\noindent,
            parbox=false,
            boxsep=-1pt,
            left=0.75cm,
            right=3mm,
            top=4pt,
            breakable,
            pad at break*=0mm,
            vfill before first,
            overlay unbroken={
                \draw[colexam,line width=1pt]
                ([xshift=0.6cm, yshift=-0.5pt]frame.south
                west)--([xshift=0.6cm,yshift=-1pt]frame.north west)
                --([xshift=0.6cm]frame.south west)--([xshift=-13cm]frame.south east); },
            overlay first={
                \draw[colexam,line width=1pt]
                ([xshift=0.6cm, yshift=-0.5pt]frame.south
                west)--([xshift=0.6cm,yshift=-1pt]frame.north west)
                --([xshift=0.6cm]frame.south west); },
            overlay last={
                \draw[colexam,line width=1pt]
                ([xshift=0.6cm, yshift=-0.5pt]frame.south
                west)--([xshift=0.6cm,yshift=-1pt]frame.north west)
                --([xshift=0.6cm]frame.south west)--([xshift=-13cm]frame.south east); }
}
\newcounter{theorem}
\newtcolorbox[use counter=theorem]{theorem}{
    empty,
    title={Theorem ~\thetcbcounter},
    attach boxed title to top left,
    fontupper=\sl,
    boxed title style={
        empty,
        size=minimal,
        bottomrule=1pt,
        top=1pt,
        left skip=0cm,
        overlay=
            {\draw[colexam,line width=1pt]([yshift=-0.4cm]frame.north
        west)--([yshift=-0.4cm]frame.north east);}},
            coltitle=colexam,
            fonttitle=\bfseries,
            before=\par\medskip\noindent,
            parbox=false,
            boxsep=-1pt,
            left=0.75cm,
            right=3mm,
            top=4pt,
            breakable,
            pad at break*=0mm,
            vfill before first,
            overlay unbroken={
                \draw[colexam,line width=1pt]
                ([xshift=0.6cm, yshift=-0.5pt]frame.south
                west)--([xshift=0.6cm,yshift=-1pt]frame.north west)
                --([xshift=0.6cm]frame.south west)--([xshift=-13cm]frame.south east); },
            overlay first={
                \draw[colexam,line width=1pt]
                ([xshift=0.6cm, yshift=-0.5pt]frame.south
                west)--([xshift=0.6cm,yshift=-1pt]frame.north west)
                --([xshift=0.6cm]frame.south west); },
            overlay last={
                \draw[colexam,line width=1pt]
                ([xshift=0.6cm, yshift=-0.5pt]frame.south
                west)--([xshift=0.6cm,yshift=-1pt]frame.north west)
                --([xshift=0.6cm]frame.south west)--([xshift=-13cm]frame.south east); }
}
\newcounter{lemma}
\newtcolorbox[use counter=lemma]{lemma}{
    empty,
    title={Lemma~\thetcbcounter},
    attach boxed title to top left,
    fontupper=\sl,
    boxed title style={
        empty,
        size=minimal,
        bottomrule=1pt,
        top=1pt,
        left skip=0cm,
        overlay=
            {\draw[colexam,line width=1pt]([yshift=-0.4cm]frame.north
        west)--([yshift=-0.4cm]frame.north east);}},
            coltitle=colexam,
            fonttitle=\bfseries,
            before=\par\medskip\noindent,
            parbox=false,
            boxsep=-1pt,
            left=0.75cm,
            right=3mm,
            top=4pt,
            breakable,
            pad at break*=0mm,
            vfill before first,
            overlay unbroken={
                \draw[colexam,line width=1pt]
                ([xshift=0.6cm, yshift=-0.5pt]frame.south
                west)--([xshift=0.6cm,yshift=-1pt]frame.north west)
                --([xshift=0.6cm]frame.south west)--([xshift=-13cm]frame.south east); },
            overlay first={
                \draw[colexam,line width=1pt]
                ([xshift=0.6cm, yshift=-0.5pt]frame.south
                west)--([xshift=0.6cm,yshift=-1pt]frame.north west)
                --([xshift=0.6cm]frame.south west); },
            overlay last={
                \draw[colexam,line width=1pt]
                ([xshift=0.6cm, yshift=-0.5pt]frame.south
                west)--([xshift=0.6cm,yshift=-1pt]frame.north west)
                --([xshift=0.6cm]frame.south west)--([xshift=-13cm]frame.south east); }
}

\newcommand{\eps}{\varepsilon}
\usepackage[OT2,T1]{fontenc}
\DeclareSymbolFont{cyrletters}{OT2}{wncyr}{m}{n}
\DeclareMathSymbol{\Sha}{\mathalpha}{cyrletters}{"58}

\markright{Popović\hfill Seminar\hfill}


\title{University of Vienna\\
\vspace{1cm}Seminar:\\Joint RICAM Seminar\\
\vspace{0.5cm}
Summary of talk by Otmar Scherzer
}
\author{Milutin Popovic}


\begin{document}
\maketitle
\tableofcontents
\section{Sheet 7}
\subsection{Problem 1}
For a matrix $A \in \mathbb{C}^{n\times n}$, define $B, C \in
\mathbb{C}^{n\times n}$ in the following way
\begin{align}
    B = \frac{1}{2}(A + A^*), \quad C = \frac{1}{2i} (A - A^*).
\end{align}
The matrices $B$ and $C$ are Hermitian, which can be seen by directly
calculating the adjoint.
\begin{align}
    B^* &= \frac{1}{2}(A+A^*)^* = \frac{1}{2}\left( A^* + (A^*)^*\right)\\
        &= \frac{1}{2}(A^* + A) = \frac{1}{2}(A+A^*) = B,\\
        \nonumber\\
    C^* &= -\frac{1}{2i}(A-A^*)^* = -\frac{1}{2i}\left( A^* - (A^*)^*\right)\\
        &= -\frac{1}{2i}(A^* - A) = \frac{1}{2}(A -A^*) = C.
\end{align}
Additionally we can bound the eigenvalues of $A$ by the minimum and maximum
eigenvalues of $B$ and $C$ by rewriting $A$ as
\begin{align}
    A = (B + iC).
\end{align}
Now consider an arbitrary eigenpair of $A$, $(\lambda, v)$, such that $\|v\|
= 1$, the eigenvalue equation reads
\begin{align}
    Av &= (B + iC)v = \lambda v\\
        &= Bv + iCv\\
    \Leftrightarrow & v^* B v + v^*(iC)v = \lambda.
\end{align}
The real and the imaginary part of $\lambda$ can be calculated by a simple
identity
\begin{align}
    \text{Re}(\lambda)
    &=  \frac{1}{2}(\lambda + \bar{\lambda}) \\
    &= \frac{1}{2}(v^* B v + v^*(iC)v + v^*B^*v - v^* (i C)v)\\
    &= \frac{1}{2} ( v^*Bv + v^*B^*v + v^* (iC)v - v^*(iC)v)\\
    &= \frac{1}{2}(2v^* B v) = v^*Bv\\
    \nonumber\\
    \text{Im}(\lambda)
    &=  \frac{1}{2i}(\lambda - \bar{\lambda}) \\
    &= v^*Cv
\end{align}
Putting the results from above with the Reighley-Ritz Theorem, which states
that for all $D \in \mathbb{C}^{n\times n}$ Hermitian $\forall x \in
\mathbb{C}^{n}$, where $x\neq 0 $ we have a boundary from below and above by
the minimum and maximum eigenvalue of $D$
\begin{align}
    \lambda_{\text{min}}(D) \le \frac{x^*Dx}{\|x\|^2} \le
    \lambda_{\text{max}}(D)
\end{align}
Then we have
\begin{align}
    \Rightarrow  \begin{cases}
        \text{Re}(\lambda) \in [ \lambda_{\text{min}}(B),
        \lambda_{\text{max}}(B)]\\
        \text{Im}(\lambda) \in [ \lambda_{\text{min}}(C),
        \lambda_{\text{max}}(C)]\\
    \end{cases}
\end{align}
\subsection{Problem 2}
Given two Hermitian matrices $A, B \in \mathbb{C}^{n\times n}$, denote
$\left\{ \lambda_j(A) \right\}_{j=1}^n $ and $\left\{ \lambda_j(A + B)
\right\}_{j=1}^n$ the eigenvalues of $A$ and $A+B$ in increasing order. If
$B$ is positive semi-definite then we have a bound
\begin{align}
    \lambda_k(A) \le \lambda_k(A+B) \qquad \forall k \in \left\{ 1, \ldots,n
    \right\}.
\end{align}
By the Courant Fischer Theorem, let $\mathcal{V}_k$ be the set of all $k$
dimensional subsets of $\mathbb{C}^{n\times n}$ we have
\begin{align}
    \lambda_k(A)
    &= \min_{v \in \mathcal{V}_k} \max_{v\in \mathbb{C}^{n \times
    n},\; \|v\|=1} \langle v, Av\rangle .
\end{align}
And if $B$ is positive  semi-definite we have
\begin{align}
    x^* B x \ge 0 \qquad \forall x \in \mathbb{C}^{n}.
\end{align}
Since $A$ and $B$ are hermitian, then $A+B$ are hermitian too and we can
write
\begin{align}
    \lambda_k(A)
    &= \min_{v \in \mathcal{V}_k} \max_{v\in \mathbb{C}^{n \times
    n},\; \|v\|=1} \langle v, Av\rangle \\
    &\ge \min_{v \in \mathcal{V}_k} \max_{v\in \mathbb{C}^{n \times
    n},\; \|v\|=1} \langle v, Av\rangle = \lambda_k(A)
\end{align}
\subsection{Problem 3}
Let $A \in \mathbb{C}^{n\times n}$ be diagonalizable by $X = (x_1,\ldots,x_n)
\in \mathbb{C}^{n \times n}$ the matrix of right-eigenvectors $x_j \in
\mathbb{C}^{n}$ of A. For all $\varepsilon> 0 $, let $\nu$ be the eigenvalues
of $A+\varepsilon A$, then there exists and eigenvalue $\lambda$ of $A$ with
\begin{align}
    \frac{|\lambda - \nu|}{|\lambda|} \le K_p(X)\varepsilon
\end{align}
Let us rewrite
\begin{align}
    A + \varepsilon A = (1+\varepsilon) A,
\end{align}
then the eigenvalue $\nu \in \lambda(A+\varepsilon A)$ can be written as an
eigenvalue of $A$ with
\begin{align}
    \frac{\nu}{1+\varepsilon} \in \lambda (A).
\end{align}
Then the bound reads
\begin{align}
    \frac{|\lambda - \nu|}{|\lambda|}
    &= \frac{|\frac{\nu}{1+\varepsilon} - \nu|}{|\frac{\nu}{1+\varepsilon}|}\\
    &= \frac{|\nu - (1+\varepsilon)\nu|}{|\nu|}\\
    &= \varepsilon \le \varepsilon K_p(X),
\end{align}
since $K_p(X) \ge 1$ for all $X$ that diagonalize $A$, if $A$ is invertible
!.
\subsection{Exercise 4}
Given some $\mu \in \mathbb{R}$ the shifted QR-algorithm is defined as: Let
$Q_0$ be orthogonal, such that $T_0 =  Q_0^T A Q_0$ is upper Hessenberg form.
For $k \in \mathbb{N}$ determine a sequence of the matrices $T_k$ by
\begin{itemize}
    \item Determine $Q_k$ and $R_k$, s.t. $Q_k R_k = T_{k-1} - \mu I$, as a
        QR-decomposition of $T_{k-1} - \mu I$
    \item Let $T_k = R_k Q_k + \mu I$
\end{itemize}
The sequence of these matrices $T_k$ is infact similar to $A$, in the
following way
\begin{align}
    T_{k+1}
    &= R_k Q_k + \mu I \\
    &= Q_k^T ( T_k - \mu I) Q_k + \mu I\\
    &= Q_k^T T_k Q_k - \mu I + \mu I\\
    &= Q_k^T T_k Q_k\\
    &= Q_k^T \cdots Q_1^T T_0 Q_1 \cdots Q_k\\
    &= \underbrace{Q_k^T \cdots Q_0^T}_{=Q^T} A \underbrace{Q_0 \cdots Q_k}_{= Q}
\end{align}
Furthermore if $A$ is an unreduced Hessenberg matrix and $\mu$ an eigenvalue
of $A$. Then let $QR = A-\mu I$ be the QR-decomposition of $A-\mu I $, define
\begin{align}
  \overline{A} = RQ  + \mu I,
\end{align}
then
\begin{align}
    \overline{A}_{n,n} = \mu \quad \& \quad \overline{A}_{n-1, n} = 0
\end{align}
To start, if $A$ is an irreducible Hessenber then
\begin{align}
    A_{i+1, i} \neq 0 \qquad \forall i \in \left\{ 1, \ldots , n-1 \right\}.
\end{align}
Then $A-\mu I$ is singular since $\mu$ is Eigenvalue of $A$, $\det(A-\mu I)
=0 $ is an eigenvalue equation. And additionally $0$ is an eigenvalue of $A -
\mu I$, then
\begin{align}
     &\Rightarrow \overline{A} = RQ + \mu I.
\end{align}
Where $A-\mu I$ is singular and the first  $n-1$ columns are linearly
independent, since $R = Q^T(A-\mu I)$. Then the first $n-1$ columns of $R$
are linearly independent and because $R$ is also singular perserved by
rotation of $Q^T$ the last row needs to be $0$, i.e. $R_{n, \cdot} = 0^T$,
then
\begin{align}
    &R_{n,n-1} = 0,\qquad (RQ)_{n, n-1} = 0,\\
    &R_{n,n} = 0,\qquad (RQ)_{n, n} = 0.\\
\end{align}
By this we conlude
\begin{align}
    \overline{A}_{n,n} &= (RQ)_{n,n}+\mu = \mu\\
    \overline{A}_{n, n-1} &= (RQ)_{n,n-1} = 0
\end{align}

\end{document}


