\documentclass[a4paper]{article}

\usepackage[T1]{fontenc}
\usepackage[utf8]{inputenc}
\usepackage{mlmodern}

%\usepackage{ngerman}	% Sprachanpassung Deutsch

\usepackage{graphicx}
\usepackage{geometry}
\geometry{a4paper, top=15mm}

\usepackage{subcaption}
\usepackage[shortlabels]{enumitem}
\usepackage{amssymb}
\usepackage{amsthm}
\usepackage{amsmath}
\usepackage{mathtools}
\usepackage{braket}
\usepackage{bbm}
\usepackage{graphicx}
\usepackage{float}
\usepackage{yhmath}
\usepackage{tikz}
\usepackage{scratch}
\usetikzlibrary{patterns,decorations.pathmorphing,positioning}
\usetikzlibrary{calc,decorations.markings}

\usepackage[backend=biber, sorting=none]{biblatex}
\addbibresource{cite.bib}

\usepackage[framemethod=TikZ]{mdframed}

\tikzstyle{titlered} =
    [draw=black, thick, fill=white,%
        text=black, rectangle,
        right, minimum height=.7cm]


\usepackage[colorlinks=true,naturalnames=true,plainpages=false,pdfpagelabels=true]{hyperref}
\usepackage[parfill]{parskip}
\usepackage{lipsum}

\usepackage{tcolorbox}
\tcbuselibrary{skins,breakable}

\pagestyle{myheadings}

\colorlet{colexam}{black}
\newcounter{definition}
\newtcolorbox[use counter=definition]{mydef}[1]{
    empty,
    title={\textbf{Definition~\thetcbcounter}~~(\textit{#1})},
    attach boxed title to top left,
    fontupper=\sl,
    boxed title style={
        empty,
        size=minimal,
        bottomrule=1pt,
        top=1pt,
        left skip=0cm,
        overlay=
            {\draw[colexam,line width=1pt]([yshift=-0.4cm]frame.north
        west)--([yshift=-0.4cm]frame.north east);}},
            coltitle=colexam,
            fonttitle=\normalfont,
            before=\par\medskip\noindent,
            parbox=false,
            boxsep=-1pt,
            left=0.75cm,
            right=3mm,
            top=4pt,
            breakable,
            pad at break*=0mm,
            vfill before first,
            overlay unbroken={
                \draw[colexam,line width=1pt]
                ([xshift=0.6cm, yshift=-0.5pt]frame.south
                west)--([xshift=0.6cm,yshift=-1pt]frame.north west)
                --([xshift=0.6cm]frame.south west)--([xshift=-13cm]frame.south east); },
            overlay first={
                \draw[colexam,line width=1pt]
                ([xshift=0.6cm, yshift=-0.5pt]frame.south
                west)--([xshift=0.6cm,yshift=-1pt]frame.north west)
                --([xshift=0.6cm]frame.south west); },
            overlay last={
                \draw[colexam,line width=1pt]
                ([xshift=0.6cm, yshift=-0.5pt]frame.south
                west)--([xshift=0.6cm,yshift=-1pt]frame.north west)
                --([xshift=0.6cm]frame.south west)--([xshift=-13cm]frame.south east); }
}
\newcounter{theorem}
\newtcolorbox[use counter=theorem]{theorem}{
    empty,
    title={Theorem ~\thetcbcounter},
    attach boxed title to top left,
    fontupper=\sl,
    boxed title style={
        empty,
        size=minimal,
        bottomrule=1pt,
        top=1pt,
        left skip=0cm,
        overlay=
            {\draw[colexam,line width=1pt]([yshift=-0.4cm]frame.north
        west)--([yshift=-0.4cm]frame.north east);}},
            coltitle=colexam,
            fonttitle=\bfseries,
            before=\par\medskip\noindent,
            parbox=false,
            boxsep=-1pt,
            left=0.75cm,
            right=3mm,
            top=4pt,
            breakable,
            pad at break*=0mm,
            vfill before first,
            overlay unbroken={
                \draw[colexam,line width=1pt]
                ([xshift=0.6cm, yshift=-0.5pt]frame.south
                west)--([xshift=0.6cm,yshift=-1pt]frame.north west)
                --([xshift=0.6cm]frame.south west)--([xshift=-13cm]frame.south east); },
            overlay first={
                \draw[colexam,line width=1pt]
                ([xshift=0.6cm, yshift=-0.5pt]frame.south
                west)--([xshift=0.6cm,yshift=-1pt]frame.north west)
                --([xshift=0.6cm]frame.south west); },
            overlay last={
                \draw[colexam,line width=1pt]
                ([xshift=0.6cm, yshift=-0.5pt]frame.south
                west)--([xshift=0.6cm,yshift=-1pt]frame.north west)
                --([xshift=0.6cm]frame.south west)--([xshift=-13cm]frame.south east); }
}
\newcounter{lemma}
\newtcolorbox[use counter=lemma]{lemma}{
    empty,
    title={Lemma~\thetcbcounter},
    attach boxed title to top left,
    fontupper=\sl,
    boxed title style={
        empty,
        size=minimal,
        bottomrule=1pt,
        top=1pt,
        left skip=0cm,
        overlay=
            {\draw[colexam,line width=1pt]([yshift=-0.4cm]frame.north
        west)--([yshift=-0.4cm]frame.north east);}},
            coltitle=colexam,
            fonttitle=\bfseries,
            before=\par\medskip\noindent,
            parbox=false,
            boxsep=-1pt,
            left=0.75cm,
            right=3mm,
            top=4pt,
            breakable,
            pad at break*=0mm,
            vfill before first,
            overlay unbroken={
                \draw[colexam,line width=1pt]
                ([xshift=0.6cm, yshift=-0.5pt]frame.south
                west)--([xshift=0.6cm,yshift=-1pt]frame.north west)
                --([xshift=0.6cm]frame.south west)--([xshift=-13cm]frame.south east); },
            overlay first={
                \draw[colexam,line width=1pt]
                ([xshift=0.6cm, yshift=-0.5pt]frame.south
                west)--([xshift=0.6cm,yshift=-1pt]frame.north west)
                --([xshift=0.6cm]frame.south west); },
            overlay last={
                \draw[colexam,line width=1pt]
                ([xshift=0.6cm, yshift=-0.5pt]frame.south
                west)--([xshift=0.6cm,yshift=-1pt]frame.north west)
                --([xshift=0.6cm]frame.south west)--([xshift=-13cm]frame.south east); }
}

\newcommand{\eps}{\varepsilon}
\usepackage[OT2,T1]{fontenc}
\DeclareSymbolFont{cyrletters}{OT2}{wncyr}{m}{n}
\DeclareMathSymbol{\Sha}{\mathalpha}{cyrletters}{"58}

\markright{Popović\hfill Seminar\hfill}


\title{University of Vienna\\
\vspace{1cm}Seminar:\\Joint RICAM Seminar\\
\vspace{0.5cm}
Summary of talk by Otmar Scherzer
}
\author{Milutin Popovic}


\usepackage[final]{pdfpages}

\begin{document}
\maketitle
\tableofcontents
\section{Sheet 3}
\subsection{Problem 1}
Take a linear system $Ax = b$, where $A \in \mathbb{R}^{n\times n}$ and $b
\in \mathbb{R}^{n}$. We want to solve it using the Gradient decent method, an
iteration
\begin{align}
    x^{k+1} = x^{k} + \alpha_k r^{k},
\end{align}
where $r^{k} = b - Ax^{k}$ and the residual $\alpha_k= \frac{(r^{k})^T
r^k}{(r^k)^T Ar^k}$.
\subsubsection{}
We compute $x^1$ for
\begin{align}
    A =
    \begin{pmatrix}
        2 & -1 \\
        -1 & 2
    \end{pmatrix},
    \qquad
        b = \begin{pmatrix}1 \\ 1 \end{pmatrix},
\end{align}
with an initial guess $x^0 = 0$.
\begin{align}
    r^0 = \begin{pmatrix} 1 & 1 \end{pmatrix} \qquad
    Ar^0 = \begin{pmatrix} 1 & 1 \end{pmatrix} \quad \Rightarrow \quad
    \alpha_0 = 1.
\end{align}
Then for $x^1$ we have
\begin{align}
    x^1 = \begin{pmatrix} 0 \\ 0 \end{pmatrix}  + \begin{pmatrix}1 \\
1 \end{pmatrix}  = b
\end{align}
\subsubsection{}
Suppose the $k$-th error $e^k = x - x^k$ is an eigenvector of $A$ to the
eigenvalue $\lambda$, then
\begin{align}
    A e^{k}=\lambda e^{k} = \lambda (x^{k}-x) = \lambda x^{k} - \lambda x.
\end{align}
For the next iteration step we need $r^{k}$ which is
\begin{align}
    r^{k} &= b - Ax^{k} + Ax - Ax = (b - Ax) - A(x^{k} - x) = -\lambda e^{k}\\
    (r^k)^{T} r^k &= \lambda^2 (e^k)^{T}e^{k}\\
    (r^k)^{T} A r^k &= \lambda^3 (e^k)^{T}e^{k}\\
    \Rightarrow \alpha_k &= \frac{1}{\lambda}
\end{align}
Then the next step $x^{k+1}$ is
\begin{align}
    x^{k+1} = x^{k}+\alpha_k r^{k} = x^{k}-\frac{\lambda}{\lambda}e^{k} =
    x^{k} - e^{k} = x^{k}-x^{k}+ x = x,
\end{align}
i.e. $x^{k+1}$ is then the solution.
\subsection{Exercise 2}
We show the norm equivalence of the vector norms $ \|\cdot\|_\infty$ and
$\|\cdot\|_2$ and the optimality of the constants $C, C' > 0$and the
optimality of the constants $C, C' > 0$
\subsubsection{}
We show
\begin{align}
    \|v\|_\infty = \max_i |v_i| \leq \sqrt{\sum_{i}^{} v_i^2}  \quad
    \Rightarrow \quad C = 1.
\end{align}
Then
\begin{align}
    \|v\|_2 = \sqrt{\sum_i v_i^2} \le \sqrt{\sum_i \max_i |v_i|^2}  =
    \sqrt{\sum_i  \|v\|^2_\infty} = \sqrt{n} \|v\|_\infty.
\end{align}
We get
\begin{align}
    \|v\|_\infty \le \|v\|_2 \le \sqrt{n} \|v\|_\infty.
\end{align}
For $u := \frac{v}{\|v\|_\infty} \; \Rightarrow \; \|v\|_\infty = 1$ we get
\begin{align}
    1 \le  \|u\|_2 \le \sqrt{n}
\end{align}
which states optimality of the constants.
\subsection{Problem 3}
Now we do the same as in Problem 2 but with matrix norms induced by vector
norms, specifically for $\|\cdot\|_2$ and $\|\cdot\|_\infty$ matrix norms
induced by vector norms.
\subsubsection{}
We know that
\begin{align}
    \|A\|_2 = \max_{v\neq 0} \frac{\|Ax\|_2}{\|x\|_2} \qquad \|A\|_2 &=
    \rho(A^*A)\\
    \|A\|_\infty = \max_{i,j}  |a_{ij}|
\end{align}
Together with the norm inequalities
\begin{align}
    \|Av\|_\infty &\le \|A\|_\infty \|v\|_\infty\\
    \|A\|_2^2 = \rho(A^TA) \le \|A^TA\|_\infty\\
    \Rightarrow \|\rho(A^TA)\|_\infty = \|A^TA x\|_\infty \le \|A^T
    A\|_\infty \|x\|_\infty,
\end{align}
thereby
\begin{align}
    \|A^T A\|_\infty &= \max_{i,j} \left|(A^T A)_{ij})\right| = \max_{i,j}
    \left| \sum_l A_{il} A_{lj} \right|  \\
                    &=\max_{i,j} \sum_l \left| A_{il} A_{lj} \right| \leq n
                    \|A\|_\infty^2\\
                    \nonumber\\
    \Rightarrow  \|A\|_2 &\leq \sqrt{n} \|A\|_\infty
\end{align}
\subsubsection{}
Let $A \in \mathbb{C}^{n\times n}$, $b \in \mathbb{C}^n$ defined as
\begin{align}
    A =
    \begin{pmatrix}
        1 & \cdots  &  1 \\
        0 & \cdots  &  0 \\
        \vdots & \vdots  &\vdots   \\
        0 & \cdots & 0
        \end{pmatrix} \qquad b = \begin{pmatrix} 1 \\ \vdots \\ 1
    \end{pmatrix}
\end{align}
Then $Ab = \begin{pmatrix} n & 0 & \cdots & 0 \end{pmatrix}^T$ and the
$\|\cdot\|_2$ of $A$ is
\begin{align}
    \|A\|_2 = \max_{v\neq 0} \frac{\|Av\|_2}{\|v\|} = \frac{\|Ab\|_2}{\|b\|}
    = \frac{n}{\sqrt{n}} = \sqrt{n}
\end{align}
\subsubsection{}
Let $A$ be defined as above, we show that $\|A\|_\infty = \sqrt{n} \|A\|_2$,
where $C = \sqrt{n} $ is optimal
\begin{align}
    \|A\|_\infty = \max_i \sum_j |A_{ij}| = n = \sqrt{n} \sqrt{n}  = \sqrt{n}
     \|A\|_2,
\end{align}
thereby $C$ is optimal.
\subsubsection{}
We show that $\|A\|_2 \le \sqrt{n} \|A\|_\infty$ for all $A \in
\mathbb{C}^{n\times n}$, and equality holds for $A$ whose columns are all
zero, accept the first one filled with ones. We know the norms
$\|\cdot\|_2$ and $\|\cdot\|_\infty$ are induced by vector norms which are
consistent. Then for all $v \in \mathbb{C}^{n}$ we have
\begin{align}
    \|v\|_\infty &\le\sqrt{n} \|v\|_2\\
    \Leftrightarrow \|A\|_\infty &\le \sqrt{n} \|A\|_2 \qquad\ \forall \ A
    \in \mathbb{C}^{n\times n}.
\end{align}
\subsection{Problem 4}
Let $\langle x, y \rangle$ be the standard Euclidean scalar product on
$\mathbb{R}^n$ and $\|\cdot\|_2$ the Euclidean norm. Let $B\in
\mathbb{R}^{n\times n}$ be an antisymmetric matrix, i.e. $B^T = -B$. And let
$A:= I - B$
\subsubsection{}
We show that for all $x \in \mathbb{R}^{n}$ we have $\langle Bx, x\rangle =0$
and $\langle Ax, x\rangle = \|x\|_2^2$.
\begin{align}
    \langle Bx, x\rangle = (Bx)^T x = x^T B^T x = -x^T B x\\
    (-x^TBx)^T &= x^TBx\; \Rightarrow\; x^T Bx = jx^TBx\\
    \Rightarrow 2x^T B x &= 2 \langle Bx, x\rangle = 0.
\end{align}
And the second statement follows from the previous equation
\begin{align}
    \langle Ax, x\rangle &= \rangle\left(I-B\right)x,x\rangle \\
                         &=\langle x, x\rangle - \underbrace{\langle Bx,
                         x\rangle}_{=0} = \langle x, x\rangle =
    \|x\|_2^2
\end{align}
\subsubsection{}
We show that $\|Ax\|_2^2 = \|x\|_2^2 + \|Bx\|_2^2$ and that $\|A\|_2 =
\sqrt{1+\|B\|_2^2}$. We start with the vector norm
\begin{align}
    \|Ax\|^2_2
    &= \|x - Bx\|_2^2 \\
    &= \langle x -Bx, x-Bx\rangle \\
    &= (x^T - x^T B^T)(x - Bx)\\
    &= x^T x - x^T B^T x - x^T Bx + x^T B^T Bx\\
    &= x^T x + x^TB^TBx = \|x\|_2^2 + \|Bx\|_2^2
\end{align}
where $i, j$ run from $1$ to $n$. Then the vector induced norm follows
\begin{align}
    \|A\|_2^2
    &= \|I-B\|_2^2 = \sup_{x\neq 0} \frac{\|Ax\|_2^2}{\|x\|^2_2}\\
    &= \sup_{x\neq 0} \frac{\|x\|^2_2 +\|Bx\|_2^2 }{\|x\|_2^2} = 1+
    \sup_{x\neq 0}\frac{\|Bx\|_2^2}{\|x\|^2_2}= 1+\|B\|_2^2,
\end{align}
pulling the square root on both sides gives us the answer.
\subsubsection{}
We show that $A$ is invertible and the inverse matrix norm is given
\begin{align}
    \|A^{-1}\|^2 = \max_{x\neq}\frac{\|x\|^2}{\|Ax\|^2}
\end{align}
To show that $A$ is invertible we use the Rank-Nullity Theorem
\begin{align}
    n &=\text{rg}(A) + \dim\text{Im}(A)\\
    \Rightarrow n - \dim\text{Im}(A) = rg(A).
\end{align}
If $A$ is invertible it has maximal rank that means $\dim \text{Im}(A) = 0$.
Let's look what the image of A is
\begin{align}
    Ax = 0 \Rightarrow \|Ax\|_2^2 = \|x\|_2^2 + \|Bx\|_2^2 = 0
\end{align}
holds only for $x = 0$, thereby $\dim \text{Im}(A) = 0$ and the rank is
maximal. Now let $A^{-1}y =x$ then
\begin{align}
    \frac{\|A^{-1}y\|_2}{\|y\|_2} &= \frac{\|x\|_2}{\|Ax\|_2}\\
    \Rightarrow \|A^{-1}\|_2 &= \max_{x\neq 0} \frac{\|x\|_2}{\|Ax\|_2}.
\end{align}
\subsubsection{}
Next we show that $\|A^{-1}\|_2 \le 1$.
\begin{align}
    \|A^{-1}\|_2 = \max_{x\neq 0} \frac{\|x\|_2}{\|Ax\|_2} =
    \frac{1}{\sqrt{1+\|B\|_2^2}} \le 1
\end{align}
\subsection{Problem 5}
We take $A$ and $B$ from last exercise.
\subsubsection{}
We show that for $k \in \{1,\ldots,m\}$ and $\mathcal{W} \subset
\mathbb{R}^n$ with $\dim \mathcal{W} = k$ spanned by $w_1,\ldots,w_k \in
\mathbb{R}^n$. We show that if $x \in \mathcal{W}$ is such that
\begin{align}
    \langle Ax, w\rangle = \langle b , w \rangle \qquad \forall w\in
    \mathcal{W}.
\end{align}
then $\|x\|_2 \le \|b\|_2$. We can choose $b = x$ then
\begin{align}
    \langle Ax, x\rangle &= \|x\|_2^2 = \langle b, x\rangle \leq
    \|b\|_2\|x\|_2.\\
    \Rightarrow \|x\|_2 \le \|b\|_2
\end{align}
\subsubsection{}
Next we show that for $x :=\sum_j c_j w_j$ and
\begin{align}
    \sum_j c_j \langle Aw_j, w_i\rangle = \langle b , w_i\rangle
\end{align}
for $i = 1, \ldots , k$, we have a unique solution for $x \in \mathcal{W}$.
We do this by showing that every solution is the 0 solution $b=0$
\begin{align}
    \langle 0 , w\rangle = 0 = \langle Ax ,x \rangle = \|x\|_2^2.
\end{align}
\subsubsection{}
Lastly for $x^* := A^{-1}b$ we show and inequality relation
\begin{align}
    \|x^* - x\|_2 \le \|A\|_2 \min_{x \in \mathcal{W}}\|x^* -w\|_2.
\end{align}
(We take the minimum $w = x$ and calculate)
\begin{align}
    \|x^* - x\|_2^2
    &= \langle A(x^* -x), x^* - x\rangle\\
    &= \langle A(x^* -x), x^* - x +w -w\rangle\\
    &= \langle A(x^* -x), x^* - w\rangle + \langle A(x^* -x), w - x\rangle\\
    &= \langle A(x^* -x), x^* - w\rangle +\underbrace{\langle A(x^* -x), w -
    x\rangle}_{=0}\\
    \le \|A\|_2 \|x^* - x\|_2\|x^* - w\|.
\end{align}
When we take the minimum and divide by $\|x^* - x\|_2$ we get
\begin{align}
    \|x^* - x\| \le \|A\|_2\min_{x\in\mathcal{W}} \|x^* - w\|_2.
\end{align}

%\printbibliography
\end{document}
