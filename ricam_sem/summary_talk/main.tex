\documentclass[a4paper]{article}

\usepackage[T1]{fontenc}
\usepackage[utf8]{inputenc}
\usepackage{mlmodern}

%\usepackage{ngerman}	% Sprachanpassung Deutsch

\usepackage{graphicx}
\usepackage{geometry}
\geometry{a4paper, top=15mm}

\usepackage{subcaption}
\usepackage[shortlabels]{enumitem}
\usepackage{amssymb}
\usepackage{amsthm}
\usepackage{mathtools}
\usepackage{braket}
\usepackage{bbm}
\usepackage{graphicx}
\usepackage{float}
\usepackage{yhmath}
\usepackage{tikz}
\usetikzlibrary{patterns,decorations.pathmorphing,positioning}
\usetikzlibrary{calc,decorations.markings}

%\usepackage[backend=biber, sorting=none]{biblatex}
%\addbibresource{uni.bib}

\usepackage[framemethod=TikZ]{mdframed}

\tikzstyle{titlered} =
    [draw=black, thick, fill=white,%
        text=black, rectangle,
        right, minimum height=.7cm]


\usepackage[colorlinks=true,naturalnames=true,plainpages=false,pdfpagelabels=true]{hyperref}
\usepackage[parfill]{parskip}
\usepackage{lipsum}

\usepackage{tcolorbox}
\tcbuselibrary{skins,breakable}

\pagestyle{myheadings}

\newcommand{\eps}{\varepsilon}
\usepackage[OT2,T1]{fontenc}
\DeclareSymbolFont{cyrletters}{OT2}{wncyr}{m}{n}
\DeclareMathSymbol{\Sha}{\mathalpha}{cyrletters}{"58}

\markright{Popović\hfill Numerical Analysis\hfill}


\title{University of Vienna\\ Faculty of Mathematics\\
\vspace{1cm}Numerical Analysis Problems
}
\author{Milutin Popovic}



\begin{document}

\maketitle

\tableofcontents

\section{Intro}
The following questions are answered:
\begin{itemize}
    \item iterative regularization with NN functions
    \item application of NNs on inverse problems
    \item What generalized NNs are best suited for IPs?
\end{itemize}

\subsection{Posing the problem}
Consider linear operator equation between Hilbert spaces $\mathbf{X}$ and
$\mathbf{Y}$
\begin{align}
    F\mathbf{x} = \mathbf{y}.
\end{align}
For the problem modeling we introduce a function, called \textbf{Coding}
$\Psi: \vec{P} \to \mathbf{X}$ which maps NN parameters to images functions,
a nonlinear operator. Our problem can be written as follows
\begin{align}
    N(\vec{p}) = F\Psi(\vec{p}) = \mathbf{y}, \label{eq: main}
\end{align}
where $\mathbf{X}$ is the image space, $\mathbf{Y}$ the data space and $\vec{P}$ the parameter
space. In the case the operator in question $F$ is nonlinear then we would of
course have a nonlinear equation, which we are not considering right now. The
talk aims to explain the link between the general regularization of the
degree of ill-posedness and nonlinearity and investigates generalized
Gauss-Newton solvers, by the outer inverse or by approximations.
\subsection{Decomposition cases (review)}
An operator $N$ satisfies the \textit{1st decomposition case} in an open empty
neighborhood $\mathcal{B}\left(\vec{p}\;^{\dagger}; \rho \right) \subseteq
\vec{P} $ (an open ball at point $\vec{p}\;^{\dagger}$ with radius $\rho$), if
there exists a linear operator $F:\vec{P}\to \mathbf{X}$ and a nonlinear
operator $\Psi:\mathbf{X} \to \mathbf{Y}$ such that.
\begin{align}
    N(\vec{p}) = \Psi(F\vec{p}).
\end{align}
The \textit{2nd decomposition case} for operator $N$ in the same setting is
satisfied, if there exists a linear operator $F: \mathbf{X} \to \mathbf{Y}$
and a nonlinear operator $\Psi: \vec{P} \to \mathbf{X}$ such that
\begin{align}
    N(\vec{p}) = F\Psi(\vec{p}).
\end{align}
\subsection{Gauss-Newton type method for 2nd decomposition case}
We are dealing with the operator $\Psi:\mathcal{D} \subseteq \vec{P} :=
\mathbb{R}^{n_*} \to \mathbf{X}$. The derivative of $\Psi$ \textbf{cannot be
invertible}!. So how do we decompose the 2nd case
\begin{align}
    N(\vec{p}) = F\Psi(\vec{p}).
\end{align}
To prove convergence we need introduce the Lipschitz-differentiable immersion.
\section{Background}
\subsection{Newton-Mysovskii}
The local convergence of the Newton method is guaranteed under the so called Newton-Mysovskii
 conditions. In this section the results are shown for the simple case in the
 finite dimensional space, when the nonlinear operator has derivative which
 are invertible. This result is going to be extended as aim of the summary.

 \begin{theorem}
     Let $N: \mathcal{D}(N) \subseteq \mathbb{R}^{n}\to \mathbb{R}^{n}$ be
     continuously Fr\'echet differentiable on a non-empty, open and convex
     set $\mathcal{D}\left( N \right) $. Let $\vec{p}\;^{\dagger} \in
     \mathcal{D}(N)$ be the solution of $N(\vec{p}\;) = \mathbf{y}$, where
     $\mathbf{y} \in \mathbb{R}^{n}$. Also assume that
     \begin{enumerate}
         \item $N'(\vec{p}\;)$ is invertible $\forall \vec{p} \in
             \mathcal{D}(n)$ and
         \item The Newton-Mysovskii condition hold, i.e. $\exists C_N > 0:$
            \begin{align}
                &\big\| N'(\vec{p}\;)^{-1}\left( N'(\vec{q} + s(\vec{p} -
                \vec{q}\;) - N'(\vec{q}\;)\right) (\vec{p} - \vec{q})
                \big\|_{\vec{P}}
                \leq s C_N \|\vec{p} - \vec{q} \;\|^{2}_{\vec{P}}\\
                & \forall \vec{p}, \vec{q} \in \mathcal{D}(N), \; s \in[0,
                1],
            \end{align}
     \end{enumerate}
     Now let $\vec{p}\;^{0} \in \mathcal{D}(N)$ the starting point of the
     Newton iteration, which should be sufficiently close to the solution,
     i.e. satisfying
     \begin{align}
         &\overline{\mathcal{B}\left(\vec{p}\;^{0}, \rho \right)}\subseteq
         \mathcal{D}(N) \qquad \text{with}\\
         &\rho := \|\vec{p}\;^{\dagger} - \vec{p}^{0}\|_{\vec{P}} \quad
         \text{and} \quad h:= \frac{\rho C_l C_L}{2} <1. \label{eq: locality}
     \end{align}
     Then the Newton iteration with starting point $\vec{p}\;^{0}$ and
     iterates $\left\{\vec{p}\;^{k}  \right\}_{k \in \mathbb{N}_0} \subseteq
     \overline{\mathcal{B}(\vec{p}\;^{0}, \rho)} $ of the
     form
     \begin{align}
         \vec{p}\;^{k+1} = \vec{p}\;^{k} - N'(\vec{p}\;^{k})^{-1}\left(
         N\left(\vec{p}\;^{k}  \right) - \mathbf{y} \right)  \qquad k \in
         \mathbb{N}_0,
     \end{align}
     converge to $\vec{p}\;^{\dagger} \in \overline{\mathcal{B}(\vec{p}\;^{0},
     \rho)}$, \textbf{quadratically}.
 \end{theorem}

\subsection{Moore-Penrose Inverse}
We study the case where $\mathbf{Y}$ is an infinite dimensional Hilbert
space. In this regard it is necessary to replace the inverse in the classical
Newton method because the liberalizations of the operator $N$ cannot be be
invertible. This is done by introducing the so called Moore-Penrose inverse
or more general the outer inverse and we refer to the Gauss-Newton method to
distinguish between the classical version.
\begin{mydef}{Inner, outer and Moore Penrose inverse
        \label{def: moore-penrose}}
    $L: \vec{P} \to \mathbf{Y}$ be a linear and bounded operator between
    vector spaces $\vec{P}$ and $\mathbf{X}$. Then
    \begin{enumerate}
        \item $B: \mathbf{Y} \to \vec{P}$ is called \textbf{left-inverse} to
            $L$ if
            \begin{align}
                BL = I
            \end{align}
        \item $B: \mathbf{Y} \to \vec{P}$ is called \textbf{right-inverse} to
            $L$ if
            \begin{align}
                LB = I
            \end{align}
        \item $B: \vec{P} \to \vec{P}$ is called an \textbf{inverse} to
            $L$ if it is both a left and a right inverse to $L$.
        \item $B: \vec{P} \to \vec{P}$ is called an \textbf{outer inverse} to
            $L$ if
            \begin{align}
                BLB = B
            \end{align}
        \item Let $\vec{P}$ and $\mathbf{Y}$ be Hilbert spaces, $L: \vec{P}
            \to \mathbf{Y}$ be linear bounded operator. Denote the
            orthogonal projections $P$ and $Q$ onto the nullspace of $L$,
            $\mathcal{N}(L)$ closed and the closure of the range of $L$,
            $\overline{\mathcal{R}\left(L  \right)} $. This means that for all $\vec{p}
            \in \vec{P}$ and $\mathbf{y} \in \mathbf{Y}$ we have
            \begin{align}
                &P\vec{p} = \text{argmin}
                \left\{
                    \|\vec{p}_1-\vec{p}\|_{\vec{P}} : \vec{p}_1 \in
                \mathcal{N}(L) \right\},\\
                &Q\mathbf{y} = \text{argmin}
                \left\{
                    \|\mathbf{y}_1 - \mathbf{y}\|_\mathbf{Y}: \mathbf{y} \in
                    \overline{\mathcal{R}(L)} \right\}
            \end{align}
            Then the operator $B: \mathcal{D}(B) \subseteq \mathcal{Y} \to
            \vec{P}$ with $\mathcal{B}(B):= \mathcal{R} \dotplus
            \mathcal{R}^{\perp}$ is called \textbf{Moore-Penrose inverse} of
            $L$ if the following conditions(identities) hold
            \begin{align}
                &LBL = L, \nonumber\\
                &BLB = B, \nonumber\\
                &BL= I-P, \label{eq: moore-penrose}\\
                &LB = Q|_{\mathcal{D}(B)} \nonumber.
            \end{align}

    \end{enumerate}
    The left and right inverses are used in a different context. For a left
    inverse the nullspace of $L$ has to be trivial, in contrast to $B$.
    On the other hand for the right inverse the nullspace of $B$ needs to be
    trivial.


\end{mydef}

\subsection{Lipschitz-differentiable immersion}
\begin{mydef}{Lipschitz-differentiable immersion}
    Let there be $n_* = N*(n+2)$ neural nets depending on the parameters
    $(\vec{\alpha}, \mathbf{w}, \vec{\theta})$. Let $\Psi'$ be
    Lipschitz-continuous and
    \begin{align}
        \mathbf{X}_{\vec{p}} =
        \text{span}\{\partial_{p_i}\Psi(\vec{p})\;:\;i=1,\ldots,n_*\},
        \label{eq: lpdi-property}
    \end{align}
    has $\text{rank}(n_*)$.
    And let $\Psi'(\vec{p})^{\dagger}$ denote a generalized inverse,
    which replaces the standard $\Psi^{-1}$ in the standard Newton's method.
    TODO: more in detail definition.
\end{mydef}
We link the inverse of the Lipschitz Lipschitz-differentiable immersion with
the Moore-Penrose inverse together with the necessary boundary constraints
for the Gauss-Newton method
\begin{theorem}
    \label{thm: moore-penrose}
    \begin{enumerate}
        \item The function $\Psi'(\vec{p}\;)^{\dagger}: \mathbf{X} \to \vec{P}$
            is the Moore-Penrose inverse of $\Psi'(\vec{p}\;)$.
        \item For all $\vec{p} \in \mathcal{D}(\Psi) \subseteq \vec{P}$
            there is a non-empty closed neighborhood where
            $\Psi'(\vec{p})^{\dagger}$ is uniformly bounded and it is
            Lipschitz-continuous, i.e.
            \begin{align}
                &\|\Psi'(\vec{p}\;)^{\dagger} -
                \Psi'(\vec{q}\;)^{\dagger}\|_{\mathbf{X}\to\vec{P}}
                \leq C_L \|\vec{p} - \vec{q}\;\|_{\vec{P}}&\\
                &\|\Psi'(\vec{p}\;)^{\dagger}
                \|_{\mathbf{X}\to\vec{P}} \leq C_l\qquad &\text{for}\;\;
                \vec{p}, \vec{q}\in \mathcal{D}(\Psi).
            \end{align}
    \item The operator $P_{\vec{p}}: \mathbf{X} \to \mathbf{X}_{\vec{p}}$ is
            bounded
    \end{enumerate}
\end{theorem}
\begin{proof}
    The proof can be found in the appendix \ref{proof: thm-moore-penrose}
\end{proof}

We can now wrap the results back to the original problem of the Gauss-Newton
iteration of \ref{eq: main}
\begin{lemma}
    \label{lem: moore-penrose}
    Let $F$ be as in \ref{eq: main} linear, bounded with trivial nullspace
    and dense range (for the Moore-Penrose inverse). Let $\Psi:
    \mathcal{D}(\Psi)\subseteq \mathbb{R}^{n_*} \to \mathbf{X}$ be a
    Lipschitz-differentiable immersion. And $N = F\circ \Psi$ \ref{eq: main}.
    Here it is important to see the immanent result that for $N$,
    $\mathcal{D}(N) = \mathcal{D}(\Psi)$, therefore $\forall \vec{p} \in
    \mathcal{D}$ the derivative of the operator $N$ has a Moore-Penrose
    inverse $N'(\vec{p}\;)^{\dagger}$, satisfying
    \begin{enumerate}
        \item The decomposition property of the Moore-Penrose inverse
            \begin{align}
                N'(\vec{p}\;)^{\dagger}\mathbf{z} =
                \Psi'(\vec{p}\;)^{\dagger}F^{-1}\mathbf{z} \qquad \forall
                \vec{p} \in \mathcal{D}(N),\; \mathbf{z} \in \mathcal{R}(F)
                \subseteq \mathbf{Y}
            \end{align}
            which means that
            \begin{align}
                &N'(\vec{p}\;)^{\dagger}N'(\vec{p}\;) = I \quad \text{on} \;\;
                \mathbb{R}^{n_*} \qquad \text{and}\\
                &N(\vec{p}\;)N'(\vec{p}\;)^{\dagger} =
                \mathcal{Q}|_{\mathcal{R}(FP_{\vec{p}})},
            \end{align}
            where $\mathcal{Q} : \mathbf{Y} \to
            \overline{\mathcal{R}(FP_{\vec{p}})}\dotplus\mathcal{R}(FP_{\vec{p}})^{\perp}$.
            So in the definition of the Moore-Penrose
            \ref{def: moore-penrose},
            $P$ in \ref{eq: moore-penrose} is $P \equiv 0$.
        \item The generalized Newton-Mysovskii conditions are also satisfied
            \begin{align}
                &\big\| N'(\vec{p}\;)^{\dagger}\left( N'(\vec{q} + s(\vec{p} -
                \vec{q}\;) - N'(\vec{q}\;)\right) (\vec{p} - \vec{q})
                \big\|_{\vec{P}}
                \leq s C_l C_L \|\vec{p} - \vec{q} \;\|^{2}_{\vec{P}}\\
                & \forall \vec{p}, \vec{q} \in \mathcal{D}(N), \; s \in[0,
                1],
            \end{align}
            where $C_l, C_L$ are the Lipschitz-constants.
    \end{enumerate}
\end{lemma}
\begin{proof}
    The proof can be found in the appendix \ref{proof: lem-moore-penrose}
\end{proof}

Bringing it all together to show the local convergence of the Gauss-Newton
method, where $N = F\circ \Psi$ is a composition of a linear bounded operator
and a Lipschitz-differentiable immersion.
\begin{theorem}
    \label{thm: gauss-newton-convergence}
    Assume there exists a $\vec{p}\;^{\dagger} \in \mathcal{D}(\Psi)$ which
    satisfies
    \begin{align}
        N(\vec{p}\;^{\dagger}) = \mathbf{y},
    \end{align}
    and assume there exists a $\vec{p}\;^{0} \in \mathcal{D}(\Psi)$ as in
    \ref{eq: locality}, satisfying locality, Then the iterates
    $\{\vec{p}\;^{k}\}_{k\in \mathbb{N}_0}$  of the Gauss-Newton method of
    the form
     \begin{align}
         \vec{p}\;^{k+1} = \vec{p}\;^{k} - N'(\vec{p}\;^{k})^{\dagger}\left(
         N\left(\vec{p}\;^{k}  \right) - \mathbf{y} \right)  \qquad k \in
         \mathbb{N}_0, \label{eq: gauss-newton-iteration}
     \end{align}
     are well-defined in $\overline{\mathcal{B}\left(\vec{p}\;^{0}, \rho
     \right) }$ and converge quadratically to $\vec{p}\;^{\dagger}$.
\end{theorem}
\begin{proof}
    The proof can be found in the appendix \ref{proof: thm-gauss-newton-convergence}
\end{proof}


\subsection{Neural networks}
Shallow neural network coders are of the following form
\begin{align}
    \Psi:
    \mathcal{D}(\Psi) := \mathbb{R}^{n_*} =
    \mathbb{R}^{N}\times \mathbb{R}^{n \times N}
    \times \mathbb{R}^{N}
    &\to \mathbf{X} :=
    L^{2}\left([0, 1]^{n}\right),\\
    \vec{p} = (\vec{\alpha}, \mathbf{w}, \vec{\theta}) &\mapsto
    \left(\vec{x} \to \sum_{j=1}^{N} \alpha_j\sigma\left(
    \vec{\mathbf{w}}_j^{T}\vec{x} + \omega_j \right)  \right),
\end{align}q
where $\sigma$ is an activation function, such as tanh or sigmoid.

A standard deep neural network (DNN) with $L$ layers is a function depending on $\vec{x} \in
\mathbb{R}^{n}$ with parameters $\vec{p}:=\left( \vec{\alpha}_l,
\mathbf{w}_l, \vec{\theta}_l  \right)_{l=1}^{L}$
\begin{align}
    \vec{x}\to\Psi(\vec{x}) := \sum_{j_L=1}^{N_L} \alpha_{j_L,L}\sigma_L\
    \left( p_{j_L, L} \left( \sum_{j_{L-1}=1}^{N_{L-1}}\cdots
    \left( \sum_{j_1=1}^{N_1}\alpha_{j_1,1}\sigma_1\left(p_{j_1,1}(\vec{x})
    \right)  \right)  \right)  \right),
\end{align}
where
\begin{align}
    p_{j_l}(\vec{x}) = \mathbf{w}_{j, l}^{T}\vec{x} + \theta_{j,l},
\end{align}
with $\alpha_{j,l}, \theta_{j,l} \in \mathbb{R}$ and $\vec{x},
\mathbf{w}_{j,l} \in \mathbb{R}^{n} \;\; \forall l=1,\ldots,L$. And is
probably not a Lipschitz-continuous immersion!


The solution involves either reconstructing the function or the coefficient use
Tikhonov regularization or use newton type methods, the talk explains the
solution for decomposable operators wrt. the 2nd decomposition case for
Gauss-Newton type methods.

Using variational methods, Tikhonov regularization (some background on this
here)
\begin{align}
    \|N(\vec{p}) - \mathbf{y}\|^{2} + \alpha \|\vec{p}\|^{2} \to \min,
\end{align}
or alternatively state space regularization (some background on this)
\begin{align}
    \|N(\vec{p}) - \mathbf{y}\|^{2}
    + \alpha \|\mathbf{x} - \mathbf{x}_0\|^{2}
    \to \min \quad \text{s.t} \quad \Psi(\vec{p}) = \mathbf{x}.
\end{align}
Alternatively use iterative methods, Newton's iteration would look like the
following
\begin{align}
    \vec{p}\;^{k+1} = \vec{p} - N'\left(p^{-k}\right)^{-1}\left(N(\vec{p}) -
    \mathbf{y}  \right),
\end{align}
where $N'$ is the Jacobian.

Usually it is assumed that the nonlinear operator $\Psi$ is well-posed.
Here we need to see B. Hofmann On the degree of ill-posedness of nonlinear
problems. Where we assume that the nonlinear operator $\Psi$ is well-posed.


\section{Newton's method with the neural network operator}
In this section the main results of the talk are explained. The aim is to use
the universal approximation properties of neural networks, the fact that they
can approximate continuous functions arbitrarily well, to the inverse problem
in \ref{eq: main} using the Gauss-Newton method. To ensure convergence it is
necessary to show that the considered neural network structure is a
Lipschitz-differentiable immersion. As it will be shown, a direct implication
of this is to show that the among other the activation function, its
derivative and its first moment of the derivative are linearly independent.
For this, results from \cite{lamperski_2022} are used and it is conjectured
that the statement from \ref{eq: lpdi-property} is fulfilled, meaning that
the function from $\mathbf{X}_{\vec{p}}$ are linearly independent.
\newline
Convergence is based on the immersion property of the network functions
\begin{align}
    \text{span}\{\partial_{p_i}\Psi(\vec{p})\;:\;i=1,\ldots,n_*\}, \qquad
    \text{has rank}(n_*).
\end{align}
To show the Newton-Mysovskii conditions for neural network functions the
notation.
\begin{align}
    \vec{p} := (\vec{\alpha}, \mathbf{w}, \vec{\theta}) \in
    \mathbb{R}^{N}\times \mathbb{R}^{n\cdot N} \times \mathbb{R}^{N} =
    \mathbb{R}^{n_*}.
\end{align}

For $\alpha_i \neq 0$, this, in particular, requires that the functions
\begin{align}
    & \frac{\partial \Psi}{\partial \alpha_s} =\sigma(\rho),\quad
     \frac{\partial \Psi}{\partial w_s^{t}} =\sigma'(\rho)x_t,\quad
     \frac{\partial \Psi}{\partial \theta_s} =\sigma'(\rho),\\
    & \text{where} \qquad
    \rho = \sum_{i=1}^{n}w_s^{i}x_i + \theta_s, \\
\end{align}
assuming that the activation function is continuously differentiable,
are \textbf{linearly independent} and that $\alpha_s \neq 0$ -
\textbf{sparse} coefficients cannot be recovered.
\subsection{Linear independence problem}
The question is if
\begin{align}
    \frac{\partial \Psi}{\partial \alpha_s} ,
    \frac{\partial \Psi}{\partial w_s^{\dagger}} ,
    \frac{\partial \Psi}{\partial \theta_s} \label{eq: linear_indep}
\end{align}
Partial answer for $\frac{\partial \Psi}{\partial \alpha_s} (\vec{x}) =
\sigma\left( \sum_{i=1}^{n} w_s^{i}x_i + \theta_s \right)$ in the
\cite{lamperski_2022} theorem:
\begin{theorem}
    \label{thm: lamperski}
    For all activation functions \textit{HardShrink, HardSigmoid, HardTanh,
    HardSwish, LeakyReLU, PReLU, ReLU, ReLU6, RReLU, SoftShring, Threshold,
    LogSigmoid, Sigmoid, SoftPlus, Tanh and TanhShring and the PyTorch
    functions CELU, ELU, SELU} the shallow neural network functions formed by
    \textbf{randomly generated vectors} $(\mathbf{w}, \vec{\theta})$ are
    \textbf{linearly independent}.
\end{theorem}
But here it is also needed more that the first derivative of the sigmoid functions and
the first moment of the first derivative together with the above result are
linearly independent. But the answer is not satisfactory because its not
known. More or less with a probability $1$ it can be proven that the functions
above are linearly independent.

For the sigmoid function we have some symmetries because
\begin{align}
    \sigma'(\mathbf{w}^{T}_j \vec{x} + \theta_j)
    = \sigma'\left(-\mathbf{w}_j^{T}\vec{x} - \theta_j  \right)
\end{align}
or in another formulation
\begin{align}
    \Psi'(\vec{\alpha}, \mathbf{w}, \vec{\theta}) = \Psi'(\vec{\alpha},
        -\mathbf{w}, -\vec{\theta})
\end{align}
Conjecture: obvious symmetries = "random set" from \cite{lamperski_2022}. It
is conjectured that the above functions in \ref{eq: linear_indep} are
linearly independent and the results are build from here on out.

\begin{conjecture}
    \label{conj: main}
    Assume that the functions from \ref{eq: linear_indep} are locally
    linearly independent.
\end{conjecture}
From this it follows that the shallow network coders are a
Lipschitz-differentiable immersion (for a suitable choice of an activation
function), so Gauss-Newton method converges locally.
\subsection{Gauss-Newton iteration with coding networks}
A direct implication of the local convergence of the Gauss-Newton method is
the immersion property. In this section the proof of the
Lipschitz-differentiable immersion for the shallow neural network coders and
the convergence of the Gauss-Newton method are going to be summarized.
\begin{lemma}
    \label{lemma: immersion}
    Let the activation function $\sigma$ be one of the function from
    \ref{thm: lamperski} \cite{lamperski_2022}. Let $F:\mathbf{X}=L^{2}\left(
    [0, 1]^{n}\right) \to \mathbf{Y}$ be linear, bounded and with trivial
    nullspace and closed range. Assume the Conjecture \ref{conj: main}
    holds, then
    \begin{itemize}
        \item $\forall \vec{p} = (\vec{\alpha}, \mathbf{w}, \vec{\theta}) \in
            \mathbb{R}^{n_*} \subseteq \mathcal{D}(\Psi)$,
            $\mathcal{R}(D\Psi[\vec{p}\;])$ is a linear
            subspace of $\mathbf{X}$ of dimension $n_*$
        \item There exists an open neighborhood $\mathcal{U} \subseteq
            \mathbb{R}^{n_*}$ of vectors $\vec{p}$ s.t. $\Psi$ is a
            Lipschitz-differentiable immersion in $\mathcal{U}$.
    \end{itemize}
\end{lemma}
\begin{proof}
    The proof can be found in the appendix \ref{proof: lem-immersion}
\end{proof}


The finial results states the local convergence of the Gauss-Newton method
for shallow network coders.
\begin{theorem}
    Using the same setup and assumptions as above, additionally let $\vec{p}
    \in \mathbf{D}(\Psi)$ be a starting point of the Gauss-Newton iteration
    \ref{eq: gauss-newton-iteration} and $\vec{p}\;^{\dagger} \in
    \mathcal{D}(\Psi)$ the solution of equation \ref{eq: main}. Then the
    Gauss-Newton method converges quadratically if
    $\vec{p}\;^{0}$ is sufficiently close to $\vec{p}\;^{\dagger}$.
\end{theorem}
\begin{proof}

    Trivially applying Lemma \ref{lemma: immersion} to Theorem
    \ref{thm: gauss-newton-convergence}.
\end{proof}




\section{Results}
\subsection{Numerical results(simplified)}
The simplification is
\begin{align}
    &N = F \circ \Psi \\
    &\mathbf{y}^{\dagger} = F\Psi(\vec{p}\;^{\dagger}) \qquad \text{is
    attainable}
\end{align}
Then the Gauss-Newton method is
\begin{align}
    \vec{p}\;^{k+1} = \vec{p}\;^{k} - \Psi'\left(\vec{p})\;^{k}  \right)^{\dagger}
    \left( \Psi(\vec{p}\;^{k} - \Psi^{\dagger} \right) \qquad k \in
    \mathbb{N}_0.
\end{align}
Do some numerical  results or explain the ones in the talk.
\subsection{Landweber iteration}
Instead of the Gauss-Newton iteration we consider the Landweber iteration
\begin{align}
    \vec{p}\;^{k+1} = \vec{p}\;^{k} - \lambda \Psi'\left(\vec{p}\;^{k})  \right)^{\dagger}
    \left( \Psi(\vec{p}\;^{k} - \Psi^{\dagger} \right) \qquad k \in
    \mathbb{N}_0.
\end{align}
Needs about 500 iterations
\subsection{The catch}
If the observed convergence rate of the Gauss-New ton change completely if the
solution is not attainable. Then the conjecture is that the non-convergence
because of multiple solutions.
Also the implementation of the simplified Gauss-Newton requires inversion of
$F$ , which is not done in practice, this is for Landweber.

\subsection{Alternative to DNNs}
Instead of using Deep Neural Networks where we do not know the result if the
the immersion is invertible, we consider Quadratic neural network functions
defined as follows
\begin{align}
    \Psi(\vec{x}) := \sum_{j=1}^{N} \alpha_j\sigma\left(\vec{x}^{T}A_j\vec{x}
        + \mathbf{w}_j^{T}\vec{x} + \theta_j \right),
\end{align}
with $\alpha_j, \theta_j \in \mathbb{R}, \mathbf{w}_j \in \mathbb{R}^{n}$
and $A_j \in \mathbb{R}^{n \times n}$. We can also constrain the class of
$A_j$ and $\mathbf{w}_j$ which leads us to circular networks, circular
affine, elliptic, parabolic...
\begin{theorem}
    Quadratic neural network functions satisfy the universal approximation
    property.
\end{theorem}
The immersion property of circular network functions
\begin{align}
    \Psi(\vec{x}) := \sum_{j=1}^{N} \alpha_j\sigma\left(r_j\vec{x}^{T}\vec{x}
        + \mathbf{w}_j^{T}\vec{x} + \theta_j \right),
\end{align}
and
\begin{align}
    \text{span}\{\partial_{p_i}\Psi(\vec{p})\;:\;i=1,\ldots,n_*\}, \qquad
    \text{has rank}(n_*).
\end{align}
For $\alpha_i \neq 0$, this in particular requires that the functions
\begin{align}
    \frac{\partial \Psi}{\partial r_s}  = \sigma\left( \rho \right)
    \vec{x}^{T}\vec{x}, \quad
    \frac{\partial \Psi}{\partial \alpha_s}  = \sigma\left( \rho \right)
    , \quad
    \frac{\partial \Psi}{\partial w_s^{t}}  = \sigma'\left( \rho \right)
    x_t,\quad
    \frac{\partial \Psi}{\partial \theta_s}  = \sigma'\left( \rho \right)
\end{align}
are \textbf{linearly independent}.
\newline

Results of these types of networks is that the Shepp-Logan can be represented
with \textbf{10 nodes} with elliptic neurons and \textbf{one layer}. Where as
for affine networks, both shallow and deep we need infinity neurons. Here
figure tensorflow approximations of a circle 15 neurons linear.

\appendix
\section{Proofs}
The following section is dedicated to fill the proofs to the theorems and
lemmas in the main text.
\begin{myproof}{Theorem \ref{thm: moore-penrose}}
    \label{proof: thm-moore-penrose}
    \begin{enumerate}
        \item To show that $\Psi'(\vec{p}\;)^{\dagger}$ is indeed the
            Moore-Penrose inverse we need to verify the four identities given
            in \ref{eq: moore-penrose}. Aligning the notation in the
            difinition of the Moore-Penrose inverse then
            \begin{align}
                &L = \Psi'(\vec{p}\;) : \vec{P}=\mathbb{R}^{n_*} \to
                \mathbf{X}, \\
                & B = \Psi'(\vec{p}\;)^{\dagger}: \mathbf{X} \to \vec{P}, \\
                &\mathcal{D}(B) = \mathcal{D}(\Psi'(\vec{p}\;)^{\dagger}) =
                \mathbf{X}, \\
                &P: \vec{P} \to \vec{P} \qquad \text{the zero operator},\\
                &Q = P_{\vec{p}}:\mathbf{X} \to \mathbf{X}_{\vec{p}}
                \qquad \text{the projection operator to }
                \mathbf{X}_{\vec{p}}.
            \end{align}
            For the third identity with $P = 0$, let $\vec{q} =
            (q_i)_{i=1}^{n_*} \in \vec{P}$ and we have
            \begin{align}
                \Psi'(\vec{p}\;)^{\dagger}\Psi'(\vec{p}\;)\vec{q} &=
                \Psi'(\vec{p}\;)^{-1}\left( \sum_{i=1}^{n_*} q_i
                \partial_{p_i} \Psi(\vec{p}\;) \right) \\
              &=\left(q_i  \right)_{i=1}^{n_*}  \\
              &= \vec{q}.
            \end{align}
            The fourth identity we use the fact that $(\partial_{p}
            \Psi(\vec{p}\;))_{i=1}^{n_*}$ is a basis, so for all $\mathbf{x}
            = (\mathbf{x_1}, \mathbf{x_2}) \in \mathbf{X}$ there exists an
            $x_i$, $i \in \{1,\ldots,n_*\}$ such that we can represent any
            $\mathbf{x}$ through
            \begin{align}
                \mathbf{x} = \sum_{i=1}^{n_*}
                x_i\partial_{p_i}\Psi(\vec{p}\;) + \mathbf{x_2},
            \end{align}
        with $\mathbf{x}_2 \in \mathbf{X}^{\perp}_{\vec{p}}$. Meaning that
        \begin{align}
            P_{\vec{p}}\;\mathbf{x} = \sum_{i=1}^{n_*} x_i\partial_{p_i}
            \Psi'(\vec{p}\;) \qquad \text{and},\\
            \Psi'(\vec{p}\;)^{\dagger}\mathbf{x} = (x_i)_{i=1}^{n_*}=\vec{x}.
        \end{align}
        Then the identity falls down to
        \begin{align}
            \Psi'(\vec{p}\;)\Psi'(\vec{p}\;)^{\dagger}\mathbf{x} =
            \Psi'(\vec{p}\;)\vec{x} = P_{\vec{p}}\;\vec{x}
        \end{align}
        For the second identity, use the results from the first and the
        fourth identity then fact that $\forall \mathbf{x} \in
        \mathbf{X}$:
        \begin{align}
            \Psi'(\vec{p}\;)^{\dagger}\Psi'(\vec{p}\;)
            \Psi'(\vec{p}\;)^{\dagger}\mathbf{x}
            &= \Psi'(\vec{p}\;)^{\dagger}
            \Psi'(\vec{p}\;)\vec{x} \\
            &= \vec{x}\\
            &= \Psi(\vec{p}\;)^{\dagger} \mathbf{x}.
        \end{align}
        For the first identity using the previous results $\forall \vec{q}
        \in \vec{P}$:
        \begin{align}
            \Psi'(\vec{p}\;)
            \Psi'(\vec{p}\;)^{\dagger}
            \Psi'(\vec{p}\;)
            &= \Psi'(\vec{p}\;) \vec{q}\\
            (&= L).
        \end{align}

        \item For the boundary constraint
    \end{enumerate}
\end{myproof}

\begin{myproof}{Lemma \ref{lem: moore-penrose}}
    \label{proof: lem-moore-penrose}
    \begin{enumerate}
        \item
            First of all by the chain rule it holds that
            \begin{align}
                N'(\vec{p}\;) = F\Psi'(\vec{p}\;) \qquad
                \forall \vec{p} \in \mathcal{D}(\Psi) = \mathcal{D}(N).
            \end{align}
            To prove the decomposition property of the Moore-Penrose
            inverse it is necessary to verify the equations
            \ref{eq: moore-penrose} defining it. With the notation
            \begin{align}
                &L:= N'(\vec{p}\;) = F\Psi'(\vec{p}\;): \vec{P} \to
                \mathbf{Y} \\
                &B:= \Psi'(\vec{p}\;)^{\dagger}F^{-1} : \mathcal{R}(F)
                \subseteq \mathbf{Y} \to \vec{P}.
            \end{align}
            Also note that by assumption $F$ has a dense range and consider
            $B$ on $\mathcal{R}(F) \dotplus
            \underbrace{\mathcal{R}(F)^{\perp}}_{=\{0\}}$. Thereby from for
            a fixed $\vec{p}$ the notation of \ref{eq: moore-penrose} is
            \begin{align}
                &\mathcal{D}(B) =
                \mathcal{D}(\Psi'(\vec{p}\;)^{\dagger}F^{-1}) =
                \mathcal{R}(F),\\
                &\mathcal{R}(L) = \big\{F\Psi'(\vec{p}\;)\vec{q}\;:\;\vec{q}
                \in \mathbb{R}^{n_*} \big\} = \mathcal{R}(FP_{\vec{p}}).
            \end{align}
            Note that $Q$ is an orthogonal projection onto
            $\overline{\mathcal{R}(L)}$, i.e.
            $\overline{\mathcal{R}(FP_{\vec{p}})}$, then for
            \begin{align}
                \mathbf{z} &= F\mathbf{x} \\
                    &= FP_{\vec{p}}\;
                    \mathbf{x} + F\left(I-P_{\vec{p}} \right) \mathbf{x}
            \end{align}
            we have that
            \begin{align}
                Q\mathbf{z} &= Q\left(FP_{\vec{p}}\;\mathbf{x}
                   + F(I-P_{\vec{p}})(\mathbf{x})\right) \\
                &= FP_{\vec{p}}\; \mathbf{x}.
            \end{align}
            Finally applying Lemma \ref{thm: moore-penrose} and the
            invertability of $F$ on the range of $F$ shows that
            \begin{align}
                LBL &= F\Psi'(\vec{p}\;)\Psi'(\vec{p}\;)^{\dagger}
                F^{-1}F\Psi'(\vec{p}\;) \\
                    &=
                    F\Psi'(\vec{p}\;)
                    \Psi'(\vec{p}\;)^{\dagger}\Psi'(\vec{p}\;) \nonumber\\
                    &= F \Psi'(\vec{p}\;) \nonumber \\
                    &= L \nonumber \\
                    \nonumber \\
                BLB &= \Psi'(\vec{p}\;)^{\dagger}F^{-1}F\Psi'(\vec{p}\;)
                \Psi'(\vec{p}\;)^{\dagger}F^{-1} \\
                    &=
                    \Psi'(\vec{p}\;)^{\dagger}
                    \Psi'(\vec{p}\;)\Psi'(\vec{p}\;)^{\dagger}F^{-1} \nonumber\\
                    &= \Psi'(\vec{p}\;)^{\dagger}F^{-1} \nonumber \\
                    &= B \nonumber \\
                    \nonumber \\
                BL &= \Psi'(\vec{p}\;)^{\dagger}F^{-1}F\Psi'(\vec{p}\;) \\
                   &= I - P \\
                   &= I \quad \text{on }  \mathbb{R}^{n_*}.
            \end{align}
            The fourth identity is proven by taking a $\mathbf{z} \in
            \mathcal{R}(F)$, so there exists a $\mathbf{x} \in \mathbf{X}$
            such that $F\mathbf{x} = \mathbf{z}$, then it follows
            \begin{align}
                LB\mathbf{z} &=
                F\Psi'(\vec{p}\;)\Psi'(\vec{p}\;)^{\dagger}
                F^{-1}\mathbf{z} \\
                 &= F\Psi'(\vec{p}\;)\Psi'(\vec{p}\;)^{\dagger}
                     \mathbf{x} \\
                &= FP_{\vec{p}}\mathbf{x} \\
                &= Q\mathbf{z}.
            \end{align}
        \item As for the generalized Newton-Mysovskii conditions, the
            operator $\Psi'(\vec{p})$ is locally bounded and locally
            Lipschitz continuous on $\mathcal{D}(\Psi)$, meaning that there are
            constants $C_L, C_l > 0$ such that $\forall \vec{p}, \vec{q} \in
            \mathcal{D}(\Psi)$ the following inequalities hold
            \begin{align}
                &\big\|\Psi'(\vec{p}\;) - \Psi'(\vec{q}\;)
                \big\|_{\vec{P}\to\mathbf{X}} \leq C_L \|\vec{p} - \vec{q} \|
                \\
                &\big\|\Psi'(\vec{p}) \big\|_{\vec{P} \to \mathbf{X}} \le
                C_l.
            \end{align}
            Thus
            \begin{align}
                &\Big\| N'(\vec{p}\;)^{\dagger}\left( N'(\vec{q} + s(\vec{p} -
                \vec{q}\;) - N'(\vec{q}\;)\right)
                (\vec{p} - \vec{q}) \Big\|_{\vec{P}} \\
                &= \Big\|
                \Psi'(\vec{p}\;)^{\dagger}F^{-1}\left( F\Psi'(\vec{q}\;
                + s\left( \vec{p} - \vec{q}\; \right)-F\Psi'(\vec{q}\;)
            \right)) \left(\vec{p} - \vec{q}  \right) \Big\|_{\vec{P}}
            \nonumber \\
                &= \Big\|
                \Psi'(\vec{p}\;)^{\dagger}\left(\Psi'(\vec{q}\;
                + s\left( \vec{p} - \vec{q}\; \right)-F\Psi'(\vec{q}\;)
            \right)) \left(\vec{p} - \vec{q}  \right) \Big\|_{\vec{P}}
            \nonumber\\
                &\leq C_lC_L s \|\vec{p} - \vec{q} \|^{2}_{\vec{P}},
                \nonumber
            \end{align}
        for all $\vec{p}, \vec{q} \in \mathcal{D}(\Psi) = \mathcal{D}(N)$.

    \end{enumerate}

\end{myproof}

\begin{myproof}{Theorem \ref{thm: gauss-newton-convergence}}
    \label{proof: thm-gauss-newton-convergence}
    The distance to the solution and the first iterate is $\rho =
    \|\vec{p}\;^{\dagger} - \vec{p}\;^{0}\|$. Also $\mathcal{D}(\Psi) =
    \mathcal{D}(N)$ since $F$ is defined all over $\mathcal{X}$. First of all
    we have to prove that all iterates lie in the closed ball, i.e.
    $\vec{p}\;^{k} \in \overline{\mathcal{B}(\vec{p}\;^{\dagger}; \rho)}$.,
    this is done by induction. The case $k=0$ is fulfilled by the existence assumption.
    For $\vec{p}\;^{k}$ use the equations \ref{eq: moore-penrose}
    \begin{align}
        &N'(\vec{p}\;^{k})N'(\vec{p}\;^{k})^{\dagger}N'(\vec{p}\;^{k})
        \left(\vec{p}\;^{k+1} - \vec{p}\;^{\dagger}  \right)\\
        &= N'(\vec{p}\;^{k})(\vec{p}\;^{k+1} - \vec{p}\;^{k}).
    \end{align}
    Then the definition of the Gauss-Newton method
    \ref{eq: gauss-newton-iteration} implies
    \begin{align}
        &N'(\vec{p}\;^{k})(\vec{p}\;^{k+1}-\vec{p}\;^{\dagger}) =\\
        &=N'(\vec{p}\;^{k})N'(\vec{p}\;^{k})^{\dagger}\left(
        N(\vec{p}\;^{\dagger})-N(\vec{p}\;^{k})
        - N'(\vec{p}\;^{k})(\vec{p}\;^{\dagger} - \vec{p}\;^{k})\right) .
    \end{align}
    Now using the third identity in \ref{eq: moore-penrose}, by the assumption
    of this theorem $P = 0$ and using the second identity in
    \ref{eq: moore-penrose} with the fact that $F$ is injective, it follows
    that
    \begin{align}
        \vec{p}\;^{k+1} - \vec{p}\;^{\dagger}
        &=
        N'(\vec{p}\;^{k})^{\dagger}N'(\vec{p}\;^{k})(\vec{p}\;^{k+1} -
        \vec{p}^{\dagger}) \\
        &=
        N'(\vec{p}\;^{k})^{\dagger}\left(N(\vec{p}\;^{\dagger})-N(\vec{p}\;^{k})
        - N'(\vec{p}\;^{k})(\vec{p}\;^{\dagger} - \vec{p}\;^{k}) \right) \\
        &= \Psi'(\vec{p}\;^{k})^{\dagger}\left(\Psi(\vec{p}\;^{\dagger}-
        \Psi(\vec{p}\;^{k}) -
    \Psi'(\vec{p}\;^{k})(\vec{p}\;^{\dagger}-\vec{p}\;^{k}) \right).
    \end{align}
    Finally the Newton-Mysovskii conditions give use
    \begin{align}
        \|\vec{p}\;^{k+1} - \vec{p}\;^{\dagger}\|
        &\le \frac{C_lC_L}{2}
        \|\vec{p}\;^{k} - \vec{p}\;^{\dagger}\|^{2}_{\vec{P}} \\
        &\le \frac{C_lC_L\rho}{2}
        \|\vec{p}\;^{k}-\vec{p}\;^{\dagger}\|_{\vec{P}} \\
        &< \|\vec{p}\;^{k} - \vec{p}\;^{\dagger}\|_{\vec{P}},
    \end{align}
    meaning that $\vec{p}\;^{k+1} \in \mathcal{B}(\vec{p}\;^{\dagger};
    \rho)$, thereby the Gauss-Newton method is well defined. Also
    \begin{align}
        \|\vec{p}\;^{k+1} - \vec{p}\;^{\dagger}\|
        &\leq \left( \frac{C_lC_L\rho}{2}  \right)^{k+1}
        \|\vec{p}\;^{0} - \vec{p}\;^{\dagger}\| \\
        &\le \left( \frac{C_lC_L\rho}{2}  \right)^{k+1} \rho,
    \end{align}
    where $\frac{C_lC_L\rho}{2} < 1$, meaning that the Gauss-Newton method
    converges to $0$ as $k\to \infty$, quadratically. \qed
\end{myproof}

\begin{myproof}{Lemma \ref{lemma: immersion}}
    \label{proof: lem-immersion}
    \begin{itemize}
        \item Because of the differentiability assumptions of $\sigma$, for
    all fixed $\vec{p}$, we have that $D\Psi[\vec{p}\;] \in
    L^{2}\left([0,1]^{n}\right)$. So the Conjecture \ref{conj: main} directly
    implies that $\mathcal{R}(D\Psi[\vec{p}\;])$ is a linear subspace of
    $\mathbf{X}$ of dimension $n_*$.
\item Note that the operator $D^{2}\Psi[\vec{p}\;]: \mathbb{R}^{n_*} \to
    L^{2}([0, 1]^{n})$ is continuous (assuming that the activation function
    $\sigma$ is twice differentiable. Considering a nonempty neighborhood
    $\mathcal{U}$ of $\vec{p}$ with a compact closure, the continuity of
    $D^{2}\Psi$ w.r.t $\vec{p}$ gives that $D\Psi$ is
    Fr\'echet-differentiable with Lipschitz-continuous derivative on
    $\mathcal{U}$. Meaning there exist constants $C_l, C_L > 0$ such that
    for all $\vec{q}, \vec{p} \in \mathcal{D}(\Psi)$ it holds that
        \begin{align}
            &\big\|\Psi'(\vec{p}\;) - \Psi'(\vec{q}\;)
            \big\|_{\vec{P}\to\mathbf{X}} \leq C_L \|\vec{p} - \vec{q} \|
            \\
            &\big\|\Psi'(\vec{p}) \big\|_{\vec{P} \to \mathbf{X}} \le
            C_l.
        \end{align}
        \qed
    \end{itemize}

\end{myproof}





\nocite{kaltenbacher2008}
\nocite{frischauf2022universal}
\printbibliography

\end{document}
