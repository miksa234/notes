\documentclass[a4paper]{article}

\usepackage[T1]{fontenc}
\usepackage[utf8]{inputenc}
\usepackage{mlmodern}

%\usepackage{ngerman}	% Sprachanpassung Deutsch

\usepackage{graphicx}
\usepackage{geometry}
\geometry{a4paper, top=15mm}

\usepackage{subcaption}
\usepackage[shortlabels]{enumitem}
\usepackage{amssymb}
\usepackage{amsthm}
\usepackage{amsmath}
\usepackage{mathtools}
\usepackage{braket}
\usepackage{bbm}
\usepackage{graphicx}
\usepackage{float}
\usepackage{yhmath}
\usepackage{tikz}
\usepackage{scratch}
\usetikzlibrary{patterns,decorations.pathmorphing,positioning}
\usetikzlibrary{calc,decorations.markings}

\usepackage[backend=biber, sorting=none]{biblatex}
\addbibresource{cite.bib}

\usepackage[framemethod=TikZ]{mdframed}

\tikzstyle{titlered} =
    [draw=black, thick, fill=white,%
        text=black, rectangle,
        right, minimum height=.7cm]


\usepackage[colorlinks=true,naturalnames=true,plainpages=false,pdfpagelabels=true]{hyperref}
\usepackage[parfill]{parskip}
\usepackage{lipsum}

\usepackage{tcolorbox}
\tcbuselibrary{skins,breakable}

\pagestyle{myheadings}

\colorlet{colexam}{black}
\newcounter{definition}
\newtcolorbox[use counter=definition]{mydef}[1]{
    empty,
    title={\textbf{Definition~\thetcbcounter}~~(\textit{#1})},
    attach boxed title to top left,
    fontupper=\sl,
    boxed title style={
        empty,
        size=minimal,
        bottomrule=1pt,
        top=1pt,
        left skip=0cm,
        overlay=
            {\draw[colexam,line width=1pt]([yshift=-0.4cm]frame.north
        west)--([yshift=-0.4cm]frame.north east);}},
            coltitle=colexam,
            fonttitle=\normalfont,
            before=\par\medskip\noindent,
            parbox=false,
            boxsep=-1pt,
            left=0.75cm,
            right=3mm,
            top=4pt,
            breakable,
            pad at break*=0mm,
            vfill before first,
            overlay unbroken={
                \draw[colexam,line width=1pt]
                ([xshift=0.6cm, yshift=-0.5pt]frame.south
                west)--([xshift=0.6cm,yshift=-1pt]frame.north west)
                --([xshift=0.6cm]frame.south west)--([xshift=-13cm]frame.south east); },
            overlay first={
                \draw[colexam,line width=1pt]
                ([xshift=0.6cm, yshift=-0.5pt]frame.south
                west)--([xshift=0.6cm,yshift=-1pt]frame.north west)
                --([xshift=0.6cm]frame.south west); },
            overlay last={
                \draw[colexam,line width=1pt]
                ([xshift=0.6cm, yshift=-0.5pt]frame.south
                west)--([xshift=0.6cm,yshift=-1pt]frame.north west)
                --([xshift=0.6cm]frame.south west)--([xshift=-13cm]frame.south east); }
}
\newcounter{theorem}
\newtcolorbox[use counter=theorem]{theorem}{
    empty,
    title={Theorem ~\thetcbcounter},
    attach boxed title to top left,
    fontupper=\sl,
    boxed title style={
        empty,
        size=minimal,
        bottomrule=1pt,
        top=1pt,
        left skip=0cm,
        overlay=
            {\draw[colexam,line width=1pt]([yshift=-0.4cm]frame.north
        west)--([yshift=-0.4cm]frame.north east);}},
            coltitle=colexam,
            fonttitle=\bfseries,
            before=\par\medskip\noindent,
            parbox=false,
            boxsep=-1pt,
            left=0.75cm,
            right=3mm,
            top=4pt,
            breakable,
            pad at break*=0mm,
            vfill before first,
            overlay unbroken={
                \draw[colexam,line width=1pt]
                ([xshift=0.6cm, yshift=-0.5pt]frame.south
                west)--([xshift=0.6cm,yshift=-1pt]frame.north west)
                --([xshift=0.6cm]frame.south west)--([xshift=-13cm]frame.south east); },
            overlay first={
                \draw[colexam,line width=1pt]
                ([xshift=0.6cm, yshift=-0.5pt]frame.south
                west)--([xshift=0.6cm,yshift=-1pt]frame.north west)
                --([xshift=0.6cm]frame.south west); },
            overlay last={
                \draw[colexam,line width=1pt]
                ([xshift=0.6cm, yshift=-0.5pt]frame.south
                west)--([xshift=0.6cm,yshift=-1pt]frame.north west)
                --([xshift=0.6cm]frame.south west)--([xshift=-13cm]frame.south east); }
}
\newcounter{lemma}
\newtcolorbox[use counter=lemma]{lemma}{
    empty,
    title={Lemma~\thetcbcounter},
    attach boxed title to top left,
    fontupper=\sl,
    boxed title style={
        empty,
        size=minimal,
        bottomrule=1pt,
        top=1pt,
        left skip=0cm,
        overlay=
            {\draw[colexam,line width=1pt]([yshift=-0.4cm]frame.north
        west)--([yshift=-0.4cm]frame.north east);}},
            coltitle=colexam,
            fonttitle=\bfseries,
            before=\par\medskip\noindent,
            parbox=false,
            boxsep=-1pt,
            left=0.75cm,
            right=3mm,
            top=4pt,
            breakable,
            pad at break*=0mm,
            vfill before first,
            overlay unbroken={
                \draw[colexam,line width=1pt]
                ([xshift=0.6cm, yshift=-0.5pt]frame.south
                west)--([xshift=0.6cm,yshift=-1pt]frame.north west)
                --([xshift=0.6cm]frame.south west)--([xshift=-13cm]frame.south east); },
            overlay first={
                \draw[colexam,line width=1pt]
                ([xshift=0.6cm, yshift=-0.5pt]frame.south
                west)--([xshift=0.6cm,yshift=-1pt]frame.north west)
                --([xshift=0.6cm]frame.south west); },
            overlay last={
                \draw[colexam,line width=1pt]
                ([xshift=0.6cm, yshift=-0.5pt]frame.south
                west)--([xshift=0.6cm,yshift=-1pt]frame.north west)
                --([xshift=0.6cm]frame.south west)--([xshift=-13cm]frame.south east); }
}

\newcommand{\eps}{\varepsilon}
\usepackage[OT2,T1]{fontenc}
\DeclareSymbolFont{cyrletters}{OT2}{wncyr}{m}{n}
\DeclareMathSymbol{\Sha}{\mathalpha}{cyrletters}{"58}

\markright{Popović\hfill Seminar\hfill}


\title{University of Vienna\\
\vspace{1cm}Seminar:\\Joint RICAM Seminar\\
\vspace{0.5cm}
Summary of talk by Otmar Scherzer
}
\author{Milutin Popovic}


\begin{document}
\maketitle
\tableofcontents

\section{Sheet 3}
\subsection{Exercise 13}
\subsubsection{Part a}
Solve
\begin{align}
    \text{min}\quad & -x_1 - 2x_2,\\
    \text{s.t.}\quad & x_1^{2} + x_2^{2} \le 4 \nonumber\\
    &x_1\ge 0, x_2 \ge 0 \nonumber
\end{align}
rewriting it in to reduced notation
\begin{align}
    \text{min}\quad & -x_1 - 2x_2,\\
    \text{s.t.}\quad & g_1(x_1, x_2) = x_1^{2} + x_2^{2} -4 \le 0 \nonumber\\
    &g_2(x_1,x_2) = - x_1 \le 0\nonumber\\
    &g_3(x_1,x_2) = - x_2 \le 0\nonumber
\end{align}
We know that for a KKT point $(x, \lambda)$ we have that the
Lagrangian of the problem satisfies
\begin{align}
    \nabla_x L(x, \lambda) = 0\\
    \lambda\geqq 0,\quad \lambda^Tg(x) = 0.
\end{align}
Then for $-\lambda_2 x_1 =0$ and $-\lambda_3 x_2 =0$ the only solution is for
$\lambda_2, \lambda_2 = 0$. Now we have a system of three equations with
three unknowns $x_1, x_2, \lambda_1$ that we can solve
\begin{align}
    &\nabla f(x) + \nabla(\lambda^{T}g(x)) =0\\
    &\begin{pmatrix}
        -1 + 2x_1\lambda_1\\
        -2 + 2x_2\lambda_1\\
        -\lambda_1(x_1^{2}+x_2^{2}-4)
    \end{pmatrix}
    =\begin{pmatrix} 0\\0\\0 \end{pmatrix}.
\end{align}
Solving the first and second equation we get
\begin{align}
    x_1 = \frac{1}{2\lambda_1},\quad x_2 = \frac{1}{\lambda_1}\\
    x_2 = \frac{1}{2}x_1.
\end{align}
Plugging this into equation 3 and considering $x_1 \ge 0$, $x_2 \ge 0$
which tells us what root to take we get
\begin{align}
    x_1^{2}+\frac{1}{4}x_1^{2} -4 = 0\\
    \Rightarrow x_1 = \frac{4}{\sqrt{5}}, \quad x_2 = \frac{2}{\sqrt{5} }
\end{align}
The solution the optimization problem is $x^{*}=(\frac{4}{\sqrt{5}},
\frac{2}{\sqrt{5}})^{T}$.
\subsubsection{Part b}
Verify if $x=(2,4)^{T}$ is an optimal solution of the optimization problem
and determine a KKT point.
\begin{align}
    \text{min}\quad & (x_1-4)^{2} + (x_2-3)^{2},\\
    \text{s.t.}\quad & x_1^{2}\le x_2 \nonumber\\
    &x_2 \le 4 \nonumber
\end{align}
i.e.
\begin{align}
    \text{min}\quad & (x_1-4)^{2} + (x_2-3)^{2},\\
    \text{s.t.}\quad &g_1(x) = x_1^{2} - x_2\le 0  \nonumber\\
    &g_2(x) = x_2 -4 \le 0 \nonumber
\end{align}
We again use the KKT optimality conditions
\begin{align}
    \nabla f(x^{*})  + \nabla (\lambda^{T}g(x)) = 0\\
    \begin{pmatrix}
        2(x_1-4) + 2\lambda_1x_1\\
        2(x_2-3) - \lambda_1 + \lambda_2
    \end{pmatrix} =
    \begin{pmatrix} 0\\0 \end{pmatrix}
\end{align}
substituting for $x=(2,4)^{T}$ gives \begin{align} \begin{pmatrix}
        -4 + 4\lambda_1\\
        2 -\lambda_1 + \lambda_{2}
    \end{pmatrix}
    =
    \begin{pmatrix}
        0\\0
    \end{pmatrix}.
\end{align}
Which gives $\lambda_1 = 1, \lambda_2 = -1$ and tells us that $x=(2,4)^{T}$
is an optimal solution, and $\left( x^{*}=(2,4)^{T}, \lambda^{*}= (1,
-1)^{T}\right)$ is a KKT point.
\subsection{Exercise  14}
Solve the following optimization problem
\begin{align}
    \text{min}\quad & \Sigma_{i=1}^{n} \left( x_i - a_i \right)^{2},\\
    \text{s.t.}\quad & \Sigma_{i=1}^{n} x_i^{2} \le 1 \nonumber\\
    & \Sigma_{i=1}^{n} x_i = 0 \nonumber
\end{align}
To solve this we use the KKT optimality conditions for $g(x) =
\Sigma_{i=1}^{n}x_i^{2}$ and $h(x) = \Sigma_{i=1}^{n}x_i=0$.
\begin{align}
    &\nabla f(x) + \lambda \nabla g(x)
    +\mu \nabla h(x)= 0\\
    &\lambda \ge 0,\; g(x) \le 0,\; \lambda g(x) = 0, \;h(x) = 0.
\end{align}
From the first equation we get
\begin{align}
    &2x_i - 2a_i + 2  \lambda x_i + \mu = 0 \\
    &2\left( 1+\lambda \right) x_i +\mu -2a_i = 0\\
    &x_i=  \frac{2a_i - \mu}{2(1+\lambda)}
    \qquad \forall i \in
    \{1,\ldots,n\}.
\end{align}
By substituting the derived expression for $x_i$ into $h(x) =0$ we get
\begin{align}
    &\sum\frac{2a_i - \mu}{2(1+\lambda)} = 0\\
    \Rightarrow & \mu = \frac{2}{n}\sum a_i
\end{align}
plugging this into $\lambda g(x) = 0$ we get
\begin{align}
    &\sum \left( \frac{2a_i-\mu}{2\left( 1+\lambda \right)} \right)
    ^{2}-1=0\\
    &\sum \left( 2a_i -\mu \right)^{2} = 4(1+\lambda)^{2}\\
    &=4 \sum a_i^{2}-2\mu \sum a_i - n \mu^{2}= 4\left( 1+\lambda
    \right)^{2}\\
    &\sum a_i^{2} = (1+\lambda)^{2}.
\end{align}
Since $\lambda \ge 0 $ then $(1+\lambda) \ge 1$ and the root is positive
\begin{align}
    \lambda = 1 - \sqrt{\sum a_i}
\end{align}
Then $x_i$ becomes
\begin{align}
    x_i &= \frac{2a_i - \mu}{2(1-\lambda)} \\
        &= \frac{2a_i - \frac{2}{n}\sum_j a_j}{2\sum_j a_j}\\
        &= \frac{a_i}{\sum_j a_j} - \frac{1}{n}\\
        &= \frac{1}{n}\left( \frac{a_i}{\left\langle a \right\rangle } -1 \right)
\end{align}
where $\left\langle a \right\rangle$ denotes the standard mean of
$a=(a_1,\ldots,a_n)^{T}$.
\subsection{Exercise 15}
Consider the function
\begin{align}
    f : \;\;\mathbb{R}^{2}&\to \mathbb{R}\\
    (x_1, x_2) &\mapsto 3x_1^{4}-4x_1^{2}x_2 + x_2^{2}
\end{align}
Prove that the following statements for $x^{*}=(0,0)^{T}$ are true
\begin{enumerate}
    \item $x^{*}$ is a critical point of f
    \item $x^{*}$ is a strict local minimum of f along any line going through
        the origin
    \item $x^{*}$ is not a local minimum of $f$
\end{enumerate}
For 1. we check if $\nabla f(x^{*}) = 0$ then $x^{*}$ is a critical point
\begin{align}
&    \nabla f(x) =
    \begin{pmatrix}
        12x_1 + 8 x_1 x_2\\
        4x_1^{2} + 2x_2
    \end{pmatrix} \\
&    \nabla f(x^{*}) =
    \begin{pmatrix}
        0\\
        0
    \end{pmatrix}.
\end{align}
For 2. we need minimize $f(x)$ subjected to all lines through the origin. We
start with lines $x_2 = m x_1$ for $m \neq 0$.
\begin{align}
    f(x_1, x_2=mx_1) = 3x_1^{4} - 4mx_1^{3} + m^{2}x_1^{2},
\end{align}
set $g(x) = f(x, mx)$. We need to check that $x=0$ is local minimum of $g(x)$
\begin{align}
    &g'(x) = 12x^{3}-12mx^{2} + 2m^{2}x\\
    &g''(x) = 36x^{2} - 24m x + 2m^{2} .\\
    &g'(0) = 0 \qquad  g''(0) = 2m^{2}>0.
\end{align}
So $f$ is a strict local min along lines $x_2 = mx_1$. Next we check along
$x_1 = mx_2$
\begin{align}
   f(x_1=mx_2, x_{2}) = 3m^{4}x_2^{4}-tm^{2}x_2^{3} + x_2^{2}
\end{align}
set $g(x) = f(x_1=mx_2, x_2)$
\begin{align}
 &   g'(x) = 12m^{4}x^{3}-12m^{2}x^{2}+2x\\
 &   g''(x) = 36m^{4}x^{2} - 24m^{2}x + 2\\
 & g'(0) =0 \qquad g''(0) =  2 > 0.
\end{align}
For 3. we need to show that $x^{*}=(0,0)^{T}$ is not a local minimum of $f$.
Consider $x_2 = 2x_1^{2}$
\begin{align}
    f(x_1, 2x_1^{2}) &= 3x_1^{4} - 8x_1^{4} + 4x_1^{4}\\
                     &= -x_1^{4} < f(x^{*}) = 0\qquad \forall x_1 \in \mathbb{R}\setminus \{0\}
\end{align}
We have found function values smaller than that of $f(x^{*}) = 0$.
\subsection{Exercise 16}
\subsubsection{Part a}
Formulate a statement concerning the solutions of the optimization problem
\begin{align}
    \text{max}\quad & x_1,\\
    \text{s.t.}\quad & x_1^{2} + x_2^{2} \le 1 \nonumber\\
    &(x_1 - 1)^{2} + x_2^{2} \ge 1 \nonumber\\
    &x_1 + x_2 \ge 1\nonumber.
\end{align}
using geometric arguments and verify this statement by means of arguments.
\newline
Let $X$ be the optimization domain
\begin{align}
    X = \left\{ \left( x_1, x_2 \right) \in \mathbb{R}^{2}: x_1^{2}+x_2^{2}
    \le 1, (x_1 -1)^{2} + x_2^{2} \ge 0 , x_1 +x_2 \ge 0  \right\}
\end{align}
Then $X$ has the following graphical representation in the red domain of the
plot below.
\begin{figure}[H]
    \centering
    \begin{tikzpicture}[yscale=1, xscale=1]
        \begin{axis}[
            xmin=-2.3, xmax=2.3,
            ymin=-2.3, ymax=2.3,
            xlabel=$x_1$, ylabel=$x_2$,
            axis lines = middle,
        ]
            \addplot[domain=-1:2, samples=100, color=black, name path=A]{-x+1};
            \addplot[domain=-1:1, samples=100, color=black, name
                path=B1]{sqrt(1-x^2)};
            \addplot[domain=-1:1, samples=100, color=black, name
                path=B2]{-sqrt(1-x^2)};
            \addplot[domain=0:2, samples=100, color=black, name
                path=C1]{sqrt(1-(x-1)^2)};
            \addplot[domain=0:2, samples=100, color=black, name
                path=C2]{-sqrt(1-(x-1)^2)};

            \addplot[domain=-2:2, samples=100, color=orange, name
                path=D, opacity=0]{2};

            \addplot[domain=-2:2, samples=100, color=orange, name
                path=E, opacity=0]{-2};

            \addplot[domain=-2:2, samples=100, color=orange, name
                path=F, opacity=0]{0};

            \addplot[domain=-2:2, samples=100, color=black, name
                path=G, opacity=1]{x};

            \addplot[domain=-2:2, samples=100, color=black, name
                path=G, opacity=1]{x};
            \draw[red] (axis cs:0.5, 2) -- (axis cs:0.5, -2);

           \addplot[fill=orange,fill opacity=0.3] fill between[of=A and D,soft
                clip={domain=-1:2},];
           \addplot[fill=blue,fill opacity=0.3] fill between[of=B1 and F,soft
                clip={domain=-1:1},];

           \addplot[fill=blue,fill opacity=0.3] fill between[of=B2 and F,soft
                clip={domain=-1:1},];

           \addplot[fill=green,fill opacity=0.1] fill between[of=D and E,soft
                clip={domain=-2:0},];

           \addplot[fill=green,fill opacity=0.1] fill between[of=C2 and E,soft
                clip={domain=0:2},];

           \addplot[fill=green,fill opacity=0.1] fill between[of=C1 and D,soft
                clip={domain=0:2},];

           \addplot[fill=red,fill opacity=1] fill between[of=A and B1,soft
                clip={domain=0:0.2928},];

           \addplot[fill=red,fill opacity=1] fill between[of=C1 and B1,soft
                clip={domain=0.2928:1/2},];

        \end{axis}
    \end{tikzpicture}
    \caption{Area: Red: $X$, Blue: $x_1^{2}+ x_2^{2} \le 1$, Green:
    $(x_1-1)^{2} +x_2^{2} \ge 1$ and Orange: $x_1 + x_2 \ge 1$}
    \label{fig: ex16a}
\end{figure}
Solutions are in the red area. But since $f(x_1, x_2) = x_1$ actually only depends
on $x_1$ we can choose any $x_2$ then the maximum in the area is at $x_1=
\frac{1}{2}$. The analytical argumentation on the other hand follows KKT
optimality condition, for this we transform the maximization problem into a
minimization problem by multiplying the objective function $f$ by $-1$.
\begin{align}
    \text{min}\quad & -x_1,\\
    \text{s.t.}\quad & g_1\left(x  \right) = x_1^{2} + x_2^{2} -1 \le 0 \nonumber\\
    &g_2(x) = 1- (x_1 - 1)^{2} - x_2^{2} \le 0 \nonumber\\
    &g_3(x) = 1- x_1 - x_2 \le 0\nonumber.
\end{align}
then $\nabla L(x, \lambda) =0$, $\lambda^{T}g(x) = 0$ and $\lambda^{T} \geqq
0 $ will give us the
optimal solution for the optimization problem
\begin{align}
    \begin{pmatrix}
        -1 + 2\lambda_1x_1 - 2\lambda_2(x_1 -1) - \lambda_3\\
        \lambda_1 x_2 - 2\lambda_2 x_2 - \lambda_3
    \end{pmatrix}
    =
    \begin{pmatrix} 0\\0 \end{pmatrix}
\end{align}
we set $x_2 = 0$ since $x_2$ is not dependent on objective then we get that
$\lambda_3 = 0$ and
\begin{align}
    \lambda_1 = -\lambda_2 \frac{1-(x_1 - 1)^{2}}{x_2^{2}-1}.
\end{align}
we are left with
\begin{align}
    -1 + 2\lambda_1x_1 - 2\lambda_2(x_1 -1) - \lambda_3.\label{eq: 16a lag}
\end{align}
Then $\lambda^{T}g(x)$ gives us
\begin{align}
    \lambda_1 = -\lambda_2 \frac{1-(x_1-1)^2}{x_1^{2} - 1}
\end{align}
substituting into \ref{eq: 16a lag} back and calculating we arrive at the equation
\begin{align}
    x_1^{2} - x_1 +1 = 0
\end{align}
which gives $x_1=\frac{1}{2}$.
\subsubsection{Part b}
Verify if $x^{*} = (1, 1)^{T}$ fulfills the constraint qualifications of
LICQ, MFCQ, ABADIE-CQ.
\begin{align}
    \text{min}\quad & x_1,\\
    \text{s.t.}\quad & g_1\left(x  \right) = x_1 + x_2 -2 \le 0 \nonumber\\
    &g_2(x) = 1-x_1x_2 \le 0 \nonumber\\
    &g_3(x) = -x_1 \le 0\nonumber.
    &g_4(x) = -x_2 \le 0\nonumber.
\end{align}
Tangent cone are all tangent vectors $x_2 = \frac{1}{x_1} = 1$
\begin{align}
    T_X(x^{*}) = \left\{
        \begin{pmatrix} \lambda\\-\lambda \end{pmatrix} ,
        \begin{pmatrix} -\lambda\\\lambda \end{pmatrix}
        : \lambda \ge 0
    \right\}
\end{align}
The linearized tangent cone is at
\begin{align}
    T_\text{lin} (x^{*}) &=
    \left\{ d \in \mathbb{R}^{2}:
        \begin{pmatrix}1\\1  \end{pmatrix}^{T}d \le 0,
        \begin{pmatrix}-1\\-1  \end{pmatrix}^{T}d \le 0,
    \right\} \\
    &=
    \left\{
        \begin{pmatrix} \lambda\\-\lambda \end{pmatrix} ,
        \begin{pmatrix} -\lambda\\\lambda \end{pmatrix}
        : \lambda \ge 0
    \right\}.
\end{align}
Which means $x^{*}$ fulfills ABADIE-CQ. For MFCQ we need strict inequality
$\nabla g_i (x^{*})^{T}d < 0$ for $i \in \{1, 2\}$, which is not fulfilled for any $d \in
T_\text{lin}(x)$. For LICQ we need that $\{\nabla g_i (x^{*})\}_{i\in \{1,
2\} }$ are linearly independent. But
\begin{align}
    \begin{pmatrix} 1\\1  \end{pmatrix} ,
    \begin{pmatrix} -1, -1 \end{pmatrix}
\end{align}
are not linearly independent.
\subsection{Exercise 17}
Find out by using second order optimality conditions if $x^{*} = (0, 0)^{T}$
is a local minimum of
\begin{align}
    \text{min}\quad & -x_1^{2} + x_2,\\
    \text{s.t.}\quad & g_1\left(x  \right) = x_1^{3} - x_2 \le 0 \nonumber\\
    &g_2(x) = -mx_1+x_2 \le 0 \nonumber\\
\end{align}
where $m\ge0$.
We need to check that
\begin{align}
    d^{T}\nabla^{2}_x L(x^{*}, \lambda^{*})d >0 \qquad \forall d \in
    T_2(x^{*}),
\end{align}
    where
\begin{align}
    T_2(x^{*}) = \{d\in \mathbb{R}^{2}: &\nabla g_i(x^{*})d =0\;\; i \in
    \mathcal{A}_\ge (x^{*}), \\
    &\nabla g_i(x^{*})d \le 0\;\; i \in
    \mathcal{A}_0 (x^{*}) \}\\
\end{align}
and
\begin{align}
    \mathcal{A}_0(x^{*}) = \{i \in \mathcal{A}(x^{*}): \lambda_i^{*} = 0\} \\
    \mathcal{A}_>(x^{*}) = \{i \in \mathcal{A}(x^{*}): \lambda_i^{*} > 0\} \\
\end{align}
The gradients are
\begin{align}
    \nabla g_1 (x)|_{x^{*}} = (0, -1)^{T}\\
    \nabla g_2 (x)|_{x^{*}} = (-m, 1)^{T}.\\
\end{align}
Note that the KKT conditions $\lambda^{T}g(x)=0$ and $\lambda \geqq 0$ are
satisfied only if
\begin{align}
    \lambda^{T}g(x) = \lambda_1(-x_2 + x_1^{3}) + \lambda_2(-mx_1 + x_2) =
    0\\
    \lambda_1 = \lambda_2 \frac{mx_1 - x_2}{x_1^{3} -x_2},
\end{align}
since $x_1^{3} - x_2 \le 0$ then the $\lambda^{T} \geqq 0$ is satisfied only
if $\lambda 2 = 0$ which means $\lambda_1 =0$. Which gives us
$\mathcal{A}_> = \emptyset$ and $\mathcal{A}_0 = \{1, 2\}$ which means
$T_2(x^{*}) = T_\text{lin}(x^{*})$ and
\begin{align}
    T_\text{lin}(x^{*}) &=
    \left\{
        d \in \mathbb{R}^{2}:
        \begin{pmatrix}
            0\\
            -1
        \end{pmatrix}^{T}
        d \le 0,
        \begin{pmatrix}
            -m\\
            -1
        \end{pmatrix}^{T}
        d \le 0,
    \right\} \\
    &=
    \left\{
        \begin{pmatrix}
            \lambda\\
            \lambda m
        \end{pmatrix}
        :\lambda \ge 0
    \right\}
\end{align}
Now we calculate the hessian of the Lagrangian
\begin{align}
    L(x, \lambda) &= f(x) + \lambda_1 g_1(x) + \lambda_2 g_2(x)\\
            &= f(x)\\
    \nabla^2 L(x, \lambda) &= \nabla^2 f(x)\\
                           &=
    \begin{pmatrix}
        -2 & 0\\
        0 & 0
    \end{pmatrix}
\end{align}
Then
\begin{align}
    \begin{pmatrix}
        \lambda\\
        \lambda m
    \end{pmatrix}^{T}
    \begin{pmatrix}
        -2 & 0\\
        0 & 0
    \end{pmatrix}
    \begin{pmatrix}
        \lambda\\
        \lambda m
    \end{pmatrix} = -2\lambda^{2} \not> 0.
\end{align}
We conclude that $x^{*}=(0,0)^{T}$ is not a local minimum of the optimization
problem.
\subsection{Exercise 18}
Let $(t_k)_{k\ge 0} \subseteq \mathbb{R}$ be a monotonically decreasing
sequence and $t^{*}$ an accumulation point of it. Show that the sequence
$(t_k)_{k\ge 0}$ converges to $t^{*}$.
\newline
We know that $t ^{*}$ is an accumulation point of $(t_k)_{k\ge 0}$ so
\begin{align}
    &\forall U_\varepsilon(t^{*}) = [t^{*}-\varepsilon, t^{*}+\varepsilon],
    \varepsilon>0 \;\; \forall N\in \mathbb{N} \;\; \exists n \ge N: t_n \in
    U_\varepsilon(t^{*})\\
    &\text{i.e.}\quad |t_n - t^{*}| < \varepsilon \qquad \forall n\ge N \in \mathbb{N}
\end{align}
since $(t_k)_{k\ge 0}$ monotonically decreasing, $t_0 > t_1 > \ldots>
t_k > \ldots $ we have that $\forall n \in N$
\begin{align}
    \varepsilon_n  > |t_n - t^{*} | > |t_{n+1} - t^{*}|
\end{align}
so there exists a positive, strictly monotonically decreasing subsequence $\left(
\varepsilon_k \right)_{k\ge 0}$  of $(t_k)_{k\ge 0}$ defined by $\varepsilon_n >
|t_n - t^{*}|$ and $\varepsilon_n > \varepsilon_{n+1}$ that converges to $0$
so $(t_k)_{k\ge 0}$ converges to $t^{*}$.
\end{document}
