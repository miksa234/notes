\documentclass[a4paper]{article}

\usepackage[T1]{fontenc}
\usepackage[utf8]{inputenc}
\usepackage{mlmodern}

%\usepackage{ngerman}	% Sprachanpassung Deutsch

\usepackage{graphicx}
\usepackage{geometry}
\geometry{a4paper, top=15mm}

\usepackage{subcaption}
\usepackage[shortlabels]{enumitem}
\usepackage{amssymb}
\usepackage{amsthm}
\usepackage{mathtools}
\usepackage{braket}
\usepackage{bbm}
\usepackage{graphicx}
\usepackage{float}
\usepackage{yhmath}
\usepackage{tikz}
\usetikzlibrary{patterns,decorations.pathmorphing,positioning}
\usetikzlibrary{calc,decorations.markings}

%\usepackage[backend=biber, sorting=none]{biblatex}
%\addbibresource{uni.bib}

\usepackage[framemethod=TikZ]{mdframed}

\tikzstyle{titlered} =
    [draw=black, thick, fill=white,%
        text=black, rectangle,
        right, minimum height=.7cm]


\usepackage[colorlinks=true,naturalnames=true,plainpages=false,pdfpagelabels=true]{hyperref}
\usepackage[parfill]{parskip}
\usepackage{lipsum}

\usepackage{tcolorbox}
\tcbuselibrary{skins,breakable}

\pagestyle{myheadings}

\newcommand{\eps}{\varepsilon}
\usepackage[OT2,T1]{fontenc}
\DeclareSymbolFont{cyrletters}{OT2}{wncyr}{m}{n}
\DeclareMathSymbol{\Sha}{\mathalpha}{cyrletters}{"58}

\markright{Popović\hfill Numerical Analysis\hfill}


\title{University of Vienna\\ Faculty of Mathematics\\
\vspace{1cm}Numerical Analysis Problems
}
\author{Milutin Popovic}


\begin{document}
\maketitle
\tableofcontents

\section{Sheet 7}
\subsection{Exercise 43}
Consider the optimization problem
\begin{align}
    \text{min}\quad &f(x) := (x_1+1)^{2}+(x_2+2)^2,\\
    \text{s.t.}\quad & g_1(x) := -x_1 \le 0 \nonumber\\
    & g_2(x) := -x_2 \le 0 \nonumber
\end{align}
with $x = (x_1,x_2)^{T}$. For $\alpha >0$, find the minimum $x^{*}(\alpha)$
of the penalty function
\begin{align}
    P(x;\alpha) := f(x) + \frac{\alpha}{2} \|g_+(x)\|^2
\end{align}
and the limit points $x^{*}=\lim_{\alpha \to +\infty} x^{*}(\alpha)$ and
$\lambda^{*} = \lim_{\alpha \to +\infty}\alpha g_+ (x^{*}(\alpha))$. Find out
if $(x^{*}, \lambda^{*})$ is a KKT point of the constrained optimization
problem.
\newline
First we find the minimum of $P(x;\alpha)$.
\begin{align}
    \nabla P(x;\alpha)
    = \nabla f(x) + \frac{\alpha}{2}\left(\nabla\left(\max(0,-x_1)\right)^{2}
    +\nabla \left( \max\left(0,-x_2\right) \right)^{2} \right)
\end{align}
since $\frac{\partial }{\partial x_i} \max(0, -x_i)^2$ is $2x_i$ for $x_i < 0$
and $0$ otherwise for all $i=1,2$, so we have the equations
\begin{align}
    \nabla P(x;\alpha) =
    \begin{pmatrix}
        2(x_1+1)\\
        2(x_2+2)
    \end{pmatrix}
    + \alpha
    \begin{pmatrix}
        x_1\\
        x_2
    \end{pmatrix}= 0
\end{align}
which gives
\begin{align}
    x^{*}(\alpha) =
    \begin{pmatrix}
        -2(2+\alpha)^{-1}\\
        -4(2+\alpha)^{-1}
    \end{pmatrix}\\
    x^{*} = \lim_{\alpha \to +\infty}x^{*}(\alpha)=
    \begin{pmatrix} 0\\0 \end{pmatrix}
\end{align}
and
\begin{align}
    \lambda^{*}
    &= \lim_{\alpha \to +\infty}\alpha g_+(x^{*}(\alpha))\\
    &= \lim_{\alpha \to +\infty}
    \begin{pmatrix}
        \max(0, \frac{2\alpha}{\left(2+\alpha\right) }\\
        \max(0, \frac{4\alpha}{\left(2+\alpha\right) }
    \end{pmatrix}\\
    &=
    \begin{pmatrix}2 \\ 4 \end{pmatrix}.
\end{align}
All that is left is to show that $(x^{*}, \lambda^{*})$ is a KKT point by
$\nabla_x L(x^{*},\lambda^{*}) =0$
\begin{align}
    &\nabla f(x^{*}) + \lambda_1^{*} \nabla g_1(x^{*}) + \lambda_2^{*}\nabla
    g_2(x^{*}) = \\
    &=
    \begin{pmatrix} 2\\ 4 \end{pmatrix}
    + \begin{pmatrix} -2 \\ -4 \end{pmatrix}\\
    &=\begin{pmatrix} 0\\0 \end{pmatrix},
\end{align}
we conclude that $\left(x^{*}=(0,0)^{T}, \lambda^{*} = (2, 4)^{T}\right)$ is a
KKT point.
\subsection{Exercise 44}
Consider the optimization problem
\begin{align}
    \text{min}\quad &f(x) := x^{2},\\
    \text{s.t.}\quad & g(x) := 1-\ln(x)\le 0 \nonumber
\end{align}
and the penalized optimization problem
\begin{align}
    \min_{x\in\mathbb{R}} P(x;\alpha) = f(x) + \alpha \phi\left(
    \frac{g(x)}{\alpha} \right).
\end{align}
with $\phi(t) = e^{t} -1$ (\textit{exponential penalty function}). For
$\alpha >0$ find the optimal solution $x^{*}\left( \alpha \right)$ of the
penalized optimization problem and prove $ x^{*}$, the limit of
$x^{*}(\alpha)$ as $\alpha \downarrow 0$ is an optimal solution of the
constrained optimization problem.
\newline
To find the minimum we differentiate $P(x;\alpha)$ w.r.t $x$
\begin{align}
    \frac{d}{dx}P(x;\alpha) = \frac{d}{dx}\left( x^{2} + \alpha\left(
    \exp\left( \frac{1-\ln(x)}{\alpha} \right)  -1 \right)  \right) =\\
    &= 2x - e^{\frac{1}{\alpha}}x^{-\frac{\alpha+1}{\alpha}}
\end{align}
setting to $0$ give the equation
\begin{align}
    &x^{-\frac{\alpha+1}{\alpha}} = 2e^{-\frac{1}{\alpha}}x\\
    &x^{*}(\alpha) = \left( \frac{1}{2}e^{\frac{1}{\alpha}}
    \right)^{\frac{\alpha}{2\alpha+1} }.
\end{align}
Then
\begin{align}
    x^{*} = \lim_{\alpha \downarrow 0}x^{*}(\alpha) = e.
\end{align}
First of all $f(x) = x^{2}$ is strictly convex and the condition $1-\ln(x)\le
0$ is equivalent the condition $x \ge e$. So there is no $y \in \{x\in
\mathbb{R}: x > e\}$ such that $f(y) = y^2 < e^2$. We conclude that $x^{*}=e$
is the optimal solution for the constrained optimization problem.
\subsection{Exercise 45}
Consider the optimization problem
\begin{align}
    \text{min}\quad &f(x) := x^{2},\\
    \text{s.t.}\quad & h(x) := x-1= 0 \nonumber
\end{align}
and its optimal solution $x^{*}=1$. For $\overline{\alpha}>0$ such that
$x^{*}$ is a minimum of the $\ell_1$-penalty function $P_1(\cdot, \alpha)$
for all $\alpha \ge \overline{\alpha}$.
We have that for $\overline{\alpha}$ and $x^{*}$ and some $x \in \mathbb{R}$ that
\begin{align}
   &P_1(x;\overline{\alpha}) < P_1(x^{*},\overline{\alpha})\\
   &x^{2} + \overline{\alpha} |x-1| < 1  \qquad \big|\frac{d}{dx}\cdot\\
   &2x + \overline{\alpha} \frac{x-1}{|x-1|} < 0\\
   &\alpha < -\frac{2x|x-1|}{x-1} \longrightarrow 2 \quad \text{as
   $x\downarrow 1$ }
\end{align}
so $\alpha \ge 2$.
\subsection{Exercise 46}
Consider the optimization problem in Exercise 40
\begin{align}
    \min \quad &f(x):=\gamma + c^{T}x + \frac{1}{2}x^{T}Qx, \label{eq: opp}\\
    \text{s.t}\quad &h(x) := b^{T}x = 0, \nonumber
\end{align}
The penalized optimization problem
\begin{align}
    P(x;\alpha) := f(x) + \frac{\alpha}{2}\left( h(x) \right)^{2}
\end{align}
with solution
\begin{align}
    x^{*}(\alpha) = \left( \frac{\alpha}{1+\alpha b^{T}Q^{-1}b}Q^{-1} bb^{T}
    -I\right) Q^{-1} c
\end{align}
and solution to the constrained optimization problem
\begin{align}
    x^{*} &= \lim_{\alpha \to \infty}x^{*}(\alpha)\\
          &= \left(\frac{Q^{-1}bb^{T}}{b^{T}Q^{-1}b} -I \right) Q^{-1} c.
\end{align}
\subsubsection{Part a}
Prove that
\begin{align}
    \mu^{*} := \lim_{\alpha \to +\infty}\alpha h(x^{*}(\alpha))
\end{align}
is a Lagrange multiplier corresponding to the optimal solution $x^{*}$.
\begin{align}
    \alpha h(x^{*}(\alpha))
    &= \alpha b^{T}x^{*}(\alpha)\\
    &= \alpha \left( \frac{\alpha}{1+\alpha b^{T}Q^{-1}b}b^{T}Q^{-1}b\ b^{T}
    -b^{T}\right) Q^{-1}c\\
    &= \left( \frac{\alpha^{2}}{1+\alpha b^{T}Q^{-1}b}b^{T}Q^{-1}b
     - \alpha\right) b^{T}Q^{-1}c\\
    &= \left( \frac{\alpha^{2} b^{T}Q^{-1}b - \alpha -
    \alpha^{2}b^{T}Q^{-1}b}{1+\alpha b^{T}Q^{-1}b} \right) b^{T}Q^{-1}c\\
    &= \left(\frac{-\alpha}{1+\alpha b^{T}Q^{-1}b} \right) b^{T}Q^{-1}c\\
\end{align}
then we let the $\alpha \to +\infty$ and we get
\begin{align}
    \mu^{*} = -\frac{b^{T}Q^{-1}c}{b^TQ^{-1}b}.
\end{align}
Now we check if $\mu^{*}$ is the Lagrange multiplier w.r.t $x^{*}$.
\begin{align}
    L(x, \mu)
    &= f(x) + \mu h(x)\\
    &= \gamma + c^{T}x + \frac{1}{2}x^{T}Qx + \mu b^{T}x,
\end{align}
we need the condition $\nabla L(x^{*},\mu^{*}) = 0$, which is satisfied if
\begin{align}
    &\nabla L(x^{*}, \mu^{*}) = c + Qx^{*} + \mu^{*}b = 0 \qquad \Big|
    b^{T}Q^{-1}\\
    &-\mu^{*}b^{T}Q^{-1}b = b^{T}Q^{-1}c + b^{T}x^{*}.\\
\end{align}
we know that $b^{T}x^{*}=0$ is satisfied then
\begin{align}
    \mu^{*} = -\frac{b^{T}Q^{-1}c}{b^TQ^{-1}b}.
\end{align}
which is the same as taking the limit.
\subsubsection{Part b}
A bit confused here.
\subsection{Exercise 47}
Prove that the following functions are NCP-functions.
\begin{enumerate}
    \item minimum function
        \begin{align}
            \varphi(a,b) = \min \{a, b\}
        \end{align}
    \item Fischer-Burgmeister function
        \begin{align}
            \varphi(a,b) = \sqrt{a^{2}+b^{2}} - a - b
        \end{align}
    \item penalized minimum function
        \begin{align}
            \varphi(a,b) = 2\lambda \min \{a, b\}  + (1-\lambda) a_+ b_+
        \end{align}
        where $a_+ = \max \{0, a\}$, $b_+ = \max \{0, b\} $ and $\lambda \in
        (0, 1)$
\end{enumerate}
For 1. we have that $\min \{a, b\}  =0$ if
\begin{align}
    &\Leftrightarrow a=0 \quad \text{for}\quad b\ge 0\quad \text{then}\quad
    ab =0\\
    &\Leftrightarrow b=0 \quad \text{for}\quad a\ge 0\quad \text{then}\quad
    ab =0.
\end{align}
The minimum function is an NCP-function
\newline
For 2. we have
\begin{align}
    \varphi(a,b) = \sqrt{a^{2}+b^{2}} - a -b =0
\end{align}
then
\begin{align}
    a^{2} + b^{2} = (a+b)^{2},
\end{align}
here we need $a\ge 0$ and $b\ge 0$ to preserve the root. Solving the above we
get $2ab =0$ or simply $ab =0$, which means $\varphi$ is an NCP-function
\newline
For 3. we have that
\begin{align}
    &\varphi(a,b) = -2\lambda \min (a,b) + (1-\lambda) \max(0, a) \max(0,b) = 0\\
    &-2\lambda \min (a,b) = (1-\lambda) \max(0, a) \max(0, b) = 0.
\end{align}
The solution is either $a = 0$ with $b \ge 0$ or $b=0$ with $a\ge 0$ in the
first case we get that $a\cdot b=0$, which means this is an NCP function.
\subsection{Exercise 48}
Let $(x^{*}, \lambda^{*}, \mu^{*}) \in \mathbb{R}^{n + m + p}$ be a
KKT point of the optimization problem.
\begin{align}
    \text{min}\quad &f(x),\\
    \text{s.t.}\quad & g_i(x) \le 0, i=1,\ldots,m \nonumber
                     &h_j(x) =0, j=1,\ldots,p \nonumber
\end{align}
all functions are considered to be twice continuously differentiable.
Additionally we have that
\begin{itemize}
    \item $g_i(x^{*}) + \lambda^{*}_i \neq 0$ for all $i=1,\ldots,p$
    \item $\{\nabla g_i(x^{*})\}_{i\in \mathcal{A}(x^{*})}$ and $\{\nabla
        h_j(x^{*})\}_{j=1,\ldots,p}$ are linearly independent (LICQ)
    \item second order sufficient optimality condition is satisfied
\end{itemize}
Let $\Phi: \mathbb{R}^{n+m+p}\to \mathbb{R}^{n+m+p}$ be defined as
\begin{align}
   \Phi  :=
   \begin{pmatrix}
       \nabla_x L(x,\lambda ,\mu)\\
       h(x)\\
       \phi(-g(x), \lambda)
   \end{pmatrix}
\end{align}
where
\begin{align}
    \phi(-g(x), \lambda) :=
    \begin{pmatrix}
        \varphi(-g_1(x), \lambda_1)\\
        \vdots\\
        \varphi(-g_m(x), \lambda_1)\\
    \end{pmatrix} \in \mathbb{R}^{m}
\end{align}
and $\varphi:\mathbb{R}^{2} \to \mathbb{R}$ with $\varphi(a,b) = \min \{a,
b\}$. Show that the matrix $\nabla \Phi$ is well defined and regular.
\newline
The matrix is well defined because first of all, the functions $f, g_i, h_j$
are $C^{2}$ and $\min \{-g_i(x), \lambda_i\}$ is differentiable because of
the strict complementarity condition, meaning that
\begin{align}
    \nabla \varphi(-g_i(x^{*}, \lambda^{*}_i) =
    \begin{cases}
        -\nabla g_i(x^{*}) \quad & i \in \mathcal{A}(x^{*})\\
        0 \quad & i \not\in \mathcal{A}(x^{*})
    \end{cases}
\end{align}
Then we need to show that he matrix $\nabla \Phi(x^{*}, \lambda^{*},
\mu^{*})$ is regular, first of all the matrix has the following form
\begin{align}
    \nabla \Phi =
    \begin{pmatrix}
        \nabla_x^{2}L(x,\lambda, \mu) & \nabla h(x)^{T} & \nabla
        \phi(x)^{T}\\
        \nabla h(x) & 0 & 0\\
        \nabla \phi(x) & 0 & 0\\
    \end{pmatrix} \in \mathbb{R}^{(n+m+p) \times  (n+m+p)}.
\end{align}
To show that $\nabla \Phi(x^{*}, \lambda^{*},
\mu^{*})$ is regular we show that $\text{ker}\left(\nabla \Phi(x^{*}, \lambda^{*},
\mu^{*})  \right) = \emptyset$.
\newline
Let $ q = (q^{(1)}, q^{(2)}, q^{(3)})^{T} \in
\mathbb{R}^{n+m+p}$ then we need to find the solution of
\begin{align}
        \nabla \Phi(x^{*}, \lambda^{*}, \mu^{*}) \begin{pmatrix}
        q^{(1)}\\q^{(2)}\\q^{(3)} \end{pmatrix} =0.
\end{align}
These are three equations
\begin{align}
    &\nabla_x^{2}L(x^{*},\lambda^{*},\mu^{*})q^{(1)} + \nabla h(x^{*})^{T} q^{(2)}
    + \nabla \phi(x^{*})^{T}q^{(3)} = 0 \label{eq: ex49.1}\\
    &\nabla h(x^{*}) q^{(1)} = 0 \label{eq: ex49.2}\\
    &\nabla \phi(x^{*}) q^{(1)} = 0 \label{eq: ex49.3}.
\end{align}
By multiplying \ref{eq: ex49.1} with $(q^{(1)})^{T}$ we get that
\begin{align}
    &(q^{(1)})^{T}\nabla_x^{2}L(x^{*},\lambda^{*},\mu^{*})q^{(1)} +(q^{1})^{T} \nabla h(x^{*})^{T} q^{(2)}
    + (q^{1})^{T}\nabla \phi(x^{*})^{T}q^{(3)}=\\
    =&(q^{(1)})^{T}\nabla_x^{2}L(x^{*},\lambda^{*},\mu^{*})q^{(1)}
    + \sum_{j=1}^{p} q_j^{(2)}\underbrace{(q^{(1)})^{T}\nabla h_j(x^{*})}_{=0
    \;\; (\ref{eq: ex49.2})}
    + \sum_{i=1}^{m} q_i^{(3)}\underbrace{(q^{(1)})^{T}\nabla \phi(-g_i(x^{*}),
\lambda_i^{*})}_{=0 \;\; (\ref{eq: ex49.3})} \\
     &= 0,
\end{align}
in summary
\begin{align}
    (q^{(1)})^{T}\nabla_x^{2}L(x^{*},\lambda^{*},\mu^{*})q^{(1)} =0.
\end{align}
Since second order sufficient optimality condition is satisfied then $q^{(1)}
\in T_2(x^{*})$, and the only solution is $q^{(1)} = 0$. Equation \ref{eq:
ex49.1} is left with
\begin{align}
    &\nabla h(x^{*})^{T}q^{(2)}+\nabla \phi(x^{*})q^{(3)} =\\
    =& \sum_{j=1}^{p} q_j^{(2)}\nabla h_j(x^{*}) + \sum_{i \in
    \mathcal{A}(x^{*})} q_i^{(3)}(-\nabla g_i(x^{*})) = 0
\end{align}
since LICQ is fulfilled these vectors are linearly independent and by
definition of linear independence the only $q^{(2)}, q^{(3)}$ fulfilling the
above condition are $q^{(2)} = 0$ and $q^{(3)} = 0$. Thereby $q = 0$ and
$\text{ker}(\nabla\Phi(x^{*},\lambda^{*},\mu^{*})) = \emptyset$, so the matrix
is regular.



























\end{document}
