\appendix
\section{Appendix: Mathematical Preliminaries}
\subsection{Leibniz Rule of Integration}
\label{appendix:leibniz}
The Leibniz integral rule for differentiation under the integral sign
initiates with an integral
\begin{align}
    \mathcal{I}(t, x) = \int_{a(t)}^{b(t)} f(t, x) dx = \mathcal{I}(t, a(t,
    a(t), b(t))).
\end{align}
And upon differentiation w.r.t. $t$, utilizes the chain rule on $a(t)$ and
$b(t)$ respectively, by
\begin{align}
    \frac{d\mathcal{I}}{dt} =
    \frac{\partial \mathcal{I}}{\partial t}+
    \frac{\partial \mathcal{I}}{\partial a}\frac{\partial a}{\partial t}+
    \frac{\partial \mathcal{I}}{\partial b}\frac{\partial b}{\partial t}.
\end{align}
Which in integral representation reads
\begin{align}
    \frac{d\mathcal{I}}{dt} = \int_{a(t)}^{b(t)}\frac{\partial f(t,
    x)}{\partial t} dx + f(t, b(t)) \frac{\partial b(t)}{\partial t}
    - f(t, a(t)) \frac{\partial a(t)}{\partial t}
\end{align}

\subsection{Identity for Vorticity}
\label{appendix:diff identity}
We start off with the standard material derivative
\begin{align}
    \frac{D\mathbf{u}}{Dt} = \frac{\partial \mathbf{u}}{\partial t}
    +(\mathbf{u}\nabla)\mathbf{u}.
\end{align}
We will use Einstein's Summation Convention, where we sum over indices that
both appear at as the bottom as the top index, to rewrite the second part of
the material derivative $(\mathbf{u}\nabla)\mathbf{u}$ into
\begin{align}
    (\mathbf{u}\times (\nabla \times \mathbf{u}))_k
    &= \varepsilon^{ijk}u_j(\nabla \times  \mathbf{u})_k \\
    &= \varepsilon^{ijk}u_j\varepsilon_{klm}\partial^l u^m\\
    &=(\delta^i_l\delta^j_m-\delta^i_m\delta^j_l)u_j\partial^l u^m\\
    &=u_m\partial^i u^m - u_l \partial^l u^i.\label{eq:identity split}
\end{align}
Now the first part in equation \ref{eq:identity split} can be rewritten into
\begin{align}
    u_m\partial^i u^m =\partial^i (\frac{1}{2}u_mu^m) .
\end{align}
Thus we get
\begin{align}
    (\mathbf{u}\times (\nabla \times \mathbf{u}))_k
    = \frac{1}{2}\partial^i(u_m u^m) + u_l \partial^l u^i,
\end{align}
which is
\begin{align}
    (\mathbf{u}\nabla)\mathbf{u} = \nabla(\frac{1}{2}\mathbf{u}\mathbf{u}) -
    \left(\mathbf{u}\times (\nabla \times  \mathbf{u})\right)
\end{align}
\subsection{Middle Curvature of an Implicit Function}
\label{appendix:curvature}
In our case the implicit function for fixed time reads
\begin{align}
    z-h\left(x_1,x_2\right) = 0.
\end{align}
The parametric representation is
\begin{align}
    \mathbf{\sigma} = \begin{pmatrix} x_1 \\ x_2 \\ h \end{pmatrix} .
\end{align}
The middle curvature of the surface parametrized by $\mathbf{\sigma}$ is
\begin{align}
    \frac{1}{R} = \text{Tr}(G^{-1}B),
\end{align}
where $G$ and $B$ are given by
\begin{align}
    G_{ij} = \frac{\partial \mathbf{\sigma}}{\partial x_i} \frac{\partial
    \mathbf{\sigma}}{\partial x_j}, \\
    B_{ij} = -\mathbf{N} \frac{\partial^2 \mathbf{\sigma}}{\partial
    x_i\partial x_j},
\end{align}
where $i, j = 1, 2$ and $\mathbf{N}$ is the normal, normalized surface vector given by
\begin{align}
    \mathbf{N} &= \frac{\frac{\partial \mathbf{\sigma}}{\partial x_1}\times
    \frac{\partial \mathbf{\sigma}}{\partial x_2}}{\|\frac{\partial \mathbf{\sigma}}{\partial x_1}\times
    \frac{\partial \mathbf{\sigma}}{\partial x_2}\|} \\
               &= \frac{1}{\sqrt{h_x^2 + h_y^2 +1}} \begin{pmatrix}
               -h_x\\-h_y\\1 \end{pmatrix}.
\end{align}
Thereby the matrices $B$ and $G$ are calculated to be
\begin{align}
    G = \begin{pmatrix} 1+h_x^2 & h_xh_y\\h_xh_y & 1+h_y^2 \end{pmatrix}
    \qquad
    B =\frac{1}{\sqrt{h_x^2 +h_y^2 +1} } \begin{pmatrix}h_{x x} &
    h_{yx}\\h_{x y} & h_{yy}  \end{pmatrix}.
\end{align}
The inverse of $G$ is
\begin{align}
    G^{-1}
    &= \frac{1}{\det(G)} \text{adj}(G)\\
    &= \frac{1}{h_x^2+h_y^2 +1} \begin{pmatrix}1+h_y^2 & -h_xh_y \\-h_xh_y  &
    1+h_x^2\end{pmatrix} .
\end{align}
Hence the middle curvature is given by the follwing
\begin{align}
    \frac{1}{R} &
    = \text{Tr}(G^{-1}B)\\
                &= \frac{1}{(h_x^2 + h_y^2+1)^{\frac{3}{2}}}
    \text{Tr}\begin{pmatrix} (1+h_y)^2 h_{x x} - h_x h_y h_{xy} & *\\
    * & (1+h_x^2)h_{yy}-h_xh_yh_{xy}\end{pmatrix}\\
    &=\frac{(1+h_y^2)h_{x x}+(1+h_y^2)h_{yy} -
    2h_xh_yh_{xy}}{\left( h_x^2+h_y^2+1 \right)^{\frac{3}{2}} }.
\end{align}
