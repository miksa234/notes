\section{The Solitary Wave and The KdV Equation}
The solitary wave is a wave of translation, it is stable and can travel long
distances. Additionally the speed depends on the size of the wave. An
interesting feature is that two solitary waves do not merge together to form
one solitary wave, but the small wave is overtaken by a larger one. If a
solitary wave is too big for the depth it splits into two, a big and a small
one. Solitary waves arise in the region $\varepsilon=O(\delta^2)$.


\subsection{Solitary Wave}
To describe
a solitary wave we begin with Euler's Equation of Motion, where we assume
there is no surface tension we set $W_e = 0$ and additionally assume
irrotational flow $\mathbf{\omega}=\nabla \times  \mathbf{u} = 0$. This means
that there exists a velocity potential $\phi(\mathbf{x},t)$ given
by $\mathbf{u} = \nabla \phi$ satisfying the Laplace equation. In regard of a
solitary wave being a plane wave, we rotate our coordinate system such that
the propagation is in the $x$-direction and a stationary \& fixed bottom
$b=0$. Ultimately leaving us with the following model
\begin{align}\label{eq:soliton}
\begin{drcases}
   & \phi_{zz} + \delta \phi_{x x }  = 0,\\
   &\text{with the boundary conditions}\\
   &\begin{drcases}
    &\phi_z = \delta^2 (\eta_t + \varepsilon \phi_x \eta_x) \\
    &\phi_t + \eta +  \frac{1}{2}\varepsilon\left( \frac{1}{\delta^2}\phi^2_z
    + \phi_x^2\right)  =0
  \end{drcases}\quad \text{on}\;\; z = 1+\varepsilon\eta,\\
   &\text{and}\\
   & \phi_z =0 \quad \text{on}\;\; z = b = 0.
\end{drcases}
\end{align}
Since the model arises in $\varepsilon = O(\delta^2)$, for convince we set
$\varepsilon=1$. The fact of the matter is we are seeking a traveling wave
solution, thereby we can go into the coordinate system of the traveling wave,
one in the variable $\xi = x - ct$ for a, from left to right traveling wave,
where $c$ is the nondimensional speed of the wave. Our goal is to find the
solution for the velocity potential $\phi(\xi, z)$ and the wave profile
$\eta(\xi)$. The chain rule gives us
\begin{align}
    \frac{\partial }{\partial x} &= \frac{\partial \xi}{\partial x}
    \frac{\partial }{\partial \xi}  = \frac{\partial }{\partial \xi}, \\
    \frac{\partial }{\partial t} &= \frac{\partial \xi}{\partial t}
    \frac{\partial }{\partial \xi}  = -c\frac{\partial }{\partial \xi}.
\end{align}
Together with the equations in \ref{eq:soliton} we obtain
\begin{align}\label{eq:soliton-xi}
    \begin{drcases}
   & \phi_{zz} + \delta \phi_{\xi\xi}  = 0,\\
   &\text{with the boundary conditions}\\
   &\begin{drcases}
    &\phi_z = \delta^2 (\phi_\xi -c)\eta_\xi \\
    &-c\phi_\xi + \eta +  \frac{1}{2}\varepsilon\left( \frac{1}{\delta^2}\phi^2_z
    + \phi_\xi^2\right)  =0
  \end{drcases}\quad \text{on}\;\; z = 1+\eta,\\
   &\text{and}\\
   & \phi_z =0 \quad \text{on}\;\; z = b = 0.
    \end{drcases}
\end{align}
\subsubsection{Exponential Decay}
We would like to analyze if the equation in \ref{eq:soliton-xi} gives viable a
solution that decays exponentially, we make the ansatz
\begin{align}
    \eta \simeq a e^{-\alpha |\psi|},\quad \phi \simeq \psi(z)e^{-\alpha
    |\xi|}, \qquad  \mid \xi \mid \rightarrow \infty,
\end{align}
where $\alpha>0$ is the exponent. The equations in \ref{eq:soliton-xi}
transforms to
\begin{align}
    \psi'' + \alpha^2 \delta^2\psi = 0.
\end{align}
The above equation is a standard well known ordinary differential equation
reading
\begin{align}
    \psi = A \cos(\alpha\delta z),
\end{align}
where $A$ is the integration constant. On the free surface $z\simeq 1$ gives
\begin{align}
    &-cA\alpha\sin(\alpha\delta) = ca\alpha,\label{eq:sol1}\\
    &cA\alpha \cos(\alpha\delta) = -a \label{eq:sol2}.
\end{align}
Dividing equation \ref{eq:sol1} with equation \ref{eq:sol2} gives
\begin{align} \label{eq:soliton-dispersion}
    c^2 = \frac{\tan\left(\alpha\delta  \right) }{\alpha\delta}.
\end{align}
We conclude that the solution for such a wave exists provided that the
dispersion relation on the wave propagation speed holds, thereby solitary
waves exhibit exponential decay in their tail and satisfy the dispersion
relation in equation \ref{eq:soliton-dispersion}.
\subsubsection{Asymptotic Analysis}
The underlining equations in \ref{eq:soliton} extend from $-\infty$ to
$\infty$, so the length scale is much greater than any finite depth of
water. Therefore the classification $\delta \rightarrow 0$ is appropriate for
a solitary wave, this however goes with the assumption
$\varepsilon\rightarrow 0$ otherwise we cannot make an appropriate expansion.
Let us look at the main equation
\begin{align}\label{eq:sol-laplace}
    \phi_{zz} + \delta \phi_{x x} = 0.
\end{align}
For small $\delta$ we conduct the $\delta^2 = O(\varepsilon)$ standard ansatz
in asymptotic analysis
\begin{align}
    \phi_{\delta}(x, t, z) \simeq \sum_{n=0}^{\infty} \delta^{2n}\phi_n(x, t,
    z).
\end{align}
Substituting $\phi_\delta$ into equation \ref{eq:sol-laplace} we get
\begin{align}
    \delta^{2\cdot 0}\left( \phi_{0zz} \right)  + \delta^{2\cdot 1}\left(
    \phi_{1zz}+\phi_{0 x x} \right)  + \delta^{2\cdot 2}\left( \phi_{2zz}+
    \phi_{1 x x} \right)  + O(\delta^{2\cdot 3}) = 0.
\end{align}
We start off with $O(\delta^{2\cdot0}) $, which gives us an arbitrary function
$\phi_{0} = \theta(x, t)$. Next we may generalize the results for all
$O(\delta^{2\cdot n})$ in the means of
\begin{align}
    \phi_{n+1zz}  = -\phi_{nx x}\qquad \forall n\in \mathbb{N} .
\end{align}
Therefore leaving us for $\phi_1$ and $\phi_2$ with
\begin{align}
    &\phi_1 = -\frac{1}{2} z^2 \theta_0(x,t) + \theta(x, t),\\
    \Rightarrow& \phi_2 =
    \frac{1}{24}z^4\theta_0(x,t)-\frac{1}{2}z^2\theta_1(x,t) + \theta_2(x,t).
\end{align}
The boundary condition on the bottom comes around to be
\begin{align}
    \phi_{nz} =0 \quad \text{on}\;\; z=0.
\end{align}
The free surface boundary condition $z= 1+\varepsilon\eta$ involves more calculation, we consider
only terms up the order of $\delta^2$, initializing with
\begin{align}
    &\phi_z = \delta^2(\eta_t + \varepsilon\phi_x \eta_x)\\
    \Leftrightarrow &\frac{1}{\delta}\phi_z = \eta_t + \varepsilon\phi_x
    \eta_x,
\end{align}
substituting $\phi_\delta$ into the above proceeds to be
\begin{align}
    \frac{1}{\delta^2}\underbrace{\phi_{0z}}_{=0} + \phi_{1z}+ \delta^2\phi_{zz}
    O(\delta^{2\cdot 2})
    &= -z\theta_{x x} + \delta^2\left( \frac{1}{6}z^3\theta_{0 x x x x} - z
    \theta_{0x x} \right) + O(\delta^{2\cdot 2})\\
    &=-(1+\varepsilon\eta)\theta_{0 x x} + \delta^2\left(
    \frac{1}{6}(1+\varepsilon\eta)^3\theta_{0 x x} -
(1+\varepsilon\eta)\theta_{0 x x} \right) \label{eq:soliton-scale-boundary1}\\
    &= \eta_t + \varepsilon\eta_x \left(
        \theta_{0x}
    \delta^2(\theta_{1x}-\frac{1}{2}( 1+ \varepsilon\eta)^2 \theta_{0x x
    x}\label{eq:soliton-scale-boundary2}
\right).
\end{align}
The second condition is
\begin{align}
    \phi_t + \eta + \frac{1}{2}\varepsilon \left( \frac{1}{\delta}\phi^2_z
    +\phi_x^2\right)  = 0,
\end{align}
becomes after substitution
\begin{align}
    &\theta_{0t}+ \delta^2\left( -\frac{1}{2}(1+\varepsilon\eta)^2\theta_{0 x xt}
    + \theta_{1t}\right) + \eta + O(\delta^{2\cdot 2})
\label{eq:soliton-scale-boundary3}
    \\&=-\frac{1}{2}\delta^2\varepsilon(-(1+\varepsilon\eta)\theta_{0 x x
    })^2\label{eq:soliton-scale-boundary4}
    -\frac{1}{2}\left( \theta_{0 x} + \delta^2\left( \theta_{1x} -
    \frac{1}{2}(1+\varepsilon\eta)^2\theta_{0 x x x x}  \right)  \right) ^2
\end{align}
In the order of $O(\delta^{0})$ as $\varepsilon \rightarrow 0$ gives us the conditions
\begin{align}
    -\theta_{0 x x} &= \eta_t \simeq \text{and}\quad
    \theta_{0t}\simeq-\eta\label{eq:solitonO0}\\
    &\Rightarrow \theta_{0 x x} - \theta_{0 t t} \simeq 0.
\end{align}
This gives us the wave equation, a simple solution in the frame of the right
moving wave dependent on $\xi = x -t$ the chain rule gives us
\begin{align}
    \frac{\partial \theta_0(\xi(x, t))}{\partial t}
    &= \frac{\partial
    \theta_0}{\partial \xi} \underbrace{\frac{\partial \xi}{\partial t}}_{=-1}
    + \frac{\partial
    \theta_0}{\partial t} \underbrace{\frac{\partial t}{\partial t}}_{=1}
        + \frac{\partial \theta_0}{\partial x} \underbrace{\frac{\partial
        x}{\partial t}}_{=0}\\
    &=-\theta_{0\xi}+\theta_{0t}.
\end{align}
substituting into \label{eq:solitionO0} we get
\begin{align}
        &2\theta_{0t\xi}\simeq\theta_{0t t},\\
    \Rightarrow\;\;&\eta= \theta_{0\xi}+O(\varepsilon).
\end{align}
As for the surface boundary condition we see that the derivatives in $t$ are
''small``. So we can proceed by the scaling $\tau = \varepsilon t$ as
$\varepsilon\rightarrow 0$, we proceed with equation given in
\ref{eq:soliton-scale-boundary1} and \ref{eq:soliton-scale-boundary2} in the
$O(\varepsilon), O(\delta^2)$
\begin{align}\label{eq:soliton-scale-boundary5}
    -(1+\varepsilon\eta)\theta_{0\xi\xi}+
    \delta^2\left(\frac{1}{6}\theta_{0\xi\xi\xi\xi} - \theta_{1\xi\xi}\right)\simeq
    \varepsilon\eta_\tau -\eta_\xi +\varepsilon\eta\theta_{0\xi}
\end{align}
and boundary equations in \ref{eq:soliton-scale-boundary3},
\ref{eq:soliton-scale-boundary4} produce
\begin{align}\label{eq:soliton-scale-boundary6}
   \varepsilon\theta_{0\tau}-\theta_{0\xi}+\delta^2\left(
   \frac{1}{2}\theta_{0\xi\xi\xi} - \theta_{1\xi} \right) +\eta \simeq
   -\frac{1}{2}\varepsilon \theta^2_{0\xi}.
\end{align}
Doing the following operation to the above equations
\ref{eq:soliton-scale-boundary5} $-$ $\frac{\partial }{\partial
\xi}$\ref{eq:soliton-scale-boundary6} turns out to be
\begin{align}
    &-\theta_{0\xi\xi}-
    \varepsilon\eta\theta_{0\xi\xi}+
    \delta\left(\frac{1}{6}\theta_{0\xi\xi\xi\xi}-\theta_{1\xi\xi}\right)
    - \varepsilon\theta_{0\xi\tau}+\theta_{0\xi\xi}-\delta^2\left(
        \frac{1}{2}\theta_{0\xi\xi\xi\xi} -
    \theta_{1\xi\xi}\right)+\eta_{\xi}\\
    &\simeq \varepsilon\eta_t - \eta_\xi+
    \varepsilon\eta\theta_{0\xi}+\varepsilon\theta_{0\xi\xi}\theta_{0\xi}.
\end{align}
In the above equation we can simplify, i.e. short some terms out and
substitute $\eta = \theta_{0\xi} + O(\varepsilon)$ and because of $\delta^2 =
O(\varepsilon)$ we set $\delta^2 = K\varepsilon$ for constant $K$, leaving us
with the equation for the surface profile, called the \textbf{Korteweg-de
Vries}, KdV equation (1895)
\begin{align}
    2\eta_\tau + 3\eta\eta_\xi + \frac{K}{3}\eta_{\xi\xi\xi} = 0.
\end{align}
The KdV equation describes the balance between linearity and dispersion in
the change of time of the wave profile. By rewriting $\eta = f(\xi-ct)$ we
get
\begin{align}
    -2cf' + 3ff' + \frac{K}{3}f''' = 0\\
    \text{with} \quad f, f', f''' \rightarrow 0\quad \text{as}\;\;  |\xi-ct|
    \Rightarrow \infty.
\end{align}
The solution is a $\text{sech}$ function
\begin{align}
    f = 2c\ \text{sech}^2\left( \sqrt{\frac{3}{2K}}(\xi-ct)\right)
\end{align}
\subsection{KdV Equation\label{sec:kdv}}
In this section we will go over the more general prerequisites and therefore
a more convincing expedition for the Korteweg-de Vries equation. We still
want to derive the wave profile of a wave in shallow water, small amplitude
regime $\delta^2 = O(\varepsilon)$, where the bottom is horizontal \&
stationary. The propagating wave can be seen as a plane wave, therefore the
coordinate system is rotated in such a way that the propagating direction is
the $x$ direction. For irrotational, inviscid flow without surface tension
$W_e=0$ that is for gravity waves, nondimensional and rescaled Euler's
Equations of Motion for such a flow are
\begin{align}
    \begin{drcases}
        \frac{Du}{Dt}=-p_x,\quad \quad \delta^2
        \frac{Dw}{Dt} = -p_z,\\
    \text{where}\\
    \frac{D}{Dt} = \frac{\partial }{\partial t}  + \varepsilon
    \left(
    u\frac{\partial u}{\partial x}
    +w\frac{\partial w}{\partial z}\right)
\\
    \text{with}\\
    \frac{\partial u}{\partial x} +\frac{\partial w}{\partial z}  = 0
    \end{drcases}
\end{align}
with free surface boundary conditions
\begin{align}
    \begin{drcases}
        p=\eta\\
        w=\eta_t+\varepsilon u \eta_x
    \end{drcases}
    \text{on}\;\; z= 1+\varepsilon\eta,
\end{align}
and bottom boundary condition
\begin{align}
    w = 0 \quad \text{on}\;\; z=b =0.
\end{align}
We note here that the solution for such a wave is a solitary wave as in
described in the previous section. In principle we expect to find such waves
rather rarely in nature, since $\delta^2 = O(\varepsilon)$ is a very special
case. Never the less this is not the case. We demonstrate that $\forall\
\delta$ as $\varepsilon$ goes to $0$ there exists a region in the position
space $(x, t)$ where the KdV balance in terms of linearity and dispersion
is observed. Indeed we can ''generate`` KdV solitary waves, provided a small
enough amplitude in the sense of $\varepsilon$ goes to $0$. First of all we
introduce a rescaling of the variables adjusted to our problem definition
\begin{align}\label{eq:epsdelta}
    x \rightarrow \frac{\delta}{\sqrt{\varepsilon} }\tilde{x}, \quad t
    \rightarrow \frac{\delta}{\sqrt{\varepsilon} }\tilde{t}\quad
    w \rightarrow \frac{\sqrt{\varepsilon} }{\delta}\tilde{w}.
\end{align}
Then the material derivative is transformed to be
\begin{align}
    \frac{D}{Dt} = \frac{\sqrt{\varepsilon}}{\delta}(\frac{\partial
    }{\partial \tilde{t}} +\varepsilon \tilde{\mathbf{u}} \nabla).
\end{align}
The initial equations become
\begin{align}
    \frac{Du}{Dt} = \frac{\sqrt{\varepsilon}}{\delta} =-
    \frac{\sqrt{\varepsilon} }{\delta} p_{\tilde{x}}\;\; &\Rightarrow\;\;
    u_{\tilde{t}} + \varepsilon(u u_{\tilde{x}} + wu_z)= -p_{\tilde{x}}.\\
    \frac{Dw}{Dt} = \frac{\varepsilon}{\delta^2}
    \frac{D\tilde{w}}{D\tilde{t}}=-p_z \;\;&\Rightarrow\;\;
    \varepsilon\left(\tilde{w}_{\tilde{t}} + \varepsilon\left(
    u\tilde{w}_{\tilde{x}}+ \tilde{w}\tilde{w}_z \right)   \right)  = -p_z,
\end{align}
with
\begin{align}
    &w
    = \frac{\varepsilon}{\delta}\tilde{w}
    = \frac{\sqrt{\varepsilon} }{\delta}
    \eta_{\tilde{t}}+\varepsilon u\frac{\sqrt{\varepsilon}}{\delta}
    \eta_{\tilde{x}}\\
    &\Rightarrow\;\;
    \begin{drcases}
    \tilde{w} = \eta_{\tilde{t}}+ \varepsilon u
    \eta_{\tilde{x}}\\
    p=\eta
    \end{drcases}
    \text{on}\;\; z = 1+\varepsilon\eta
\end{align}
and
\begin{align}
    w = 0 \quad \text{on}\;\; z= b = 0.
\end{align}
Now we replace the region $\delta^2$ with $\varepsilon = \delta^2$, while we
let $\varepsilon$ go to $0$. We conclude to the following equations, after
replacing to $x, t$ and $w$
\begin{align}\label{eq:kdv3}
    \begin{drcases}
    u_t = -p_x, \quad p_z = 0\\
    u_x + w_z = 0,\\
    \text{with}\\
    w=\eta_t \quad p=\eta \quad \text{on}\;\; z= 1\\
    w = 0 \quad \text{on}\;\; z= 0.
    \end{drcases}
\end{align}
Modification to these equations on the boundary condition, i.e. on $z=1$
leaves us with
\begin{align}
    u = -p_x = -\eta_x \quad \Rightarrow \quad u_t + \eta_x = 0
    \label{eq:kdv1}\\
    w = -zu_x\Big|_{z=1} = -u_x = \eta_t \quad \Rightarrow \quad u_x + \eta_t
    =0.\label{eq:kdv2}
\end{align}
By doing differentiation \ref{eq:kdv1} with respect to $x$ and subtracting
the equation \ref{eq:kdv2} differentiated with respect to $t$ we get the
standard wave equation for the profile of the wave
\begin{align}
    \eta_{x x}  - \eta_{t t} = 0 .
\end{align}
We choose a solution for a right going wave and go into the frame of the
moving wave by a coordinate transformation as in the last section to $\xi =
x- t$. Additionally we want to introduce a long term variable, since we have
a uniformity as $t$ (or $x$) goes to infinity. This is usually done by
rescaling $t = \varepsilon \tau$. In summary we have that $\xi = O(1)$ as
well as $\tau = O(1)$. This is for  \textbf{far field region} of the wave, and
therefore the region, where we expect KdV type balance, between dispersion
and linearity. The fact of this matter can be rigorously proven, it needs to
be show that any sufficiently fast decaying smooth solution will eventually
split into a finite superposition of two solitary waves traveling to the
right and a decaying dispersive part traveling to the left. However will not
go into this here. To transform the equations in \ref{eq:kdv3}, we look at
the chain rule w.r.t $\xi ,\tau$ evolving to
\begin{align}
    \frac{\partial }{\partial t} &= -\frac{\partial }{\partial \xi}
    +\varepsilon \frac{\partial }{\partial \tau} \\
    \frac{\partial }{\partial x} &= \frac{\partial }{\partial \xi}.
\end{align}
Then we get
\begin{align}\label{eq:kdv5}
    \begin{drcases}
        -u_\xi + \varepsilon\left(u_\tau + uu_\xi + w u_z  \right)  =
        -p_\xi\\
        \varepsilon\left( -w_\xi + \varepsilon\left( w_\tau + u w_\xi + w w_z
        \right)  \right)  = - p_z\\
        u_\xi + w_z = 0\\
        \text{with}\\
        \begin{drcases}
        w = -\eta_\xi+\varepsilon(\eta_\tau+u \eta_\xi)\\
        p=\eta
        \end{drcases}
        \text{on} \;\; z=1+\varepsilon\eta\\
        \text{and}\\
        w = 0 \quad \;\; z = b =0.
    \end{drcases}
\end{align}
The crucial part now is to consider an asymptotic expansion of in
$\varepsilon$ for velocity of the fluid particles $u, w$ and also the wave
profile $\eta$ and for the pressure variable $p$. The general asymptotic
ansatz is of the form
\begin{align}
    q\left( \xi, \tau, z; \varepsilon \right)  = \sum_{n=0}^{\infty}
    \varepsilon^n q_n\left( \xi, \tau, z \right).
\end{align}
The first equation in \ref{eq:kdv5} up to the order of $O(\varepsilon^2)$ is
of the form
\begin{align}
    \varepsilon^0\left( p_{0\xi} - u_{0\xi}\right) + \varepsilon^1\left(
    p_{1\xi} - u_{1\xi} + u_{0\tau} + u_0 u_{0\xi} + w_0u_{0z} \right)
    +O(\varepsilon^2) = 0,
\end{align}
with the main condition $p_{0\xi} = u_{0\xi}$. For the second equation in
\ref{eq:kdv5} becomes
\begin{align}
    \varepsilon^0\left( p_{0z} \right)
    +\varepsilon^1\left( p_{1z}-w_{0\xi} + w_{0\tau} + u_0w_{0\xi}+w_0w_{0z} \right)
    + O(\varepsilon^2) = 0,
\end{align}
the main condition $p_{0z} =0 $. The third equation in \ref{eq:kdv5} is the
following
\begin{align}
    \varepsilon^0(u_{0\xi}+w_{0z}) + \varepsilon^1\left( u_{1\xi}+w_{1z}
    \right)
    O(\varepsilon^2) =0,
\end{align}
where the main condition satisfies $u_{n\xi} = -w_{n\xi}$ for all $n \in
\mathbb{N}$. Further the surface condition is expanded into
\begin{align}
    p_n = \eta_n \qquad \forall\ n \in \mathbb{N},
\end{align}
and
\begin{align}
    \varepsilon^0\left(w_0 + \eta_{0\xi}\right)+
    \varepsilon^1\left( w_1 + \eta_{1\xi} + \eta_{0\tau} + \eta_0 \eta_{0\xi}\right)
    + O(\varepsilon^2) = 0,
\end{align}
Do note that the condition for for $\varepsilon^0$ is $z=1$ and for
$\varepsilon^1$ is $z=\varepsilon\eta$. The main conclusion from the above
equation is however $w_0 = -\eta_{0\xi}$. And lastly the bottom condition
remains unchanged for all $n$ as
\begin{align}
    w_n = 0 \quad \text{on}\;\; z= b=0
\end{align}
In essence $O(\varepsilon^0)$ give us the equations
\begin{align}
    u_{0\xi}=p_{0\xi},\quad p_{0z} =0,\quad u_{0\xi} + w_{0z} = 0,
\end{align}
with
\begin{align}
    p_0 = \eta_0, \quad w_0 = -\eta_{0\xi} \quad \text{on}\;\; z=1\\
    w_0 = 0 \quad \text{on}\;\; z=b=0.
\end{align}
They lead us tot he following solution which satisfies the boundary
\begin{align}
    p_0 = \eta_0, \quad u_0 = \eta_0, \quad w_0 = -z\eta_{0\xi} \quad
    \forall\ z\in[0, 1].
\end{align}
Do notice who $w_0 = -z\eta_{0\xi}$ automatically satisfies the boundary
conditions for both $z=0$ and $z=1$. Before we go on to consider
$O(\varepsilon)$, we expand $u, w$ and $p$ around $z=1$ via Taylor expansion.
This makes only since in the case $\varepsilon\rightarrow 0$
\begin{align}
    \begin{drcases}
    p_0 + \varepsilon\eta_0 p_{0z} + \varepsilon p_1 = \eta_0
    \varepsilon\eta_1 + O(\varepsilon^2)\\
    w_0 +\varepsilon\eta_0w_{0z} + \varepsilon w_1 = -\eta_{0\xi} -
    \varepsilon\eta_{1\xi} + \varepsilon\left( \eta_0 + u_0 \eta_{0\xi}
    \right) +O\left(\varepsilon^2  \right).
    \end{drcases} \text{on}\;\; z=1
\end{align}
Right off the equations of order $O(\varepsilon^1)$ become
\begin{align}
    -u_{1\xi} + u_{0\tau} + u_0u_{0\xi} + w_{0}u_{0z} = -p_{1\xi},\\
    p_{1z} = w_{0\xi} \quad \text{and} \quad u_{1\xi} + w_{1z} = 0,
\end{align}
with the boundary conditions
\begin{align}
    \begin{drcases}
        p_1 + \eta_0 p_{0z} = \eta_1\\
        w_1 + \eta_0 w_{0z} = -\eta_{1\xi} + \eta_{0t} + u
    \end{drcases}
    \text{on}\;\; z=1\\
    w_1 = 0 \quad \text{on}\;\; z =b = 0.
\end{align}
Thus
\begin{align}
    &p_{1z} = w_{0\xi} = -z\eta_{0\xi}\\
    \Rightarrow &p_1 = -\frac{1}{2}z^2 \eta_{0\xi\xi} +c,
\end{align}
where $c$ is the integration constant, together with the boundary condition
on $z=1$ we get that
\begin{align}
    c = \eta_1 + \frac{1}{2} \eta_{0\xi\xi},
\end{align}
for $p_1$ leaving is with
\begin{align}
    p_1 = \frac{1}{2} \left( 1-z^2 \right)  \eta_{0\xi\xi} +\eta_1
\end{align}
As for the condition $w_{1z} = -u_{1\xi}$ we get
\begin{align}
    w_{1z} &= -u_{1\xi} = -p_{1\xi} - u_{0\tau} - u_0u_{0\xi} - u_0u_{0z} \\
    &=\frac{1}{2} (1-z^2)\eta_{0\xi\xi\xi} - \eta_{1\xi} -\eta_{0\tau}
    -\eta_{0\xi}.
\end{align}
By integration and evaluation on $z=1$ of the above equation we get
\begin{align}\label{eq:kdv6}
    w_1\Big|_{z=1} = -\frac{1}{3} \eta_{0\xi\xi\xi} - \eta_{1\xi} -
    \eta_{0\tau} -\eta_0\eta_{0\xi},
\end{align}
on the other hand we have the original boundary condition
\begin{align}\label{eq:kdv7}
    w_1\Big|_{z=1} = -\eta_{1\xi} + \eta_{0\tau} +2\eta_{0}\eta_{0\xi} .
\end{align}
Subtracting equation \ref{eq:kdv6} from \ref{eq:kdv7} we get the KdV equation
\begin{align}
    \frac{1}{3} \eta_{0\xi\xi\xi} - 2\eta_{0\tau} - 3\eta_0\eta_{0\xi} = 0.
\end{align}





