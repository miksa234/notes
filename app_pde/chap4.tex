\section{Modeling the 2004 Tsunami}
\subsection{Description}
On the 26. December 2004, time 7:58 a powerful earthquake generated a tsunami
killing more than 275000 people and leaving millions homeless. The
hypocenter of the earthquake was 30 km under the floor of the Indian Ocean,
100 km away from Sumatra, an island in Indonesia. The earthquake displaced
an enormous amount of water, sending tsunami waves westwards across the
Indian Ocean to Sri Lanka and India and eastwards across the Andaman Basin.
to Thailand and Indonesia. The Earthquake occurred over 10 minutes along a
1000km long roughly straight line. Thereby we can model the corresponding
fluid mechanics as 2 dimensional where in Cartesian coordinates the
propagation of the tsunami wave is in $x$ direction and the $z$ direction
pointing upwards perpendicular to the flat ocean surface. However the
modeling assumption for two dimensions for the region outside the Bay of
Bengal is not valid, since the diffraction around islands and reflection from
steep shores pays a major role in the influence of the wave mechanics. Coming
back to the $2004$ tsunami, which raised the ocean floor a few meters to the
west and lowering it a few meters to the east, displacing the tectonic pates.
The tsunami waves featured westwards a wave of elevation, meaning a wave of
high amplitude, followed by a wave of depression, a wave of long wavelength
hitting the coastal areas of Sri Lanka and India in roughly three hours,
propagating a distance of approximately $1600\ \text{km}$. On the other hand
eastwards featuring a first a wave of depression following a wave of
elevation, propagating $700\ \text{km}$ in roughly two hours with a maximal
amplitude of $10\ \text{m}$. Observations tell us that as the tsunami waves
reached the shore the shape of the initial disturbance was not altered, which
is supported by measurements by a radar altimeter two hours after the
earthquake showing first an wave of elevation and then a wave of depression
westwards and respectively vice versa eastwards. The conclusion is made that
the shape of the tsunami remained approximately constant. Additionally it
should be mentioned that the tsunami waves reach very high amplitudes due to
the diminishing depth effect as they approach the shore, yet at open sea are
barely noticeable. A boat on open sea positioned at high depth in the region
of the tsunami during which the tsunami waves passed, captured the raise
from $\pm 0.8\ \text{m}$ of the boat over a period of $10\ \text{min}$.
This means that the wavelength of the tsunami wave was about $100\
\text{km}$.
\subsection{Long Wave, Shallow Water}
At weakly nonlinear levels dispersion balances linearity/nonlinearity in
certain regimes, such balance is found in the KdV equation. Among the KdV
equation, the Carissa-Holm equation (CH) also features such balance since it
arises as a high order approximation to KdV. Further there is also the
regularized long wave equation usually called Benjamin–Bona–Mahony equation, or simply
BBM. It should be noted that KdV is a solitary wave while BBM is not.
Localized disturbances of flat water surfaces propagating without change of
form need to be two dimension waves of elevation symmetric about the crest.
The linear theory does not provide any approximation to solitary waves, only
nonlinear or weakly linear approximations. These are KdV, BBM. The KdV
equation is orbitaly stable, meaning the shape and form of the profile is
stable under small perturbations (also CH \& BBM). Each solitary wave retains
is local identity, where large waves are faster than small ones. Further the
solution of the KdV equation evolves into a set of solitary waves, with
tallest in from front followed by an oscillatory tail. The KdV is the
proper equation, for our modeling purposes of tsunami waves. The main
question arises if KdV enters the regime of validity, in our case for the
$2004$ tsunami in the Andaman Basin. Or in other words are the involved
geophysical scales leading to time and space scales, compatible with
the KdV weak nonlinearity balance.
\subsection{Governing equations}
The last two section show the derivation of the modeling equation for fluid
mechanics, following with nondimensionalisation and rescaling. The scaling
takes the same form as in section \ref{sec:nondim}, we only introduce the
parameter $\alpha = h_0\sqrt{gh_0} $ and get the following equations
\begin{align}
    \begin{drcases}
    u_x + w_z = 0\\
    u_t + \varepsilon(uu_x + wu_z) = -p_x\\
    \delta^2\left(w_t + \varepsilon(w w_x + w w_z)\right) = -p_y\\
    u_z - \delta^2 w_x = 0
    \end{drcases}
\end{align}
on $(x, z) \in \mathbb{R}\times [0, 1+\varepsilon \eta(x, t)]$, with boundary
conditions
\begin{align}
    \begin{drcases}
        p = \eta \\
        w = h_t + \varepsilon u h_x
    \end{drcases}
    \text{on}\;\; z = 1+\varepsilon \eta(x, t)\\
    w = 0 \quad \text{on}\;\; z = 0.
\end{align}
The KdV validity arises in the region $\varepsilon = O(\delta^2)$, we are
going to thereby rescale $\delta$ in favour of $\varepsilon = \beta
\delta^2$, where $\beta = O(1)$ as in equation \ref{eq:epsdelta}:
\begin{align}\label{eq:epsdelta}
    x \rightarrow \frac{\delta}{\sqrt{\varepsilon} }x, \quad t
    \rightarrow \frac{\delta}{\sqrt{\varepsilon} }t\quad,
    w \rightarrow \frac{\sqrt{\varepsilon} }{\delta}w.
\end{align}
This opens up the possibility to prove that provided a suitable length- \&
time-scales for some $\delta$ the KdV will arise as a valid approximation for
the evolution of the free surface waves. Given some $\varepsilon>0$ there
exists a time in the such that the KdV balance holds, where we introduce the
variables $\xi = x- t$ and $\tau = \varepsilon t$ with equation for the wave profile
\begin{align}
    \eta_\tau - \frac{3}{2} \eta_{\xi\xi\xi} + \frac{1}{6} \eta \eta_\xi = 0,
\end{align}
for $\xi \in \mathbb{R}$ and $\tau > 0 $ and the boundary condition $\eta(\xi
,0)$ of the initial profile at $\tau = t = 0$. On the basis of satellite
measurements for the Bay of Bengal we have
\begin{align}
    a = 1\ \text{km}, \quad \lambda = 100\ \text{km},\quad h_0 = 4\
    \text{km}.
\end{align}
giving us
\begin{align}
    \varepsilon = \frac{a}{h_0} = 25 \cdot 10^{-5}\\
    \delta = \frac{h_0}{\lambda} \simeq 4*10^{-2}
\end{align}
giving us a $\beta \simeq 6,4 = O(1)$ for $\varepsilon = \beta \delta^2$.
The main issue is if the KdV balance can occur within the geophysical scales
.The conditions $x-t = O(1)$ and $\tau = O(1)$ give
\begin{align}
    \frac{x - t \sqrt{gh_0} }{\lambda} = O(1),\quad \frac{\varepsilon t
    \sqrt{gh_0} }{\lambda} = O(1).
\end{align}
Combining the above equations, we have
\begin{align}
   & \frac{x}{\lambda} = O(\varepsilon^{-1})\\
    &x  = O(\varepsilon^{-1}\lambda)
\end{align}
For the tsunami wave in the Bay of Bengal westwards towards India and Sri
Lanka we have
\begin{align}
    \lambda = 10\ \text{km}, \quad \varepsilon = 25 \cdot 10^{-5},
\end{align}
therefore a propagation distance of $x \simeq 4 \cdot 10^{5}$ is needed for
the KdV to enter the range of validity. This is however not the case since
the tsunami waves propagated $1600\ \text{km}$ westwards.

For the wave in the Andaman Basin towards Indonesia and Thailand we have
\begin{align}
    h_0 = 1\ \text{km}, \quad a = 1\ \text{m}, \quad \lambda = 100\
    \text{km}.
\end{align}
Giving us the parameters
\begin{align}
    \varepsilon  = 10^{-3}, \quad \delta = 10^{-2}.
\end{align}
this satisfys the range of validity  $\varepsilon = O(\delta^2)$, with the
requiring length scale $x = 10^{5}$.

Setting $h_0 = 4\ \text{km}$ for the Bay of Bengal and $h_0 = 1\ \text{km}$
for the Andaman Basin we have that the westwards tsunami propagated at speed
$\sqrt{gh_0}  = 712 \frac{\text{km}}{\text{h}}$ hitting the shore in about
$2h\ 10min$, while the tsunami waves eastwards propagated at speed $356
\frac{\text{km}}{\text{h}}$ hitting the coast in $1h\ 57min$. These
predictions align with the observations. As the waves approach the shore, the
tsunami mechanics enter the region of long waves over variable depth. In this
case dispersion and their front steepness play an important role, where
faster wave fronts catch up to slower ones and result in large amplitudes
with devastating effects.








