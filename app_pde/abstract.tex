\begin{abstract}
    The aim of this project is to give a master student a general idea of
    fluid mechanics \cite{johnson_1997}, to further combine this knowledge
    into modeling a problem focusing on a tsunami, generated by an earthquake
    in the Indian Ocean in 2004 (''On the propagation of tsunami waves, with
    emphasis on the tsunami of 2004`` by Adrian Constantin
    \cite{constantin_tsunami}). In this regard the project focuses on
    inviscid water flow, where the mass density of water can be taken to be
    constant and qualitatively shows how to derive Euler's Equations of
    Motion using basic multivariable calculus. The whole problem of modeling
    water waves comes around to determining the wave profile at the surface
    $z = h(x,y,t)$, where the need to introduce boundary conditions on the
    governing equations comes into play. To derive model hierarchies for
    different regimes (e.g. shallow water, long-wave or small amplitude)
    dimensional analysis and scaling of the parameters together with
    asymptotic expansion becomes essential. Asymptotic analysis then gives a
    model for the regime of the solitary wave and the KdV equation which is
    the region $\varepsilon=O\left(\delta^2 \right)$, the key to modeling the
    tsunami wave before approaching the shore.
\end{abstract}
