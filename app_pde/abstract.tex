\begin{abstract}
    The aim of this project is to give a master student a general idea of
    fluid dynamic and further combining this knowledge to model a problem
    focusing on the tsunami, generated by an earthquake in the Andaman Basin
    in 2004. In this regard focusing on inviscid water flow, where  the mass
    density of water is taken to be constant. It is qualitatively show how to
    derive Euler's Equations of Motion using basic multivariable analysis.
    The whole problem of modeling water waves comes around to determining the
    wave profile at the surface $z = h(x,y,t)$, where the need
    to introduce boundary conditions on the governing equations comes into
    play. To derive model hierarchies for different regimes (e.g. shallow
    water, long-wave or small amplitude) dimensional analysis and scaling of
    the parameters together with asymptotic expansion becomes essential.
    Asymptotic analysis gives a model for the regime of the solitary wave and
    the KdV equation which is the region $\varepsilon=O\left(\delta \right)$,
    a key to modeling the tsunami wave before approaching the shore.
\end{abstract}
