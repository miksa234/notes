\documentclass[a4paper]{article}

\usepackage[T1]{fontenc}
\usepackage[utf8]{inputenc}
\usepackage{mlmodern}

%\usepackage{ngerman}	% Sprachanpassung Deutsch

\usepackage{graphicx}
\usepackage{geometry}
\geometry{a4paper, top=15mm}

\usepackage{subcaption}
\usepackage[shortlabels]{enumitem}
\usepackage{amssymb}
\usepackage{amsthm}
\usepackage{amsmath}
\usepackage{mathtools}
\usepackage{braket}
\usepackage{bbm}
\usepackage{graphicx}
\usepackage{float}
\usepackage{yhmath}
\usepackage{tikz}
\usepackage{scratch}
\usetikzlibrary{patterns,decorations.pathmorphing,positioning}
\usetikzlibrary{calc,decorations.markings}

\usepackage[backend=biber, sorting=none]{biblatex}
\addbibresource{cite.bib}

\usepackage[framemethod=TikZ]{mdframed}

\tikzstyle{titlered} =
    [draw=black, thick, fill=white,%
        text=black, rectangle,
        right, minimum height=.7cm]


\usepackage[colorlinks=true,naturalnames=true,plainpages=false,pdfpagelabels=true]{hyperref}
\usepackage[parfill]{parskip}
\usepackage{lipsum}

\usepackage{tcolorbox}
\tcbuselibrary{skins,breakable}

\pagestyle{myheadings}

\colorlet{colexam}{black}
\newcounter{definition}
\newtcolorbox[use counter=definition]{mydef}[1]{
    empty,
    title={\textbf{Definition~\thetcbcounter}~~(\textit{#1})},
    attach boxed title to top left,
    fontupper=\sl,
    boxed title style={
        empty,
        size=minimal,
        bottomrule=1pt,
        top=1pt,
        left skip=0cm,
        overlay=
            {\draw[colexam,line width=1pt]([yshift=-0.4cm]frame.north
        west)--([yshift=-0.4cm]frame.north east);}},
            coltitle=colexam,
            fonttitle=\normalfont,
            before=\par\medskip\noindent,
            parbox=false,
            boxsep=-1pt,
            left=0.75cm,
            right=3mm,
            top=4pt,
            breakable,
            pad at break*=0mm,
            vfill before first,
            overlay unbroken={
                \draw[colexam,line width=1pt]
                ([xshift=0.6cm, yshift=-0.5pt]frame.south
                west)--([xshift=0.6cm,yshift=-1pt]frame.north west)
                --([xshift=0.6cm]frame.south west)--([xshift=-13cm]frame.south east); },
            overlay first={
                \draw[colexam,line width=1pt]
                ([xshift=0.6cm, yshift=-0.5pt]frame.south
                west)--([xshift=0.6cm,yshift=-1pt]frame.north west)
                --([xshift=0.6cm]frame.south west); },
            overlay last={
                \draw[colexam,line width=1pt]
                ([xshift=0.6cm, yshift=-0.5pt]frame.south
                west)--([xshift=0.6cm,yshift=-1pt]frame.north west)
                --([xshift=0.6cm]frame.south west)--([xshift=-13cm]frame.south east); }
}
\newcounter{theorem}
\newtcolorbox[use counter=theorem]{theorem}{
    empty,
    title={Theorem ~\thetcbcounter},
    attach boxed title to top left,
    fontupper=\sl,
    boxed title style={
        empty,
        size=minimal,
        bottomrule=1pt,
        top=1pt,
        left skip=0cm,
        overlay=
            {\draw[colexam,line width=1pt]([yshift=-0.4cm]frame.north
        west)--([yshift=-0.4cm]frame.north east);}},
            coltitle=colexam,
            fonttitle=\bfseries,
            before=\par\medskip\noindent,
            parbox=false,
            boxsep=-1pt,
            left=0.75cm,
            right=3mm,
            top=4pt,
            breakable,
            pad at break*=0mm,
            vfill before first,
            overlay unbroken={
                \draw[colexam,line width=1pt]
                ([xshift=0.6cm, yshift=-0.5pt]frame.south
                west)--([xshift=0.6cm,yshift=-1pt]frame.north west)
                --([xshift=0.6cm]frame.south west)--([xshift=-13cm]frame.south east); },
            overlay first={
                \draw[colexam,line width=1pt]
                ([xshift=0.6cm, yshift=-0.5pt]frame.south
                west)--([xshift=0.6cm,yshift=-1pt]frame.north west)
                --([xshift=0.6cm]frame.south west); },
            overlay last={
                \draw[colexam,line width=1pt]
                ([xshift=0.6cm, yshift=-0.5pt]frame.south
                west)--([xshift=0.6cm,yshift=-1pt]frame.north west)
                --([xshift=0.6cm]frame.south west)--([xshift=-13cm]frame.south east); }
}
\newcounter{lemma}
\newtcolorbox[use counter=lemma]{lemma}{
    empty,
    title={Lemma~\thetcbcounter},
    attach boxed title to top left,
    fontupper=\sl,
    boxed title style={
        empty,
        size=minimal,
        bottomrule=1pt,
        top=1pt,
        left skip=0cm,
        overlay=
            {\draw[colexam,line width=1pt]([yshift=-0.4cm]frame.north
        west)--([yshift=-0.4cm]frame.north east);}},
            coltitle=colexam,
            fonttitle=\bfseries,
            before=\par\medskip\noindent,
            parbox=false,
            boxsep=-1pt,
            left=0.75cm,
            right=3mm,
            top=4pt,
            breakable,
            pad at break*=0mm,
            vfill before first,
            overlay unbroken={
                \draw[colexam,line width=1pt]
                ([xshift=0.6cm, yshift=-0.5pt]frame.south
                west)--([xshift=0.6cm,yshift=-1pt]frame.north west)
                --([xshift=0.6cm]frame.south west)--([xshift=-13cm]frame.south east); },
            overlay first={
                \draw[colexam,line width=1pt]
                ([xshift=0.6cm, yshift=-0.5pt]frame.south
                west)--([xshift=0.6cm,yshift=-1pt]frame.north west)
                --([xshift=0.6cm]frame.south west); },
            overlay last={
                \draw[colexam,line width=1pt]
                ([xshift=0.6cm, yshift=-0.5pt]frame.south
                west)--([xshift=0.6cm,yshift=-1pt]frame.north west)
                --([xshift=0.6cm]frame.south west)--([xshift=-13cm]frame.south east); }
}

\newcommand{\eps}{\varepsilon}
\usepackage[OT2,T1]{fontenc}
\DeclareSymbolFont{cyrletters}{OT2}{wncyr}{m}{n}
\DeclareMathSymbol{\Sha}{\mathalpha}{cyrletters}{"58}

\markright{Popović\hfill Seminar\hfill}


\title{University of Vienna\\
\vspace{1cm}Seminar:\\Joint RICAM Seminar\\
\vspace{0.5cm}
Summary of talk by Otmar Scherzer
}
\author{Milutin Popovic}


\begin{document}
\maketitle
\tableofcontents

\section{Sheet 5}
\subsection{Fourier Transform}
In this section we prove the linearity of the Fourier Transform $\mathcal{F}$ on
$L^1(\mathbb{R}^d)$. For $f, g \in L^1(\mathbb{R}^d)$ and $\lambda, \mu \in
\mathbb{R}$ the linearity condition for $\mathcal{F}$ is the following
\begin{align}
    \mathcal{F}(\lambda f + \mu g) = \lambda \mathcal{F}(f) + \mu
    \mathcal{F}(g).
\end{align}
We start by using the Fourier transform definition for $x, \xi \in \mathbb{R}^d$
\begin{align}
    \mathcal{F}(\lambda f + \mu g)(\xi) &= \int_{\mathbb{R}^d} (\lambda f(x)+
    \mu g(x)) e^{-2\pi i \langle x, \xi\rangle}\ dx =\\
    &= \int_{\mathbb{R}^d} \lambda f(x) e^{-2\pi i\langle x,\xi\rangle} + \mu
    g(x) e^{-2\pi i\langle x,\xi\rangle}\ dx =\\
    &= \int_{\mathbb{R}^d} \lambda f(x) e^{-2\pi i\langle x,\xi\rangle}\ dx+
    \int_{\mathbb{R}^d} \mu
    g(x) e^{-2\pi i\langle x,\xi\rangle}\ dx =\\
    &= \lambda \int_{\mathbb{R}^d} f(x) e^{-2\pi i\langle x,\xi\rangle}\ dx+
    \mu \int_{\mathbb{R}^d}
    g(x) e^{-2\pi i\langle x,\xi\rangle}\ dx =\\
    &= \lambda \mathcal{F}(f)(\xi) + \mu \mathcal{F}(g)(\xi)
\end{align}
\subsection{Identities of the Fourier transform}
The following are three identities of the Fourier transform

\begin{table}[h!]
\centering
\begin{tabular}{| l | c | c |}
\hline
  & $g(x)$ & $\hat{g}(\xi)$ \\ \hline \hline
(1) & $f(x-x_0)$ & $e^{-2\pi ix_0 \xi} \hat{f}(\xi)$ \\ \hline
(2) & $e^{2\pi i \xi_0 x} f(x)$ & $f(\xi - \xi_0)$ \\ \hline
(3) & $f(ax)$ & $\frac{1}{a} \hat{f}(\frac{\xi}{a})$\\ \hline
\end{tabular}
    \caption{Identities of the Fourier transform for $a > 0,
    \xi_0, x \in \mathbb{R}$}
\end{table}
We start with (1)
\begin{align}
    \widehat{f(x-x_0)}
    &= \int_\mathbb{R} f(x-x_0) e^{-2\pi i x \xi}\ dx=
    \;\;\;\;\;\; (y = x-x_0)\\
    &= \int_\mathbb{R} f(y) e^{-2\pi i (y+x_0) \xi}\
    dy=\\
    &= e^{-2\pi i x_0 \xi} \int_\mathbb{R}f(y)e^{-2\pi i y
    \xi}\ dy=\\
    &= e^{-2\pi i x_0 \xi} \hat{f}(\xi).
\end{align}
For (2) we have
\begin{align}
    \widehat{e^{2\pi i x \xi_0} f(x)}
    &= \int_\mathbb{R} e^{2\pi i x \xi_0} f(x) e^{-2\pi i x \xi}\ dx =\\
    &= \int_\mathbb{R} f(x) e^{-2\pi i x (\xi -\xi_0)}\ dx=\\
    &= \hat{f}(\xi - \xi_0).
\end{align}
For (3) we have
\begin{align}
    \widehat{f(ax)}
    &= \int_\mathbb{R} f(ax) e^{-2\pi i \xi x}\ dx = \qquad \text{sub:
        $(y=ax)$}\\
    &= \int_\mathbb{R} \frac{1}{a}f(y) e^{-2\pi i \frac{\xi}{a} y}\ dy=\\
    &= \frac{1}{a} \hat{f}\left(\frac{\xi}{a}\right).
\end{align}
\subsection{The Box-Function}
Consider the following Box-Function
\begin{align}
    \Pi(x) :=
    \begin{cases}
        1\;\;\;\;\;\; -\frac{3}{2} < x < \frac{1}{2}\\
        0\;\;\;\;\; \text{else}
    \end{cases}
\end{align}
The Fourier transform of this function is
\begin{align}
    \widehat{\Pi(x)}
    &= \int_\mathbb{R} \Pi(x) e^{-2\pi i x\xi}\ dx=\\
    &= \int_{-\frac{3}{2}}^{\frac{1}{2}} e^{-2\pi i x \xi}\ dx
    =\frac{-1}{2\pi i \xi} e^{-2\pi i x\xi}
    \bigg|_{-\frac{3}{2}}^{\frac{1}{2}}=\\
    &= \frac{1}{2\pi i \xi} \left(e^{3\pi i \xi} - e^{-\pi i \xi}\right)=\\
    &= \frac{e^{\pi i \xi}\sin(2\pi\xi)}{\pi \xi}.
\end{align}

%\printbibliography
\end{document}
