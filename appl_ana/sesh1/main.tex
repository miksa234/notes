\documentclass[a4paper]{article}


\usepackage[T1]{fontenc}
\usepackage[utf8]{inputenc}
\usepackage{mlmodern}

%\usepackage{ngerman}	% Sprachanpassung Deutsch

\usepackage{graphicx}
\usepackage{geometry}
\geometry{a4paper, top=15mm}

\usepackage{subcaption}
\usepackage[shortlabels]{enumitem}
\usepackage{amssymb}
\usepackage{amsthm}
\usepackage{mathtools}
\usepackage{braket}
\usepackage{bbm}
\usepackage{graphicx}
\usepackage{float}
\usepackage{yhmath}
\usepackage{tikz}
\usetikzlibrary{patterns,decorations.pathmorphing,positioning}
\usetikzlibrary{calc,decorations.markings}

%\usepackage[backend=biber, sorting=none]{biblatex}
%\addbibresource{uni.bib}

\usepackage[framemethod=TikZ]{mdframed}

\tikzstyle{titlered} =
    [draw=black, thick, fill=white,%
        text=black, rectangle,
        right, minimum height=.7cm]


\usepackage[colorlinks=true,naturalnames=true,plainpages=false,pdfpagelabels=true]{hyperref}
\usepackage[parfill]{parskip}
\usepackage{lipsum}


\usepackage{tcolorbox}
\tcbuselibrary{skins,breakable}

\pagestyle{myheadings}

\markright{Popovic\hfill Applied Analysis\hfill}


\title{University of Vienna\\ Faculty of Mathematics\\
\vspace{1cm}Applied Analysis Problems
}
\author{Milutin Popovic}

\begin{document}
\maketitle
\tableofcontents

\section{Sheet 1}

\subsection{Fall from high}
We consider a free fall ($\dot{x}(t=0)=0$) of an object with mass $20\
\text{kg}$ from a height
$x(0) = h = 20\; \text{km}$, such that the gravitational force depends on the hight $x(t)$ in
the following way
\begin{align}\label{eq: free fall}
    \ddot{x} = -g\frac{R^2}{(x(t) + R)^2},
\end{align}
where $R$ is the radius of the earth $R \approx 6000\; \text{km}$ and $g
\approx 9.91\ \frac{m}{s^2}$ is the gravitational acceleration on the surface
of the earth. For this problem there are two possible non-dimensionalisations,
but first let us rewrite the variables in terms of non-dimensional variables
and some dimensional constants, a priori let
\begin{align}
    t &= t_c \tau \;\;\; \text{and}\\
    x &= x_c \xi.
\end{align}
With the above ansatz we get the following second derivative in
time
\begin{align}
        \frac{d^2}{dt^2} &= \frac{1}{t_c^2}\frac{d^2}{d\tau^2} \\
        \Rightarrow \frac{d^2x}{dt^2} &= \frac{x_c}{t_c^2}
        \frac{d^2\xi}{d\tau^2},
\end{align}
and thus the initial conditions can be rewritten as
\begin{align}
    \xi(0) = \frac{h}{x_c},\\
    \dot{\xi} = 0.
\end{align}
Now we can rewrite the equation of the free fall in \ref{eq: free fall} in
terms of $\xi(\tau)$ as
\begin{align}
   \frac{x_c}{gt_c^2} \ddot{\xi} = -\frac{1}{(\frac{x_c}{R}\xi +1)^2}.
\end{align}
Thereby we have three dimensional constants constants $\Pi_1, \Pi_2, \Pi_3$,
as follows
\begin{align}
    \Pi_1 = \frac{x_c}{R}, \;\;\;\; \Pi_2 = \frac{h}{x_c}, \;\;\;\;
    \Pi_3 = \frac{x_c}{gt_c^2}.
\end{align}

The first scaling is done by reducing $\Pi_1$ and $\Pi_3$ to 1, by setting
\begin{align}
        x_c = R, \;\;\;\; t_c = \sqrt{\frac{R}{g}},
\end{align}
refolmulating the initial problem in equation \ref{eq: free fall} to
\begin{align}
    &\ddot{\xi} = -\frac{1}{(\xi + 1)^2},\;\;\;\;
    \text{with} \nonumber\\
    &\xi(0) = \frac{h}{R}, \;\;\;\; \dot{\xi}(0) = 0.
\end{align}
Reducing the problem, meaining letting $R \rightarrow \infty$ makes the first
initial condition $\xi(0) \rightarrow 0$. We can conclude that this scaling
is bad since it changes the initial condition in the reduced problem.

The second scaling option reduces $\Pi_2$ and $\Pi_3$ to 1, by setting
\begin{align}
        x_c = h, \;\;\;\; t_c = \sqrt{\frac{h}{g}},
\end{align}
refolmulating the initial problem in equation \ref{eq: free fall} to
\begin{align}
    &\ddot{\xi} = -\frac{1}{(\frac{h}{R}\xi + 1)^2},\;\;\;\;
    \text{with} \nonumber\\
    &\xi(0) = 1, \;\;\;\; \dot{\xi}(0) = 0.
\end{align}
By letting $R \rightarrow \infty$ we get the following reduced problem
\begin{align}\label{eq: free fall reduced}
    \ddot{\xi} = -1.
\end{align}
Integrating and solving for $\xi(\tau = \frac{T}{t_c}) = 0$ for when the
object hits the ground we get a familliar solution
\begin{align}
    T = \sqrt{\frac{2h}{g}}
\end{align}
Now in the reduced problem the time untill the object hits the ground is
shorter since the acceleration is constant, but in the original one the
acceleration increases as the object comes closer to earth.
Additionally we can calculate the velocity at impact we need to integrate the
reduced problem \ref{eq: free fall reduced} once and put in the initial
condition
\begin{align}
    \dot{\xi}(\tau = \frac{T}{t_c}) &= -\tau = -\sqrt{2} \\
    \text{and} \;\;\; \dot{x} &= \frac{x_c}{t_c}\dot{\xi} =
    \sqrt{gh}\; \dot{\xi}\\
    \Rightarrow \dot{x}(T) &= -\sqrt{2gh}
\end{align}
The vectical throw allows for different scaling because the initial
conditions are different, and thus the solution too $x(0) = 0$ and
$\dot{x}(0) = v$.

\subsection{Scaling The Van der Pol equation}
The Van der Pol equation is a perturbation of the oscillation equation
\begin{align}\label{eq: vanderpol}
    LC\frac{d^2I}{dt^2} + (-g_1C +3g_3CI^2)\frac{dI}{dt} = -I
\end{align}
with initial conditions
\begin{align}\label{eq: van initial}
    I(0) = I_0,\;\;\;\; \dot{I}(0) = 0.
\end{align}
where $I(t)$ is the current at a time $t$, $C$ is the capacity, $L$ is the
inductivity and $g_1, g_3$ are some parameters. The units of all the
parameters are
\begin{align}
    [LC] &= s^2\\
    [g_1C] &= s\\
    [g_3C] &= sA^{-2}
\end{align}
The oscillation equation is
\begin{align}
    CL\ddot{I} + I = 0.
\end{align}
Solvable by the exponential ansatz of $I = Ae^{\lambda t}$, where $\lambda=
\pm i \sqrt{\frac{1}{LC}}$, thereby
\begin{align}
    I(t) = A_1 e^{i\sqrt{\frac{1}{LC}}t} + A_2 e^{-i\sqrt{\frac{1}{LC}}t}.
\end{align}
With the initial conditions in equation \ref{eq: van initial} we get $A_1 =
A_2$ and thus the solution to the oscillation equation is
\begin{align}
    I(t) = I_0\cos(\frac{t}{\sqrt{LC}})
\end{align}
Now that we know the reduced problem and the solution to it, we may work with
the Van-Der-Pol equation \ref{eq: vanderpol}, by determining all possible
non-dimensionalisations. Let us begin by setting
\begin{align}
    I(t) = I_c\psi,\\
    t = t_c \tau,
\end{align}
where $I_c$ and $t_c$ have the dimension of $I(t)$ and $t$ accordingly and
$\psi(\tau)$ and $\tau$ are dimensionless
The first and second derivative in time is
\begin{align}
    \frac{d}{dt} &= \frac{1}{t_c}\frac{d}{d\tau}\\
    \frac{d^2}{dt^2} &= \frac{1}{t_c^2}\frac{d^2}{d\tau^2}.
\end{align}
We can rewrite the Van-Der-Pol equation in terms of $\psi$ and $\tau$
\begin{align}
    &\frac{LC}{t_c^2}\ddot{\psi} - \frac{g_1C}{t_c}\dot{\psi}
    \frac{3g_3CI_c}{t_c}\dot{\psi}\psi = -\psi\\
    &\psi(0) = \frac{I_0}{I_c} \;\;\;\; \dot{\psi}(0) = 0
\end{align}
There are a total of four constants that we can eliminate
\begin{align}
    \Pi_1 &= \frac{I_0}{I_c},\;\;\;\;
    \Pi_2 = \frac{LC}{t_c^2},\nonumber\\
    \Pi_3 &= \frac{-g_1C}{t_c},\;\;\;\;\;
    \Pi_4 = \frac{3g_3CI_C}{t_c}.
\end{align}
The first scaling is
\begin{align}
    I_c = I_0,\;\;\; t_c=\frac{1}{\sqrt{LC}}.
\end{align}
Thereby we get the following problem
\begin{align}
    &\ddot{\psi}
    -\sqrt{\frac{C}{L}}g_1\dot{\psi}+\sqrt{\frac{C}{L}}3g_3I_0\dot{\psi}\psi
    = -\psi\\
    &\psi(0) = 1 \;\;\;\; \dot{\psi}(0) = 0
\end{align}


The second scaling is
\begin{align}
    I_c = I_0,\;\;\; t_c=g_1C.
\end{align}
Thereby we get the following problem
\begin{align}
    &\frac{L}{g_1^2C}\ddot{\psi}
    -\dot{\psi}+\frac{3g_3}{g_1}\dot{\psi}\psi
    = -\psi\\
    &\psi(0) = 1 \;\;\;\; \dot{\psi}(0) = 0
\end{align}

The third scaling is
\begin{align}
    I_c = I_0,\;\;\; t_c=g_3CI_0.
\end{align}
Thereby we get the following problem
\begin{align}
    &\frac{L}{g_3^2CI_0^2}\ddot{\psi}
    -\frac{g_1}{g_3I_0}\dot{\psi}+3\dot{\psi}\psi
    = -\psi\\
    &\psi(0) = 1 \;\;\;\; \dot{\psi}(0) = 0
\end{align}
\subsection{Scale the Schrödinger Equation}
The well known Schrödinger equation that describes quantum physics of one
particle can be written as
\begin{align}
    &i\hbar \partial_t\psi = -\frac{\hbar}{2m}\Delta \psi + V\psi \nonumber\\
    &\psi(t=0) =\psi_0
\end{align}
where $\hbar$ is the Planks constant, $\psi=\psi(x, t)$ the wave function,
$m$ the mass and $V = V(x)$ the potential in which the wave function is. The
dimensions are
\begin{align}
    [\hbar] = Js, \;\;\;\; V = J,  \;\;\;\; [\psi]= m^{-d/2}
\end{align}
for the special dimension $d$. The standard scaling ansatz is
\begin{align}
    &\psi = \psi_c \phi \\
    &t = t_c \tau \;\;\;\; x = x_c \xi,
\end{align}
by that we get the following derivatives in time and in space
\begin{align}
    \partial_{x_i} &=\frac{1}{x_{(i)c}} \partial_{\psi_i} \\
    \partial^2_{x_i} &=\frac{1}{x_{(i)c}^2} \partial_{\psi_i}^2\\
    \partial_{t} &=\frac{1}{t_c} \partial_{\psi_i} \\
\end{align}
for $i = 1, 2, 3$, or depending on the dimension we are dealing with.

Let us consider $x\in \mathbb{R}^3$ and $V = 0$ to scale the equation. First
we now have
\begin{align}
    i \partial_t\psi = -\frac{\hbar}{2m}\Delta \psi
\end{align}
with the initial condition $\phi(0) = \frac{\psi_0}{\psi_c}$. With our scaling the equation turns out to be
\begin{align}
    \frac{i\hbar t_c}{2m}\frac{1}{||\vec{x}_c||^2}\Delta_{\vec{\xi}}\ \phi =
    \partial_\tau\ \phi.
\end{align}
The constants we get are
\begin{align}
    \Pi_1 = \frac{t_c\hbar}{2m}\frac{1}{||\vec{x}_c||^2}, \;\;\;\; \Pi_2 =
    \frac{\psi_0}{\psi_c}.
\end{align}

The simple choice of
\begin{align}
    \frac{1}{||\vec{x}_c||^2} = 1, \;\;\;\; \psi_c = \psi_0, \;\;\;\; t_c =
    \frac{2m}{\hbar}||\vec{x}_c||^2,
\end{align}
simplifies the Schrodinger equation without the potential to
\begin{align}
    i\Delta_{\vec{\xi}}\ \phi = \partial_\tau \phi,
\end{align}
with the initial condition $\phi(\tau=0) = 1$.

Now consider $V = 0$, $x\in[0, L]$ and $t \in [0, T]$, the Schrodinger
equation is the same only with one spacial dimension as above, we can set
\begin{align}
    \psi_c = \psi_0, \;\;\;\; x_c =L, \;\;\;\; t_c = \frac{2mL^2}{\hbar}.
\end{align}
Thus we get
\begin{align}
    i\partial_{\xi}^2 \phi = \partial_\tau \phi,
\end{align}
with the initial condition $\phi(\tau=0) = 1$.

As a last example let us consider the quantum harmonic oscillator, that is
$V(x) = m\omega^2 x^2$ for $x\in \mathbb{R}$, where $\omega$ is the
frequency, the equation is the following
\begin{align}
    i \hbar \partial_t \psi = -\frac{\hbar^2}{2m}\partial^2_x \psi
    m\omega^2x^2 \psi.
\end{align}
By inserting the scaling
\begin{align}
    i\partial_\tau \phi = -\frac{t_c\hbar}{2mx_c^2}\partial_\xi^2 \phi
    \frac{t_cm\omega^2x_c^2}{\hbar} \xi^2 \phi
\end{align}
The dimensional constants are
\begin{align}
    \Pi_1 = \frac{t_0\hbar}{mx_c^2},\;\;\;\;\Pi_2 =
    \frac{m\omega^2x_c^2t_c}{\hbar},\;\;\;\; \Pi_3 = \frac{\psi_0}{\psi_c}.
\end{align}
The choice of scaling is
\begin{align}
    \psi_c = \psi_0, \;\;\;\; t_c = \frac{1}{\omega}, \;\;\;\; x_c =
    \sqrt{\frac{\hbar}{m\omega}}.
\end{align}
Thereby getting the following problem
\begin{align}
    i\partial_\tau \phi = -\frac{1}{2} \partial_\xi^2 \phi +\xi^2 \phi
\end{align}
with $\phi(\tau = 0) = 1$.



%\printbibliography
\end{document}
