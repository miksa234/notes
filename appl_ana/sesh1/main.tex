\documentclass[a4paper]{article}


\usepackage[T1]{fontenc}
\usepackage[utf8]{inputenc}
\usepackage{mlmodern}

%\usepackage{ngerman}	% Sprachanpassung Deutsch

\usepackage{graphicx}
\usepackage{geometry}
\geometry{a4paper, top=15mm}

\usepackage{subcaption}
\usepackage[shortlabels]{enumitem}
\usepackage{amssymb}
\usepackage{amsthm}
\usepackage{mathtools}
\usepackage{braket}
\usepackage{bbm}
\usepackage{graphicx}
\usepackage{float}
\usepackage{yhmath}
\usepackage{tikz}
\usetikzlibrary{patterns,decorations.pathmorphing,positioning}
\usetikzlibrary{calc,decorations.markings}

%\usepackage[backend=biber, sorting=none]{biblatex}
%\addbibresource{uni.bib}

\usepackage[framemethod=TikZ]{mdframed}

\tikzstyle{titlered} =
    [draw=black, thick, fill=white,%
        text=black, rectangle,
        right, minimum height=.7cm]


\usepackage[colorlinks=true,naturalnames=true,plainpages=false,pdfpagelabels=true]{hyperref}
\usepackage[parfill]{parskip}
\usepackage{lipsum}


\usepackage{tcolorbox}
\tcbuselibrary{skins,breakable}

\pagestyle{myheadings}

\newcommand{\eps}{\varepsilon}

\markright{Popović\hfill Applied Analysis\hfill}


\title{University of Vienna\\ Faculty of Mathematics\\
\vspace{1cm}Applied Analysis Problems
}
\author{Milutin Popovic}

\begin{document}
\maketitle
\tableofcontents

\section{Sheet 1}

\subsection{Fall from high}
We consider a free fall ($\dot{x}(t=0)=0$) of an object with mass $20\
\text{kg}$ from a height $x(0) = h = 20\; \text{km}$, such that the
gravitational force depends on the height $x(t)$ in the following way
\begin{align}\label{eq: free fall}
    \ddot{x}(t) = -g\frac{R^2}{(x(t) + R)^2},
\end{align}
where $R$ is the radius of the earth $R \approx 6000\; \text{km}$ and $g
\approx 9.81\ \frac{m}{s^2}$ is the gravitational acceleration on the surface
of the earth. For this problem there are two possible non-dimensionalisations,
but first let us rewrite the variables in terms of non-dimensional variables
and some dimensional constants, a priori let
\begin{align}
    t &= t_c \tau \;\;\; \text{and}\\
    x &= x_c \xi.
\end{align}
With the above ansatz we get the following second derivative in
time
\begin{align}
        \frac{d^2}{dt^2} &= \frac{1}{t_c^2}\frac{d^2}{d\tau^2} \\
        \Rightarrow \frac{d^2x}{dt^2} &= \frac{x_c}{t_c^2}
        \frac{d^2\xi}{d\tau^2},
\end{align}
and thus the initial conditions can be rewritten as
\begin{align}
    \xi(0) = \frac{h}{x_c},\\
    \dot{\xi} = 0.
\end{align}
Now we can rewrite the equation of the free fall in \ref{eq: free fall} in
terms of $\xi(\tau)$ as
\begin{align}
   \frac{x_c}{gt_c^2} \ddot{\xi} = -\frac{1}{(\frac{x_c}{R}\xi +1)^2}.
\end{align}
Thereby we have three dimensional constants constants $\Pi_1, \Pi_2, \Pi_3$,
as follows
\begin{align}
    \Pi_1 = \frac{x_c}{R}, \;\;\;\; \Pi_2 = \frac{h}{x_c}, \;\;\;\;
    \Pi_3 = \frac{x_c}{gt_c^2}.
\end{align}

The first scaling is done by reducing $\Pi_1$ and $\Pi_3$ to 1, by setting
\begin{align}
        x_c = R, \;\;\;\; t_c = \sqrt{\frac{R}{g}},
\end{align}
reformulating the initial problem in equation \ref{eq: free fall} to
\begin{align}
    &\ddot{\xi} = -\frac{1}{(\xi + 1)^2},\;\;\;\;
    \text{with} \nonumber\\
    &\xi(0) = \frac{h}{R}, \;\;\;\; \dot{\xi}(0) = 0.
\end{align}
Reducing the problem, meaning if $\frac{h}{R} \rightarrow 0$ makes the first
initial condition $\xi(0) \rightarrow 0$. We can conclude that this scaling
is bad since it changes the initial condition in the reduced problem.

The second scaling option reduces $\Pi_2$ and $\Pi_3$ to 1, by setting
\begin{align}
        x_c = h, \;\;\;\; t_c = \sqrt{\frac{h}{g}},
\end{align}
reformulating the initial problem in equation \ref{eq: free fall} to
\begin{align}
    &\ddot{\xi} = -\frac{1}{(\frac{h}{R}\xi + 1)^2},\;\;\;\;
    \text{with} \nonumber\\
    &\xi(0) = 1, \;\;\;\; \dot{\xi}(0) = 0.
\end{align}
By letting $R \rightarrow \infty$ we get the following reduced problem
\begin{align}\label{eq: free fall reduced}
    \ddot{\xi} = -1.
\end{align}
Integrating and solving for $\xi(\tau = T\sqrt{\frac{g}{h}}) = 0$ for when the
object hits the ground we get a familiar solution
\begin{align}
    T = \sqrt{\frac{2h}{g}}
\end{align}
Note that in the reduced problem the time until the object hits the ground is
\textbf{(much) shorter} since the acceleration is at its maximum $\ddot{x}(t) =
g$ for all $t$.  Yet in the original problem the acceleration (gravitation
force) \textbf{increases} as the object comes \textbf{closer} to earth . For
instance, if we let an object fall down from the height $h = R$ then its
gravitational force (acceleration) at that height would be $\ddot{x}(0) =
g/2$ and upon landing on earth the gravitational force $\ddot{x}(T) = g$,
while in the reduced solution its gravitational force would be $\ddot{x}(t) =
g$ for all $t$.

Additionally we can calculate the velocity at impact we need to integrate the
reduced problem \ref{eq: free fall reduced} once and put in the initial
condition
\begin{align}
    \dot{\xi}(\tau = \frac{T}{t_c}) &= -\tau = -\sqrt{2} \\
    \text{and} \;\;\; \dot{x} &= \frac{x_c}{t_c}\dot{\xi} =
    \sqrt{gh}\; \dot{\xi}\\
    \Rightarrow \dot{x}(T) &= -\sqrt{2gh},
\end{align}
The result is exactly the same as we would get from energy conservation
\begin{align}
    \frac{m}{2}\dot{x}^2 = mgh \quad \Rightarrow \quad \dot{x} = \sqrt{2gh}.
\end{align}
The vertical throw allows for an additional scaling because the
initial conditions are different, $x(0) = 0$ and $\dot{x}(0) = v$. Thus
the solution too.

To summarize, the assumptions that used for modeling and simplifying the
equation are
\begin{itemize}
    \item no relativistic influence,
    \item closed system, no outside influence (gravitation of the sun, air
        resistance),
    \item spherical symmetry of the earth (thereby center of mass can be
        set in the middle of earth).
\end{itemize}
By looking at our assumptions a question arises:\textbf{Is it a good
approximation to replace the attractive force of the earth by the attraction
of the whole mass concentrated at the center?}.

To answer this question more or less simply we look at the Poisson's equation
for gravity,
\begin{align}
    \ddot{\vec{x}}(\vec{r}) = -\nabla \phi(\vec{r}) \\
    \Delta \phi = 4\pi G\varrho(\vec{r}).
\end{align}
for a gravitational potential $\phi$ and the mass density of earth
$\varrho$. We assume that \textbf{the earth can be approximated by a sphere}
and then we integrate both sides along the sphere (and use the Gaussian law
for integration)
\begin{align}
    \int_{S} \nabla \ddot{\vec{x}}\ dS =
    \int_{\partial S}\ddot{\vec{x}}\ d\vec{s} = -4\pi
    G \int_S\varrho(\vec{r})\ ds = -4\pi GM.
\end{align}
Obviously $\ddot{\vec{x}}$ and $d\vec{s}$ point in the same direction. We
choose (rotate) the coordinate system such hat $\ddot{\vec{x}} =
\ddot{x}\ \mathbf{\hat{n}}$ and $d\vec{s} = \mathbf{\hat{n}}\ ds$, thereby
we get
\begin{align}
    &\ddot{x}\int_{\partial S} ds = 4\pi r^2 \ddot{x},\\
    \Rightarrow &\ddot{x} = -\frac{GM}{r^2}.
\end{align}
The further derivation to get the exact equation of motion as in \ref{eq:
free fall}, we have to keep in mind that $r = x + R$, because by our
assumptions we are not in the sphere only outside or on the border $R$.
Lastly by reformulating the constants $gR^2 = GM$ gets us to our equation of
motion
\begin{align}
    \ddot{x}(t) = -g\frac{R^2}{(x(t) + R)^2}.
\end{align}

\subsection{Scaling The Van der Pol equation}
The Van der Pol equation is a perturbation of the oscillation equation
\begin{align}\label{eq: vanderpol}
    LC\frac{d^2I}{dt^2} + (-g_1C +3g_3CI^2)\frac{dI}{dt} = -I
\end{align}
with initial conditions
\begin{align}\label{eq: van initial}
    I(0) = I_0,\;\;\;\; \dot{I}(0) = 0.
\end{align}
where $I(t)$ is the current at a time $t$, $C$ is the capacity, $L$ is the
inductivity and $g_1, g_3$ are some parameters. The units of all the
parameters are
\begin{align}
    [LC] &= s^2\\
    [g_1C] &= s\\
    [g_3C] &= sA^{-2}
\end{align}
The oscillation equation is
\begin{align}
    CL\ddot{I} + I = 0.
\end{align}
Solvable by the exponential ansatz of $I = Ae^{\lambda t}$, where $\lambda=
\pm i \sqrt{\frac{1}{LC}}$, thereby
\begin{align}
    I(t) = A_1 e^{i\sqrt{\frac{1}{LC}}t} + A_2 e^{-i\sqrt{\frac{1}{LC}}t}.
\end{align}
With the initial conditions in equation \ref{eq: van initial} we get $A_1 =
A_2$ and thus the solution to the oscillation equation is
\begin{align}
    I(t) = I_0\cos(\frac{t}{\sqrt{LC}})
\end{align}
Now that we know the reduced problem and the solution to it, we may work with
the Van-Der-Pol equation \ref{eq: vanderpol}, by determining all possible
non-dimensionalisations. Let us begin by setting
\begin{align}
    I(t) = I_c\psi,\\
    t = t_c \tau,
\end{align}
where $I_c$ and $t_c$ have the dimension of $I(t)$ and $t$ accordingly and
$\psi(\tau)$ and $\tau$ are dimensionless
The \textbf{first} and second derivative in time is
\begin{align}
    \frac{d}{dt} &= \frac{1}{t_c}\frac{d}{d\tau}\\
    \frac{d^2}{dt^2} &= \frac{1}{t_c^2}\frac{d^2}{d\tau^2}.
\end{align}
We can rewrite the Van-Der-Pol equation in terms of $\psi$ and $\tau$
\begin{align}
    &\frac{LC}{t_c^2}\ddot{\psi} - \left(\frac{3g_3I_c^2}{g_1}\psi^2 -
    1\right)\frac{g_1C}{t_c}\dot{\psi}= -\psi\\
    &\psi(0) = \frac{I_0}{I_c} \;\;\;\; \dot{\psi}(0) = 0
\end{align}
There are a total of four constants that we can eliminate
\begin{align}
    \Pi_1 &= \frac{I_0}{I_c}, \qquad
    \Pi_2 = \frac{LC}{t_c^2},\nonumber\\
    \Pi_3 &= \frac{3g_3I_c^2}{g_1}, \qquad
    \Pi_4 = \frac{g_1C}{t_c}.
\end{align}
The \textbf{first} scaling is
\begin{align}
    I_c = \sqrt{\frac{g_1}{3g_3}},\;\;\; t_c=\sqrt{LC}.
\end{align}
Thereby we get the following problem
\begin{align}
    \ddot{\psi} + (\psi^2 - 1)\eps \dot{\psi} = -\psi, \qquad \psi(0) =
    \sqrt{\frac{3g_3}{g_1}}I_0,
\end{align}
where $\eps = g_1\sqrt{\frac{C}{L}}$.

The \textbf{second} scaling is
\begin{align}
    I_c = \sqrt{\frac{g_1}{3g_3}},\;\;\; t_c=g_1C.
\end{align}
Thereby we get the following
\begin{align}
    \eps \psi'' +(\psi^2 +1)\psi' = -\psi, \qquad \psi(0) =
    \sqrt{\frac{3g_3}{g_1}}I_0,
\end{align}
where $\eps = \frac{L}{g_1^2C}$. We could also consider scaling $I_c = I_0$
with $t_c = \sqrt{LC}$ or $t_c = g_1C$ but they wouldn't develop significant
model hierarchies like the above two scaling.
\subsection{Scale the Schrödinger Equation}
The well known Schrödinger equation that describes quantum physics of the one
particle system is
\begin{align}
    &i\hbar \partial_t\psi = -\frac{\hbar^2}{2m}\Delta \psi + V\psi \nonumber\\
    &\psi(t=0) =\psi_0
\end{align}
where $\hbar$ is the reduced Plank constant, $\psi=\psi(x, t)$ the wave function,
$m$ the mass and $V = V(x)$ the potential in which the wave function is. The
dimensions are
\begin{align}
    [\hbar] = js, \;\;\;\; [V] = j,  \;\;\;\; [\psi]= m^{-d/2}
\end{align}
where $d$ is the spacial dimension. The standard scaling ansatz is
\begin{align}
    &\psi = \psi_c \phi \\
    &t = t_c \tau \;\;\;\; x = x_c \xi,
\end{align}
by that we get the following derivatives in time and in space
\begin{align}
    \partial_{x_i} &=\frac{1}{x_{(i)c}} \partial_{\psi_i} \\
    \partial^2_{x_i} &=\frac{1}{x_{(i)c}^2} \partial_{\psi_i}^2\\
    \partial_{t} &=\frac{1}{t_c} \partial_{\psi_i} \\
\end{align}
for $i = 1, 2, 3$, or depending on the dimension we are dealing with.

Let us consider $x\in \mathbb{R}^3$ and $V = 0$ to scale the equation. First
we now have
\begin{align}
    i \partial_t\psi = -\frac{\hbar}{2m}\Delta \psi
\end{align}
with the initial condition $\phi(0) = \frac{\psi_0}{\psi_c}$. With our scaling the equation turns out to be
\begin{align}
    \frac{i\hbar t_c}{2m}\frac{1}{||\vec{x}_c||^2}\Delta_{\vec{\xi}}\ \phi =
    \partial_\tau\ \phi.
\end{align}
The constants we get are
\begin{align}
    \Pi_1 = \frac{t_c\hbar}{2m}\frac{1}{||\vec{x}_c||^2}, \;\;\;\; \Pi_2 =
    \frac{\psi_0}{\psi_c}.
\end{align}

The simple choice of
\begin{align}
    \frac{1}{||\vec{x}_c||^2} = 1, \;\;\;\; \psi_c = \psi_0, \;\;\;\; t_c =
    \frac{2m}{\hbar}||\vec{x}_c||^2,
\end{align}
simplifies the Schrodinger equation without the potential to
\begin{align}
    i\Delta_{\vec{\xi}}\ \phi = \partial_\tau \phi,
\end{align}
with the initial condition $\phi(\tau=0) = 1$.
.

Now consider $V = 0$, $x\in[0, L]$ and $t \in [0, T]$, the Schrodinger
equation is the same only with one spacial dimension as above, we can set
\begin{align}
    \psi_c = \psi_0, \;\;\;\; x_c =L, \;\;\;\; t_c = \frac{2mL^2}{\hbar}.
\end{align}
Thus we get
\begin{align}
    i\partial_{\xi}^2 \phi = \partial_\tau \phi,
\end{align}
with the initial condition $\phi(\tau=0) = 1$, where $\xi \in [0, 1]$ and
$\tau \in [0, \frac{\hbar T}{2mL^2}]$.
.

In the last example let us consider the quantum harmonic oscillator
represented by the potential $V(x) = m\omega^2 x^2$ for $x\in \mathbb{R}$,
where $\omega$ is the frequency. The equation is the following
\begin{align}
    i \hbar \partial_t \psi = -\frac{\hbar^2}{2m}\partial^2_x \psi
    +m\omega^2x^2 \psi.
\end{align}
By inserting the standard scaling ansatz we get
\begin{align}
    i\partial_\tau \phi = -\frac{t_c\hbar}{2mx_c^2}\partial_\xi^2 \phi
    +\frac{t_cm\omega^2x_c^2}{\hbar} \xi^2 \phi,
\end{align}
The dimensional constants are
\begin{align}
    \Pi_1 = \frac{t_0\hbar}{mx_c^2},\;\;\;\;\Pi_2 =
    \frac{m\omega^2x_c^2t_c}{\hbar},\;\;\;\; \Pi_3 = \frac{\psi_0}{\psi_c}.
\end{align}
The choice of scaling is
\begin{align}
    \psi_c = \psi_0, \;\;\;\; t_c = \frac{1}{\omega}, \;\;\;\; x_c =
    \sqrt{\frac{\hbar}{m\omega}}.
\end{align}
Thereby getting the following problem
\begin{align}
    i\partial_\tau \phi = -\frac{1}{2} \partial_\xi^2 \phi +\xi^2 \phi
\end{align}
with $\phi(\tau = 0) = 1$.

%\printbibliography
\end{document}
