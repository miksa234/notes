\documentclass[a4paper]{article}

\usepackage[T1]{fontenc}
\usepackage[utf8]{inputenc}
\usepackage{mlmodern}

%\usepackage{ngerman}	% Sprachanpassung Deutsch

\usepackage{graphicx}
\usepackage{geometry}
\geometry{a4paper, top=15mm}

\usepackage{subcaption}
\usepackage[shortlabels]{enumitem}
\usepackage{amssymb}
\usepackage{amsthm}
\usepackage{amsmath}
\usepackage{mathtools}
\usepackage{braket}
\usepackage{bbm}
\usepackage{graphicx}
\usepackage{float}
\usepackage{yhmath}
\usepackage{tikz}
\usepackage{scratch}
\usetikzlibrary{patterns,decorations.pathmorphing,positioning}
\usetikzlibrary{calc,decorations.markings}

\usepackage[backend=biber, sorting=none]{biblatex}
\addbibresource{cite.bib}

\usepackage[framemethod=TikZ]{mdframed}

\tikzstyle{titlered} =
    [draw=black, thick, fill=white,%
        text=black, rectangle,
        right, minimum height=.7cm]


\usepackage[colorlinks=true,naturalnames=true,plainpages=false,pdfpagelabels=true]{hyperref}
\usepackage[parfill]{parskip}
\usepackage{lipsum}

\usepackage{tcolorbox}
\tcbuselibrary{skins,breakable}

\pagestyle{myheadings}

\colorlet{colexam}{black}
\newcounter{definition}
\newtcolorbox[use counter=definition]{mydef}[1]{
    empty,
    title={\textbf{Definition~\thetcbcounter}~~(\textit{#1})},
    attach boxed title to top left,
    fontupper=\sl,
    boxed title style={
        empty,
        size=minimal,
        bottomrule=1pt,
        top=1pt,
        left skip=0cm,
        overlay=
            {\draw[colexam,line width=1pt]([yshift=-0.4cm]frame.north
        west)--([yshift=-0.4cm]frame.north east);}},
            coltitle=colexam,
            fonttitle=\normalfont,
            before=\par\medskip\noindent,
            parbox=false,
            boxsep=-1pt,
            left=0.75cm,
            right=3mm,
            top=4pt,
            breakable,
            pad at break*=0mm,
            vfill before first,
            overlay unbroken={
                \draw[colexam,line width=1pt]
                ([xshift=0.6cm, yshift=-0.5pt]frame.south
                west)--([xshift=0.6cm,yshift=-1pt]frame.north west)
                --([xshift=0.6cm]frame.south west)--([xshift=-13cm]frame.south east); },
            overlay first={
                \draw[colexam,line width=1pt]
                ([xshift=0.6cm, yshift=-0.5pt]frame.south
                west)--([xshift=0.6cm,yshift=-1pt]frame.north west)
                --([xshift=0.6cm]frame.south west); },
            overlay last={
                \draw[colexam,line width=1pt]
                ([xshift=0.6cm, yshift=-0.5pt]frame.south
                west)--([xshift=0.6cm,yshift=-1pt]frame.north west)
                --([xshift=0.6cm]frame.south west)--([xshift=-13cm]frame.south east); }
}
\newcounter{theorem}
\newtcolorbox[use counter=theorem]{theorem}{
    empty,
    title={Theorem ~\thetcbcounter},
    attach boxed title to top left,
    fontupper=\sl,
    boxed title style={
        empty,
        size=minimal,
        bottomrule=1pt,
        top=1pt,
        left skip=0cm,
        overlay=
            {\draw[colexam,line width=1pt]([yshift=-0.4cm]frame.north
        west)--([yshift=-0.4cm]frame.north east);}},
            coltitle=colexam,
            fonttitle=\bfseries,
            before=\par\medskip\noindent,
            parbox=false,
            boxsep=-1pt,
            left=0.75cm,
            right=3mm,
            top=4pt,
            breakable,
            pad at break*=0mm,
            vfill before first,
            overlay unbroken={
                \draw[colexam,line width=1pt]
                ([xshift=0.6cm, yshift=-0.5pt]frame.south
                west)--([xshift=0.6cm,yshift=-1pt]frame.north west)
                --([xshift=0.6cm]frame.south west)--([xshift=-13cm]frame.south east); },
            overlay first={
                \draw[colexam,line width=1pt]
                ([xshift=0.6cm, yshift=-0.5pt]frame.south
                west)--([xshift=0.6cm,yshift=-1pt]frame.north west)
                --([xshift=0.6cm]frame.south west); },
            overlay last={
                \draw[colexam,line width=1pt]
                ([xshift=0.6cm, yshift=-0.5pt]frame.south
                west)--([xshift=0.6cm,yshift=-1pt]frame.north west)
                --([xshift=0.6cm]frame.south west)--([xshift=-13cm]frame.south east); }
}
\newcounter{lemma}
\newtcolorbox[use counter=lemma]{lemma}{
    empty,
    title={Lemma~\thetcbcounter},
    attach boxed title to top left,
    fontupper=\sl,
    boxed title style={
        empty,
        size=minimal,
        bottomrule=1pt,
        top=1pt,
        left skip=0cm,
        overlay=
            {\draw[colexam,line width=1pt]([yshift=-0.4cm]frame.north
        west)--([yshift=-0.4cm]frame.north east);}},
            coltitle=colexam,
            fonttitle=\bfseries,
            before=\par\medskip\noindent,
            parbox=false,
            boxsep=-1pt,
            left=0.75cm,
            right=3mm,
            top=4pt,
            breakable,
            pad at break*=0mm,
            vfill before first,
            overlay unbroken={
                \draw[colexam,line width=1pt]
                ([xshift=0.6cm, yshift=-0.5pt]frame.south
                west)--([xshift=0.6cm,yshift=-1pt]frame.north west)
                --([xshift=0.6cm]frame.south west)--([xshift=-13cm]frame.south east); },
            overlay first={
                \draw[colexam,line width=1pt]
                ([xshift=0.6cm, yshift=-0.5pt]frame.south
                west)--([xshift=0.6cm,yshift=-1pt]frame.north west)
                --([xshift=0.6cm]frame.south west); },
            overlay last={
                \draw[colexam,line width=1pt]
                ([xshift=0.6cm, yshift=-0.5pt]frame.south
                west)--([xshift=0.6cm,yshift=-1pt]frame.north west)
                --([xshift=0.6cm]frame.south west)--([xshift=-13cm]frame.south east); }
}

\newcommand{\eps}{\varepsilon}
\usepackage[OT2,T1]{fontenc}
\DeclareSymbolFont{cyrletters}{OT2}{wncyr}{m}{n}
\DeclareMathSymbol{\Sha}{\mathalpha}{cyrletters}{"58}

\markright{Popović\hfill Seminar\hfill}


\title{University of Vienna\\
\vspace{1cm}Seminar:\\Joint RICAM Seminar\\
\vspace{0.5cm}
Summary of talk by Otmar Scherzer
}
\author{Milutin Popovic}


\begin{document}
\maketitle
\tableofcontents

\section{Sheet 6}
\subsection{Fourier Transform of the convolution}
Consider the function $f(x)$, which has a Fourier Transform $\hat{f}(\xi)$,
now let us compute the Fourier transform of
\begin{align}
    h(x) = f(3x-1) \sin(x) .
\end{align}
We know that the Fourier transform of the convolution is (we use somewhat of
the inverse convolution theorem).
\begin{align}
    \widehat{(f(3x-1)*g(x))} = \widehat{f(3x-1)} \cdot \hat{g}(\xi).
\end{align}
The Fourier transform of $f(3x-1)$ is simply done by substituting a new
variable
\begin{align}
    \widehat{f(3x-1)} = \frac{1}{3}e^{2\pi i\frac{\xi}{3}}\
    f\left(\frac{\xi}{3}\right).
\end{align}
The Fourier transform of $\sin(x)$ can be calculated when looking at the
Fourier transform of the Dirac-delta function
\begin{align}
    \widehat{\delta(ax-b)}
    &=\int_\mathbb{R} \delta(ax-b) e^{-2\pi i x \xi}\ dx
    \;\;\;\;\;\;\; (y = ax-b)\\
    &=\int_\mathbb{R} \delta(y) e^{-2\pi i (y+b)\frac{\xi}{a}}\frac{dy}{a}\\
    &=\frac{1}{a} e^{-2\pi i \xi \frac{b}{a}}.
\end{align}
We may plug in $\sin(x)$ in the definition of the Fourier transformation and
observe where we can use the Dirac-delta to to the inverse Fourier transform
\begin{align}
    \widehat{\sin(x)}
    &=\int_\mathbb{R} \sin(x)e^{-2\pi i x\xi}\ dx=\\
    &=\frac{1}{2i}\int_\mathbb{R} (e^{ix} - e^{-ix})e^{-2\pi i \xi x}\ dx\\
    &=\frac{1}{2i}\left(
        \int_\mathbb{R} e^{ix} e^{-2\pi i \xi x}\ dx+
        \int_\mathbb{R} e^{-ix} e^{-2\pi i \xi x}\ dx
        \right).
\end{align}
Here we may use the above formula for the Fourier transform of the Dirac
delta. We choose $a=1$, $b= \pm \frac{1}{2\pi}$ and do some $y=-x$
substitutions and thereby get the following result
\begin{align}
    \widehat{\sin(x)} = \frac{1}{2i} \left(
        \delta(\xi - \frac{1}{2\pi})
        -\delta(\xi + \frac{1}{2\pi})
        \right)
\end{align}
The whole result is thereby
\begin{align}
    \widehat{f(3x-1)} * \widehat{sin(x)}
    =& \frac{1}{6i} \bigg(
        e^{2\pi
        i(\frac{\xi}{3}-\frac{1}{6\pi})}\hat{f}\big(\frac{\xi}{3}-\frac{1}{6\pi}\big)-
        e^{2\pi
            i(\frac{\xi}{3}+\frac{1}{6\pi})}\hat{f}\big(\frac{\xi}{3}+\frac{1}{6\pi}\big)
        \bigg)
\end{align}
\subsection{More Fourier Transforms}
Consider the function
\begin{align}
    f(x) = e^{-|x|}
\end{align}
The Fourier transform of this function is
\begin{align}
    \hat{f}(\xi)
    &=\int_\mathbb{R} e^{-|x| e^{-2\pi i x \xi}}\ dx\\
    &= \int_{-\infty}^0 e^x e^{-2\pi i x \xi}\ dx
    + \int_0^\infty e^{-x} e^{-2\pi i x \xi}\ dx=\\
    &= \frac{1}{1-2\pi i \xi} e^{(1-2\pi i \xi) x}\bigg|_{-\infty}^0+
        \frac{-1}{1+2\pi i \xi} e^{-(1+2\pi i \xi) x}\bigg|_{-\infty}^0 = \\
    &= \frac{1}{1-2\pi i \xi} + \frac{1}{1 + 2\pi i \xi} =\\
    &= \frac{2}{1+(2\pi \xi)^2}.
\end{align}
Let us use this result to solve the following integral
\begin{align}
    \int_\mathbb{R} \frac{\cos(a\xi)}{(2\pi \xi)^2 + 1}\ d\xi =
    \frac{1}{2}\int_\mathbb{R} \hat{f}(\xi) \text{Re}(e^{ia\xi})\ dx,\\
\end{align}
where we used the fact that $\text{Re}(e^{ia\xi}) = \cos(a\xi)$ and
$\hat{f}(\xi) = \frac{2}{1+(2\pi \xi)^2}$, thereby
\begin{align}
    \frac{1}{2}\int_\mathbb{R} \hat{f}(\xi) \text{Re}(e^{ia\xi})\ dx
    &= \frac{1}{2}\text{Re}\left(
        \int_\mathbb{R}\hat{f}(\xi)e^{ia\xi}\ d\xi
    \right)=\\
    &= \frac{1}{2}\text{Re}\left(
        \int_\mathbb{R} \hat{f}(\xi) e^{2\pi i \frac{a}{2\pi}\xi}\ d\xi
    \right)=\\
    &= \frac{1}{2}\text{Re}\left(f(\frac{a}{2\pi})\right)=\\
    &= \frac{1}{2} e^{-\frac{|a|}{2\pi}}.
\end{align}
\subsection{Finite discrete Fourier transform}
Consider $s\in \mathbb{C}^N$ with entries
\begin{align}
    s[n] = \sin\left(2\pi\xi_0\frac{n}{N}\right),
\end{align}
for same $0 < \xi_0 < N$. The finite discrete Fourier transform of $s$ is
\begin{align}
    \hat{s}[k] &= \frac{1}{N} \sum_{n=0}^{N-1} \sin\left(2\pi\xi_0\frac{n}{N}\right)
            e^{-2\pi i \frac{k}{N} n}  =\\
        &=\frac{1}{2iN}\left(
            \sum_{n=0}^{N-1}e^{2\pi i \frac{n}{N}(\xi_0 -k)} - e^{-2\pi i
            \frac{n}{N}(\xi_0 +k)}
            \right).
\end{align}
If we consider $\xi_0 \in \mathbb{Z}$, we have
\begin{align}
    \hat{s}[k] =
    \begin{cases}
        \frac{1}{2i}\;\;\;\;\;\; \xi_0 = k\\
        -\frac{1}{2i}\;\;\;\;\;\; \xi_0 = -k\\
        0   \;\;\;\;\;\; \text{else}
    \end{cases}
\end{align}
\subsection{Discrete Matrix Notation}
The convolution of two vectors $f, g \in \mathbb{C}^N$, can be expressed by a
circulate matrix applied to f
\begin{align}
    (f * g) [n] = \sum_{k=0}^{N-1} f[k] g[n-k].
\end{align}
Consider $g=s$, then the matrix takes the following values
\begin{align}
    s[n-k] = s_{nk} = \sin\left(2\pi \xi_0 \frac{n-k}{N}\right).
\end{align}
The convolution with an impulse input $f=\delta_{0k}$, a vector that is $1$
for $k=0$ and else 0 reads
\begin{align}
    \sum_k s_{nk}f_k &= \sum_k s_{nk} \delta_{0k} =\\
            &= \sin\left(2\pi \xi_0 \frac{n}{N}\right).
\end{align}

%\printbibliography
\end{document}
