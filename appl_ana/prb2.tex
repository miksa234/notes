\documentclass[a4paper]{article}

\usepackage[T1]{fontenc}
\usepackage[utf8]{inputenc}
\usepackage{mlmodern}

%\usepackage{ngerman}	% Sprachanpassung Deutsch

\usepackage{graphicx}
\usepackage{geometry}
\geometry{a4paper, top=15mm}

\usepackage{subcaption}
\usepackage[shortlabels]{enumitem}
\usepackage{amssymb}
\usepackage{amsthm}
\usepackage{amsmath}
\usepackage{mathtools}
\usepackage{braket}
\usepackage{bbm}
\usepackage{graphicx}
\usepackage{float}
\usepackage{yhmath}
\usepackage{tikz}
\usepackage{scratch}
\usetikzlibrary{patterns,decorations.pathmorphing,positioning}
\usetikzlibrary{calc,decorations.markings}

\usepackage[backend=biber, sorting=none]{biblatex}
\addbibresource{cite.bib}

\usepackage[framemethod=TikZ]{mdframed}

\tikzstyle{titlered} =
    [draw=black, thick, fill=white,%
        text=black, rectangle,
        right, minimum height=.7cm]


\usepackage[colorlinks=true,naturalnames=true,plainpages=false,pdfpagelabels=true]{hyperref}
\usepackage[parfill]{parskip}
\usepackage{lipsum}

\usepackage{tcolorbox}
\tcbuselibrary{skins,breakable}

\pagestyle{myheadings}

\colorlet{colexam}{black}
\newcounter{definition}
\newtcolorbox[use counter=definition]{mydef}[1]{
    empty,
    title={\textbf{Definition~\thetcbcounter}~~(\textit{#1})},
    attach boxed title to top left,
    fontupper=\sl,
    boxed title style={
        empty,
        size=minimal,
        bottomrule=1pt,
        top=1pt,
        left skip=0cm,
        overlay=
            {\draw[colexam,line width=1pt]([yshift=-0.4cm]frame.north
        west)--([yshift=-0.4cm]frame.north east);}},
            coltitle=colexam,
            fonttitle=\normalfont,
            before=\par\medskip\noindent,
            parbox=false,
            boxsep=-1pt,
            left=0.75cm,
            right=3mm,
            top=4pt,
            breakable,
            pad at break*=0mm,
            vfill before first,
            overlay unbroken={
                \draw[colexam,line width=1pt]
                ([xshift=0.6cm, yshift=-0.5pt]frame.south
                west)--([xshift=0.6cm,yshift=-1pt]frame.north west)
                --([xshift=0.6cm]frame.south west)--([xshift=-13cm]frame.south east); },
            overlay first={
                \draw[colexam,line width=1pt]
                ([xshift=0.6cm, yshift=-0.5pt]frame.south
                west)--([xshift=0.6cm,yshift=-1pt]frame.north west)
                --([xshift=0.6cm]frame.south west); },
            overlay last={
                \draw[colexam,line width=1pt]
                ([xshift=0.6cm, yshift=-0.5pt]frame.south
                west)--([xshift=0.6cm,yshift=-1pt]frame.north west)
                --([xshift=0.6cm]frame.south west)--([xshift=-13cm]frame.south east); }
}
\newcounter{theorem}
\newtcolorbox[use counter=theorem]{theorem}{
    empty,
    title={Theorem ~\thetcbcounter},
    attach boxed title to top left,
    fontupper=\sl,
    boxed title style={
        empty,
        size=minimal,
        bottomrule=1pt,
        top=1pt,
        left skip=0cm,
        overlay=
            {\draw[colexam,line width=1pt]([yshift=-0.4cm]frame.north
        west)--([yshift=-0.4cm]frame.north east);}},
            coltitle=colexam,
            fonttitle=\bfseries,
            before=\par\medskip\noindent,
            parbox=false,
            boxsep=-1pt,
            left=0.75cm,
            right=3mm,
            top=4pt,
            breakable,
            pad at break*=0mm,
            vfill before first,
            overlay unbroken={
                \draw[colexam,line width=1pt]
                ([xshift=0.6cm, yshift=-0.5pt]frame.south
                west)--([xshift=0.6cm,yshift=-1pt]frame.north west)
                --([xshift=0.6cm]frame.south west)--([xshift=-13cm]frame.south east); },
            overlay first={
                \draw[colexam,line width=1pt]
                ([xshift=0.6cm, yshift=-0.5pt]frame.south
                west)--([xshift=0.6cm,yshift=-1pt]frame.north west)
                --([xshift=0.6cm]frame.south west); },
            overlay last={
                \draw[colexam,line width=1pt]
                ([xshift=0.6cm, yshift=-0.5pt]frame.south
                west)--([xshift=0.6cm,yshift=-1pt]frame.north west)
                --([xshift=0.6cm]frame.south west)--([xshift=-13cm]frame.south east); }
}
\newcounter{lemma}
\newtcolorbox[use counter=lemma]{lemma}{
    empty,
    title={Lemma~\thetcbcounter},
    attach boxed title to top left,
    fontupper=\sl,
    boxed title style={
        empty,
        size=minimal,
        bottomrule=1pt,
        top=1pt,
        left skip=0cm,
        overlay=
            {\draw[colexam,line width=1pt]([yshift=-0.4cm]frame.north
        west)--([yshift=-0.4cm]frame.north east);}},
            coltitle=colexam,
            fonttitle=\bfseries,
            before=\par\medskip\noindent,
            parbox=false,
            boxsep=-1pt,
            left=0.75cm,
            right=3mm,
            top=4pt,
            breakable,
            pad at break*=0mm,
            vfill before first,
            overlay unbroken={
                \draw[colexam,line width=1pt]
                ([xshift=0.6cm, yshift=-0.5pt]frame.south
                west)--([xshift=0.6cm,yshift=-1pt]frame.north west)
                --([xshift=0.6cm]frame.south west)--([xshift=-13cm]frame.south east); },
            overlay first={
                \draw[colexam,line width=1pt]
                ([xshift=0.6cm, yshift=-0.5pt]frame.south
                west)--([xshift=0.6cm,yshift=-1pt]frame.north west)
                --([xshift=0.6cm]frame.south west); },
            overlay last={
                \draw[colexam,line width=1pt]
                ([xshift=0.6cm, yshift=-0.5pt]frame.south
                west)--([xshift=0.6cm,yshift=-1pt]frame.north west)
                --([xshift=0.6cm]frame.south west)--([xshift=-13cm]frame.south east); }
}

\newcommand{\eps}{\varepsilon}
\usepackage[OT2,T1]{fontenc}
\DeclareSymbolFont{cyrletters}{OT2}{wncyr}{m}{n}
\DeclareMathSymbol{\Sha}{\mathalpha}{cyrletters}{"58}

\markright{Popović\hfill Seminar\hfill}


\title{University of Vienna\\
\vspace{1cm}Seminar:\\Joint RICAM Seminar\\
\vspace{0.5cm}
Summary of talk by Otmar Scherzer
}
\author{Milutin Popovic}


\begin{document}
\maketitle
\tableofcontents

\section{Sheet 2}
\subsection{Problem 4}
We consider a quadratic equation with two ways to perturb it by $\eps$:
\begin{align}
    x^2 + 2\eps x -1 = 0, \label{eq: (1)}\\
    \nonumber \\
    \eps x^2 + 2x - 1 = 0.\label{eq: (2)}
\end{align}
Equation \ref{eq: (2)} is singular, because the reduced problem ($\eps
\rightarrow 0$) has only one solution at $x = \frac{1}{2}$. While the reduced
problem in \ref{eq: (1)} has two solutions for $x = \pm 1$, which is the case
for this non reduced equation. Let us thereby calculate the asymptotic
expansion of the regular case up to $O(\eps^2)$, we take the ansatz for the
asymptotic expansion
\begin{align}\label{eq: p4 ansatz}
    x_\eps = x_0 + \eps x_1 + \eps^2 x_2 + O(\eps^3).
\end{align}
By substituting $x_\eps$ into \ref{eq: (1)} and factoring out the orders of
$\eps$ we get
\begin{align}
    \eps^0 (x_0^2 - 1) + \eps^1(2x_0 + 2x_0x_1) + \eps^2(x_1^2+2x_2x_0
    +2x_1) + O(\eps^3) = 0
\end{align}
By solving the equations in order of $\eps$, for the coefficients
$x_0$, $x_1$ and $x_2$ we get
\begin{align}
    x_0 = \pm 1, \;\;\;\; x_1 = -1, \;\;\;\; x_2 = \pm \frac{1}{2}.
\end{align}
By substituting into the equation \ref{eq: p4 ansatz} we get
\begin{align}
    x_\eps = \pm 1 - \eps \pm \frac{1}{2} \eps + O(\eps^3).
\end{align}
For $\eps = 0.001$ we get
\begin{align}
    &x_\eps = -1.0010005 + O(\eps^3),  &x_\eps = 0.9990005 + O(\eps^3),\\
    &x_\eps = -1.001 + O(\eps^2),      &x_\eps = 0.999 + O(\eps^2).
\end{align}
\subsection{Problem 5}
Consider the following equations
\begin{align}
    \label{eq: p5 1}\eps y' + y = x \;\;\;\;\;\; y(0) = 1\\
    \label{eq: p5 2}\eps y' + y = x \;\;\;\;\;\; y(0) = 0\\
    \nonumber\\
    \label{eq: p5 3}\eps y' + y = x \;\;\;\;\;\; y(0) = \eps\\
    \label{eq: p5 4}\eps^2 y' + y = x \;\;\;\;\;\; y(0) = \eps\\
    \nonumber\\
    \label{eq: p5 5}y' + \eps y = x \;\;\;\;\;\; y(0) = 1\\
    \label{eq: p5 6}y' + y = \eps x \;\;\;\;\;\; y(0) = 1
\end{align}
We will go through the equations and elaborate on if the perturbation is
regular or singular, if regular we will compute the asymptotic expansion up
to second order.
Let us begin with equation \ref{eq: p5 1}. By the first look, the reduced
problem does not agree with the boundary condition
\begin{align}
    y_0 = x \;\;\;\;\; y_0(0) = 1,
\end{align}
is a contradiction in $y_0(0) = 0 \neq 1$, thereby equation \ref{eq: p5 1} is
\textbf{singularly perturbed}.

The reduced problem of equation \ref{eq: p5 2} on the other hand agrees with
the boundary condition, since
\begin{align}
    y_0 = x \;\;\;\;\; y_0(0) = 0.
\end{align}
But by doing the ansatz for the asymptotic expansion
\begin{align}
    y_\eps(x) = y_0 + \eps y_1 + \eps^2 y_2 + O(\eps^3),
\end{align}
plugging in into \ref{eq: p5 2} and separating coefficients in terms of
$\eps$, we get
\begin{align}
    \eps^0 (y_0 -x) + \eps^1(y_0' + y_1) + \eps^2(y_1' + y_2) + O(\eps^3) = 0
\end{align}
The solutions to these equations are
\begin{align}
    y_0 = x, \;\;\;\; y_1 = 1, \;\;\;\; y_2 = 0,
\end{align}
which is a contradiction to the boundary condition of $y_1(0) = 1 \neq 0$.
Thereby we can conclude that equation \ref{eq: p5 2} is \textbf{singularly
perturbed}.

Next up is equation \ref{eq: p5 3}, where by the asymptotic expansion the
first order coefficient of $\eps$, $y_2$, has the boundary condition $y_2(0)
= 0$. But by applying the ansatz of the asymptotic expansion and plugging
into the equation we get
\begin{align}
    \eps^2(y_0 - x) + \eps^1 (y_0' + y_1) + \eps^2(y_1' + y_2) + O(\eps^3) =
    0.
\end{align}
Solving these equations we get
\begin{align}
    y_0 = 0, \;\;\;\; y_1 = 1 \;\;\;\; y_2 = 0 ,
\end{align}
which is a contradiction $y_1(0) = 1 \neq 0 $, thus the equation \ref{eq: p5
3} is \textbf{singularly perturbed}.

The next equation \ref{eq: p5 4} is also singularly perturbed, we
can see this by plugging the asymptotic expansion into the equation
\begin{align}
 \eps^0 ( y_0 - x) + \eps^1(y_1) + \eps^2(y_0' + y_2) = O(\eps^3),
\end{align}
solving for the coefficients we get
\begin{align}
    y_0 = x, \;\;\;\; y_1 = 0, \;\;\;\;\; y_2 = -1,
\end{align}
which is contradiction by the boundary condition $y_2(0) = -1 \neq 0$,
thereby \ref{eq: p5 4} is \textbf{singularly perturbed}.

Equation \ref{eq: p5 5} on the first sight does not indicate for any
contradictions, we may plug the ansatz of the asymptotic expansion into the
equation and see what happens
\begin{align}
    \eps^0(y_0 -x) + \eps^1(y_1' + y_0) + \eps^2(y_2' +y_1) + O(\eps^2) = 0,
\end{align}
with the initial conditions $y_0(0) = 1$, $y_1(0) = y_2(0) = 0$.
\begin{align}
    y_0 = \frac{x^2}{2} + 1, \;\;\;\;
    y_1 = -\frac{x^3}{6} + x, \;\;\;\;
    y_2 = \frac{x^4}{24} + \frac{x^2}{2}.
\end{align}
Finally we get
\begin{align}
    y_\eps(x) = (\frac{x^2}{2}+1) + \eps(-\frac{x^3}{6} -x)
    +\eps^2(\frac{x^4}{24} + \frac{x^2}{2}) + O(\eps^3).
\end{align}
Thereby we can conclude that \ref{eq: p5 5} is \textbf{regularly perturbed}.

The last equation \ref{eq: p5 6} is also regular, let us do the asymptotic
expansion of the equation and order the equation in orders of $\eps$.
\begin{align}
    \eps^0(y_0' + y_0) + \eps^1(y_1' + y_1 -x) + \eps^2(y_2' + y_2)+
    O(\eps^3) = 0 .
\end{align}
by solving these differential equations with the boundary conditions $y_0(0)
= 1$, $y_1(0) = y_2(0) = 0$ we get
\begin{align}
    y_0 =  e^{-x} \;\;\;\; y_1 = (x-1) + e^{-x} \;\;\;\; y_2 = 0.
\end{align}
The equation we get
\begin{align}
    y_\eps(x) = e^{-x} + \eps(x-1 +e^{-x}) + O(\eps^3).
\end{align}
Thereby we can conclude that the last equation \ref{eq: p5 6} is
\textbf{regularly perturbed}.
\subsection{Problem 6}
In this section we will calculate the asymptotic expansion of a regularly
perturbed equation in two ways, by doing the regular expansion ansatz and by
substituting and expanding in terms of $\eps$. The ordinary differential
equation we are dealing with is
\begin{align}
    y' = -y + \eps y^2 \;\;\;\;\; y(0) = 1,
\end{align}
where $t > 0$ and $0 < \eps \ll 1$. The standard expansion ansatz is
\begin{align}
y_\eps(x) = y_0 + \eps y_1 + \eps^2 y_2 + O(\eps^3).
\end{align}
The ODE then expands to
\begin{align}
    \eps^0(y_0' + y_0) + \eps(y_1' + y_1 - y_0^2) + \eps^2(y_2' + y_2 -
    2y_0y_1) + O(\eps^3) = 0.
\end{align}
Equations in order of $\eps$ and $\eps^2$ are non-homogeneous ODE's. The
solution to these three coefficients with the boundary conditions $y_0(0) =
1$, $y_1(0) =
y_2(0) = 0$ we get
\begin{align}
    y_0 = e^{-x}, \;\;\;\; y_1 = -e^{-2x} + e^{-x}, \;\;\;\; y_2 = e^{-3x} -
    2e^{-2x} + e^{-x}.
\end{align}
The expansion of $y$ is then
\begin{align}
    y_\eps (x) = e^{-x} + \eps(-e^{-2x} + e^{-x}) + \eps^2(e^{-3x} - 2e^{-2x}
+ e^{-x}) + O(\eps^3).  \end{align}

The second ansatz, considers the substitution $z = \frac{1}{y}$, by
calculating the first derivative and substituting the original problem we
get
\begin{align}
    z' &= \frac{-y'}{y^2} = \frac{y-\eps y^2}{y^2} = \frac{1}{y} - \eps = z -
    \eps. \\
    z(0) &= \frac{1}{y(0)} = 1.
\end{align}
The solution is
\begin{align}
    z(x) = \eps + (1-\eps) e^x.
\end{align}
By substituting this into $y = \frac{1}{z}$ and expanding we get
\begin{align}
    y(x) &= \frac{1}{\eps+(1-\eps)e^x} = e^{-x} \frac{1}{1 - (1- e^{-x})\eps}
    \\
         &= e^{-x} \sum_{n\geq 0} \eps^n(1-e^{-x})^n.
\end{align}
which is the geometric series.
\subsection{Problem 7}
The last problem consists of a perturbation of a partial differential
equation (heat equation).
\begin{align}
    &\partial_t u(x, t) + \partial_x^2 u(x,t) - \eps u(x, t)^2 = 0
    &x\in (0, 1),\; t>0,\\
    &u(x, 0) = \tilde{u}_0(x)  &x\in(0, 1), \\
    &u(0, t) = u(1, t) = 0 & t>0.
\end{align}
The problem is regular because the reduced solution is the regular heat
equation in the one special dimension on $x\in (0, 1)$, we know this is
solvable. By doing the expansion ansatz we can derive the first equations
for the first three terms, the ansatz is always the same
\begin{align}
    u_\eps = u_0 + \eps u_1 + \eps^2 u_2 + O(\eps^3).
\end{align}
Plugging this into the perturbed problem problem and factoring out the terms
in the order of $\eps$ we get
\begin{align}
    &\eps^0 (\partial_t u_0 + \partial_x^2 u_0) + \\
    &\eps^1 (\partial_t u_1 + \partial_x^2 u_1 - u_0^2) +\\
    &\eps^2 (\partial_t u_2 + \partial_x^2 u_2 - 2u_1u_0)  + O(\eps^3) = 0.
\end{align}

We can solve the reduced problem with the initial condition $\tilde{u}_0 =
\sin(\pi x)$ by separation of variables. Setting $u(x, t) = \psi(x) \phi(t)$
and substituting into the equation we get two ordinary differential equation
\begin{align}
    \underbrace{\frac{\psi_{xx}}{\psi}}_{=k}
    +\underbrace{\frac{\phi_t}{\phi}}_{=-k} =  0,
\end{align}
for some $k$. Solving these two by the exponential ansatz.
\begin{align}
    \psi(x) &= A_1 e^{\sqrt{k} x} +A_2 e^{-\sqrt{k} x},\\
    \phi(t) &= A_3 e^{-kt}.
\end{align}
With the initial condition we get the conditions that
\begin{align}
    A_1A_3 &= -A_2 A_3,\\
    k &= \pi^2
\end{align}
we choose $A_1 = A_3 = 1$ , $A_2 = -1$. We get the following solution to the
PDE
\begin{align}
    u(x, t) = \psi(x)\phi(t) = \sin(\pi x) e^{-\pi^2 t}.
\end{align}

%\printbibliography
\end{document}
