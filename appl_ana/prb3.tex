\documentclass[a4paper]{article}

\usepackage[T1]{fontenc}
\usepackage[utf8]{inputenc}
\usepackage{mlmodern}

%\usepackage{ngerman}	% Sprachanpassung Deutsch

\usepackage{graphicx}
\usepackage{geometry}
\geometry{a4paper, top=15mm}

\usepackage{subcaption}
\usepackage[shortlabels]{enumitem}
\usepackage{amssymb}
\usepackage{amsthm}
\usepackage{amsmath}
\usepackage{mathtools}
\usepackage{braket}
\usepackage{bbm}
\usepackage{graphicx}
\usepackage{float}
\usepackage{yhmath}
\usepackage{tikz}
\usepackage{scratch}
\usetikzlibrary{patterns,decorations.pathmorphing,positioning}
\usetikzlibrary{calc,decorations.markings}

\usepackage[backend=biber, sorting=none]{biblatex}
\addbibresource{cite.bib}

\usepackage[framemethod=TikZ]{mdframed}

\tikzstyle{titlered} =
    [draw=black, thick, fill=white,%
        text=black, rectangle,
        right, minimum height=.7cm]


\usepackage[colorlinks=true,naturalnames=true,plainpages=false,pdfpagelabels=true]{hyperref}
\usepackage[parfill]{parskip}
\usepackage{lipsum}

\usepackage{tcolorbox}
\tcbuselibrary{skins,breakable}

\pagestyle{myheadings}

\colorlet{colexam}{black}
\newcounter{definition}
\newtcolorbox[use counter=definition]{mydef}[1]{
    empty,
    title={\textbf{Definition~\thetcbcounter}~~(\textit{#1})},
    attach boxed title to top left,
    fontupper=\sl,
    boxed title style={
        empty,
        size=minimal,
        bottomrule=1pt,
        top=1pt,
        left skip=0cm,
        overlay=
            {\draw[colexam,line width=1pt]([yshift=-0.4cm]frame.north
        west)--([yshift=-0.4cm]frame.north east);}},
            coltitle=colexam,
            fonttitle=\normalfont,
            before=\par\medskip\noindent,
            parbox=false,
            boxsep=-1pt,
            left=0.75cm,
            right=3mm,
            top=4pt,
            breakable,
            pad at break*=0mm,
            vfill before first,
            overlay unbroken={
                \draw[colexam,line width=1pt]
                ([xshift=0.6cm, yshift=-0.5pt]frame.south
                west)--([xshift=0.6cm,yshift=-1pt]frame.north west)
                --([xshift=0.6cm]frame.south west)--([xshift=-13cm]frame.south east); },
            overlay first={
                \draw[colexam,line width=1pt]
                ([xshift=0.6cm, yshift=-0.5pt]frame.south
                west)--([xshift=0.6cm,yshift=-1pt]frame.north west)
                --([xshift=0.6cm]frame.south west); },
            overlay last={
                \draw[colexam,line width=1pt]
                ([xshift=0.6cm, yshift=-0.5pt]frame.south
                west)--([xshift=0.6cm,yshift=-1pt]frame.north west)
                --([xshift=0.6cm]frame.south west)--([xshift=-13cm]frame.south east); }
}
\newcounter{theorem}
\newtcolorbox[use counter=theorem]{theorem}{
    empty,
    title={Theorem ~\thetcbcounter},
    attach boxed title to top left,
    fontupper=\sl,
    boxed title style={
        empty,
        size=minimal,
        bottomrule=1pt,
        top=1pt,
        left skip=0cm,
        overlay=
            {\draw[colexam,line width=1pt]([yshift=-0.4cm]frame.north
        west)--([yshift=-0.4cm]frame.north east);}},
            coltitle=colexam,
            fonttitle=\bfseries,
            before=\par\medskip\noindent,
            parbox=false,
            boxsep=-1pt,
            left=0.75cm,
            right=3mm,
            top=4pt,
            breakable,
            pad at break*=0mm,
            vfill before first,
            overlay unbroken={
                \draw[colexam,line width=1pt]
                ([xshift=0.6cm, yshift=-0.5pt]frame.south
                west)--([xshift=0.6cm,yshift=-1pt]frame.north west)
                --([xshift=0.6cm]frame.south west)--([xshift=-13cm]frame.south east); },
            overlay first={
                \draw[colexam,line width=1pt]
                ([xshift=0.6cm, yshift=-0.5pt]frame.south
                west)--([xshift=0.6cm,yshift=-1pt]frame.north west)
                --([xshift=0.6cm]frame.south west); },
            overlay last={
                \draw[colexam,line width=1pt]
                ([xshift=0.6cm, yshift=-0.5pt]frame.south
                west)--([xshift=0.6cm,yshift=-1pt]frame.north west)
                --([xshift=0.6cm]frame.south west)--([xshift=-13cm]frame.south east); }
}
\newcounter{lemma}
\newtcolorbox[use counter=lemma]{lemma}{
    empty,
    title={Lemma~\thetcbcounter},
    attach boxed title to top left,
    fontupper=\sl,
    boxed title style={
        empty,
        size=minimal,
        bottomrule=1pt,
        top=1pt,
        left skip=0cm,
        overlay=
            {\draw[colexam,line width=1pt]([yshift=-0.4cm]frame.north
        west)--([yshift=-0.4cm]frame.north east);}},
            coltitle=colexam,
            fonttitle=\bfseries,
            before=\par\medskip\noindent,
            parbox=false,
            boxsep=-1pt,
            left=0.75cm,
            right=3mm,
            top=4pt,
            breakable,
            pad at break*=0mm,
            vfill before first,
            overlay unbroken={
                \draw[colexam,line width=1pt]
                ([xshift=0.6cm, yshift=-0.5pt]frame.south
                west)--([xshift=0.6cm,yshift=-1pt]frame.north west)
                --([xshift=0.6cm]frame.south west)--([xshift=-13cm]frame.south east); },
            overlay first={
                \draw[colexam,line width=1pt]
                ([xshift=0.6cm, yshift=-0.5pt]frame.south
                west)--([xshift=0.6cm,yshift=-1pt]frame.north west)
                --([xshift=0.6cm]frame.south west); },
            overlay last={
                \draw[colexam,line width=1pt]
                ([xshift=0.6cm, yshift=-0.5pt]frame.south
                west)--([xshift=0.6cm,yshift=-1pt]frame.north west)
                --([xshift=0.6cm]frame.south west)--([xshift=-13cm]frame.south east); }
}

\newcommand{\eps}{\varepsilon}
\usepackage[OT2,T1]{fontenc}
\DeclareSymbolFont{cyrletters}{OT2}{wncyr}{m}{n}
\DeclareMathSymbol{\Sha}{\mathalpha}{cyrletters}{"58}

\markright{Popović\hfill Seminar\hfill}


\title{University of Vienna\\
\vspace{1cm}Seminar:\\Joint RICAM Seminar\\
\vspace{0.5cm}
Summary of talk by Otmar Scherzer
}
\author{Milutin Popovic}


\begin{document}
\maketitle
\tableofcontents

\section{Sheet 3}
\subsection{Problem 8}
Let us look at functions $f: \mathcal{D} \mapsto \mathbb{R}$ that show
boundary layer behavior at the following manifolds.

The \textbf{first} for $\mathcal{D} = \mathbb{R}^2$ and $S = \{0\}$ we have a
function e.g.
\begin{align}
    f_{\eps}(x, y) = e^{-\frac{x}{\eps}} + y,
\end{align}
with the reduced equation
\begin{align}
    \lim_{\eps \rightarrow 0} f_{\eps}(x, y) =
    \begin{cases}
        y \;\;\;\;\;\;\;\;\;\; x > 0\\
        1+y \;\;\;\; x = 0\\
    \end{cases}
\end{align}

The \textbf{second} example is $\mathcal{D} = \mathbb{R}^n$ and $S = \{|x| = 1\}$.
\begin{align}
    f_\eps(x_1,\dots,x_n) = \tanh\left(\frac{|x| - 1}{\eps} \right),
\end{align}
with the reduced equation
\begin{align}
    \lim_{\eps \rightarrow 0} f_{\eps}(x_1,\dots, x_n) =
    \begin{cases}
        -1 \;\;\;\; |x| < 0\\
        1  \;\;\;\;\;\;\; |x| > 0\\
    \end{cases}
\end{align}

The \textbf{third} example is $\mathcal{D} = \mathbb{R}^3$ and $S = \{x_1 =
1\}$
\begin{align}
    f_\eps(x_1, x_2, x_3) = \tanh\left(\frac{x_1 - 1}{\eps}\right)+x_2x_3
\end{align}
with the reduced equation
\begin{align}
    \lim_{\eps \rightarrow 0} f_{\eps}(x_1,x_2,x_3) =
    \begin{cases}
        -1 + x_2x_3 \;\;\;\; x_1 < 0\\
        1 + x_2x_3 \;\;\;\;\;\;\; x_1 > 0\\
    \end{cases}
\end{align}
\subsection{Problem 9}
Consider a linear BVP
\begin{align}
    Lu := -\eps u'' + b(x)u' + c(x)u = f(x),\\
    u(0) = u(1) = 0,
\end{align}
for $0 < \eps \ll \eps_0$ and $b, c, f \in C([0,1])$ with the conditions
\begin{align}
    c(x) \geq 0, \qquad b(x) \geq \beta > 0 \qquad x\in[0, 1]
\end{align}
We are to show that for all $x\in[0, 1)$ the reduced solution $u_0$ of the
above BVP satisfies
\begin{align}
    \lim_{\eps \rightarrow 0} u_\eps(x) = u_0(x)))),
\end{align}
where the reduced solution $u_0$ is the solution to the following
differential equation
\begin{align}
    b(x)u' + c(x)u = f(x), \quad u(0) = 0.
\end{align}
The hint was given: Set
\begin{align}
    w_1(x) = e^{\beta x} \quad w_2(x) = e^{-\beta\frac{1-x}{\eps}},
\end{align}
such that $Lw_1 \geq \gamma > 0$ for some suitable $\gamma$ and $Lw_2 \geq
0$. Then for
\begin{align}
    v = \pm (u_\eps - u_0), \qquad w = A\eps w_1 + B\eps w_2,
\end{align}
for some suitable $A, B$. The following comparison principal is applicable:
IF
\begin{align}
    &Lv(x) \leq Lw(x) \quad \forall x \in (0, 1) \label{eq:cond1}\\
    &v(0) \leq w(0)  \label{eq:cond2}\\
    &v(1) \leq w(1) \label{eq:cond3}\\
\end{align}
then
\begin{align}
    \Longrightarrow v(x) \leq w(x) \quad \forall x\in(0, 1)
\end{align}
which holds for $u, v \in C^2((0, 1)) \cap C([0, 1])$. Thus a boundary layer
is possible only at $x=1$. On the other hand, for $b(x) \leq \beta < 0$ it
follows that the boundary layer is possible only at $x=0$.

We shall go through the chronological order of the conditions\ref{eq:cond1},
\ref{eq:cond2}, \ref{eq:cond3} and check them. So for \ref{eq:cond1}
we have that
\begin{align}
    Lw(x) &= A\eps Lw_1(x) + B Lw_2(x) \\
          &\geq A\eps Lw_1(x) = A\eps e^{\beta x} \left(-\eps \beta^2 -
              b(x)\beta+c(x)\right)\\
          &\geq A\eps \beta e^{\beta x} \left(1-\eps\right)\\
          &\geq \eps A\beta^2 e^{\beta}(1-\eps) = \gamma > 0
\end{align}
And obviously
\begin{align}
    Lv(x) \leq 0 ,
\end{align}
by that we have that
\begin{align}
    Lv(x) \leq \gamma \leq  Lw(x).
\end{align}
For the condition \ref{eq:cond2} we have
\begin{align}
    w(0) &= A\eps w_1(0) Bw_2(0) = Be^{-\frac{\beta}{\eps}},\\
    v(0) &= \pm\left(u_\eps(0) - u_0(0)\right) = 0.
\end{align}
By the simple choice $B \geq 0$ we satisfy the condition
\begin{align}
    v(0) \leq w(0).
\end{align}
Now for the last condition \ref{eq:cond3} we have
\begin{align}
    w(1) &= A\eps e^\beta + B \geq A\eps e^\beta,\\
    v(1) &= \mp u_0(1) = 0.
\end{align}
And choose $A = \frac{\pm u(1)}{\eps} e^{-\beta}$, which satisfies the last
condition
\begin{align}
    v(1) \leq w(1).
\end{align}
Thereby we have
\begin{align}
    &v(x) &\leq w(x)\\
    &\Rightarrow \lim_{\eps \rightarrow 0} v(x) &\leq \lim_{\eps \rightarrow
    0} w(x) = 0\\
    &\Rightarrow \lim_{\eps \rightarrow 0} v(x) = 0
\end{align}
uniformly on $(0, 1)$.
\subsection{Problem 10}
Consider the following BVP
\begin{align}
    -\eps u'' + (1 + x)u' + u = 2, \qquad u(0) = u(1) - 0,
\end{align}
for $0 < \eps \ll 1$. \textbf{Where can this problem have a boundary layer?}
To answer this question we need to look at the reduced problem
\begin{align}
    -(1+x)u' + u = 2.
\end{align}
The solution to the equation is
\begin{align}
    \bar{u}(x) = 2 + A(x+1).
\end{align}
According to the boundary conditions it is unclear what the value of the
constant is, according to $\bar{u}(0)=0$ we get $A = -2$ or according to
$\bar{u}(1)=0$ we get $A = -1$. Ultimately this means that there exists a
boundary layer near $x=1$ or $x=0$. We choose $x=0$ and according to this the
local variable $\xi = x\eps^{-\alpha}$ ($x = \xi \eps^{-\alpha}$). The
derivatives of $u$ are calculated using the chain rule
\begin{align}
    \frac{du}{dx}&= \frac{du}{d\xi}\frac{d\xi}{dx} = \eps^{-\alpha} \dot{u}\\
    \frac{d^2u}{dx^2}&= \eps^{-\alpha} \frac{d^2u}{d\xi^2}\frac{d\xi}{dx} =
    \eps^{-2\alpha} \ddot{u}.
\end{align}
The BVP transforms as follows
\begin{align}
    -\eps^{1-\alpha}\ddot{u} - \dot{u} + \eps(u - \xi\dot{u} - 2) =
    \begin{cases}
        -\ddot{u} - \dot{u} = 0 \;\;\;\;\; \alpha=1\\
        -\dot{u} = 0 \;\;\;\;\;\;\;\; 0<\alpha<1
    \end{cases}
\end{align}
Choosing $\alpha = 1$ for a reasonable solution
\begin{align}
    \hat{u}(\xi) = Be^{-\xi},
\end{align}
which converges in the local limit (!). Thereby we have a asymptotic
representation up to the degree of $\eps$

\begin{align}
    u_\eps(x) &= \bar{u}(x) + \hat{u}(\psi) + O(\eps)\\
              &= 2 + A(1+x) + B e^{-\frac{x}{\eps}}  + O(\eps)
\end{align}
And by the boundary conditions
\begin{align}
    u_\eps(0) = 2+A+B=0, \qquad u_\eps(1) = 2+2A+B=0,
\end{align}
we get that the constants are
\begin{align}
    A = -4, \qquad B = 2.
\end{align}
The asymptotic representation is thereby
\begin{align}
    u_\eps(x) = 2 - 4(1+x) + 2 e^{-\frac{x}{\eps}}  + O(\eps)
\end{align}
%\printbibliography
\end{document}
