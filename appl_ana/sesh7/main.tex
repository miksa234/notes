\documentclass[a4paper]{article}


\usepackage[T1]{fontenc}
\usepackage[utf8]{inputenc}
\usepackage{mlmodern}

%\usepackage{ngerman}	% Sprachanpassung Deutsch

\usepackage{graphicx}
\usepackage{geometry}
\geometry{a4paper, top=15mm}

\usepackage{subcaption}
\usepackage[shortlabels]{enumitem}
\usepackage{amssymb}
\usepackage{amsthm}
\usepackage{mathtools}
\usepackage{braket}
\usepackage{bbm}
\usepackage{graphicx}
\usepackage{float}
\usepackage{yhmath}
\usepackage{tikz}
\usetikzlibrary{patterns,decorations.pathmorphing,positioning}
\usetikzlibrary{calc,decorations.markings}

%\usepackage[backend=biber, sorting=none]{biblatex}
%\addbibresource{uni.bib}

\usepackage[framemethod=TikZ]{mdframed}

\tikzstyle{titlered} =
    [draw=black, thick, fill=white,%
        text=black, rectangle,
        right, minimum height=.7cm]


\usepackage[colorlinks=true,naturalnames=true,plainpages=false,pdfpagelabels=true]{hyperref}
\usepackage[parfill]{parskip}
\usepackage{lipsum}

\usepackage[OT2,T1]{fontenc}
\DeclareSymbolFont{cyrletters}{OT2}{wncyr}{m}{n}
\DeclareMathSymbol{\Sha}{\mathalpha}{cyrletters}{"58}

\usepackage{tcolorbox}
\tcbuselibrary{skins,breakable}

\pagestyle{myheadings}

\markright{Popović\hfill Applied Analysis\hfill}


\title{University of Vienna\\ Faculty of Mathematics\\
\vspace{1cm}Applied Analysis Problems
}
\author{Milutin Popovic}

\begin{document}
\maketitle
\tableofcontents

\section{Sheet 7}
\subsection{Dirac Comb}
The Dirac train or Dirac comb on defined in the following way
\begin{align}
    \Sha_m[n] =
    \begin{cases}
        1\;\;\;\;\;\; n = 0, \pm m, \pm 2m,\dots\\
        0\;\;\;\;\;\; \text{else}
    \end{cases}
\end{align}
The discrete Fourier transform of the Dirac comb in $\mathbb{C}^N$ is
\begin{align}
    \widehat{\Sha_m[n]}
    &=\frac{1}{N}\sum_{n=0}^{N-1} \Sha_m[n] e^{-2\pi i \frac{k}{N}n}=\\
    &=\frac{1}{N}\sum_{n=0}^{N-1}
        \left(
            m\sum_{l\in\mathbb{Z}} \delta(n-lm)
            \right)
            e^{-2\pi i \frac{k}{N}n}=\\
    &=\frac{m}{N}\sum_{l\in\mathbb{Z}} e^{-2\pi i \frac{k}{N}lm}=
    \;\;\;\;\;\;\;\;\;\;\;\;\; (m = \frac{N}{m'})\\
    &= \frac{1}{m'} \sum_{l\in\mathbb{Z}} e^{-2\pi i \frac{k}{m'}l} =\\
    &= \frac{1}{m}\Sha_{\frac{N}{m}}[k]
\end{align}
\subsection{Schwartz Space}
The Schwartz space $\mathcal{S}(\mathbb{R}^d)$, for $d \in \mathbb{N}$ is
defined as
\begin{align}
    &\mathcal{S} :=
\bigg\{
    f\in\mathcal{C}^\infty(\mathbb{R}^d):
    \forall\alpha,\beta\in\mathbb{N}^d\;\; \lVert f \rVert_{\alpha,\beta}
    < \infty
\bigg\},\\
&\lVert f \rVert_{\alpha, \beta} :=
\sup_{x\in\mathbb{R}^d}\left|x^\alpha (D^\beta f) (x) \right|.
\end{align}
Our aim is to show that if $f\in\mathcal{S}(\mathbb{R})$ then $\hat{f} \in
\mathcal{S}(\mathbb{R})$. The condition is obviously
\begin{align}
    &\lVert \hat{f} \rVert_{\alpha, \beta} =
    \sup_{\xi\in\mathbb{R}}\left|\xi^\alpha (D^\beta \hat{f}) (\xi)
    \right|<\infty,
\end{align}
for all $\alpha, \beta \in \mathbb{N}$.
We can start with what we know about the Fourier transform
\begin{align}
    \xi^\alpha \hat{f}(\xi) &= \mathcal{F}\left(\frac{1}{(2\pi
    i)^\alpha}(D^{\alpha}f)(x)\right)\\
            D^{\beta}\hat{f}(\xi) &= \mathcal{F}\left(
        (-2\pi i x)^\beta f(x)
    \right).
\end{align}
Combining the two relations above we get
\begin{align}
    \xi^\alpha (D^\beta \hat{f})(\xi) =
    \mathcal{F}\left(\frac{(-2\pi i x)^\beta}{(2\pi
    i)^\alpha}x^\beta(D^{\alpha}f)(x)\right)=: \mathcal{F}(g(x))\\
\end{align}
If we call this function $g$, then $g\in\mathcal{S}(\mathbb{R})$ and
$g\in L^1(\mathbb{R})$. Applying the Riemann-Lebesgue Lemma we get
\begin{align}
    \hat{g}(\xi) = \int_\mathbb{R} g(x) e^{-2\pi i x \xi}\ dx \longrightarrow 0
    \;\;\;  \text{as $|\xi| \rightarrow \infty$ }
\end{align}
Thereby $\hat{g} \in \mathcal{S}(\mathbb{R})$ and thus $\hat{f} \in
\mathcal{S}(\mathbb{R})$.
\subsection{Tempered Distributions}
Tempered distributions are the elements of
\begin{align}
    \mathcal{S}'(\mathbb{R}^d) :=
    \bigg\{
        L: \mathcal{S}(\mathbb{R}^d) \rightarrow \mathbb{C} | \text{$L$ is
        linear and continuous}
    \bigg\}.
\end{align}
Consider $\xi$ as a tempered distribution, buy acting on $\varphi \in
\mathcal{S}(\mathbb{R})$ we have
\begin{align}
    \xi(\phi) = \int_\xi \xi \varphi(\xi)\ d\xi.
\end{align}
The Fourier transform of $\xi$ is
\begin{align}
    \hat{\xi}(\varphi)
    &=\xi(\hat{\varphi})
    = \int_\mathbb{R} \xi \hat{\varphi}(\xi)\ d\xi=\\
    &= \int_\mathbb{R}\xi \int_\mathbb{R} \varphi(x) e^{-2\pi i \xi x}\ dx\
    d\xi=\\
    &=\int_\mathbb{R}\int_\mathbb{R} \xi e^{-2\pi i \xi x}\ d\xi \varphi(x)\
    dx=\\
    &=\int_\mathbb{R} \frac{i}{2\pi}\frac{d}{dx}\int_\mathbb{R}e^{-2\pi i \xi
x} \ d\xi \varphi(x)\ dx=\\
    &= \int_\mathbb{R}i\frac{1}{2\pi} \frac{d}{dx}\left(2\pi \delta(x)\right)
    \varphi(x)\ dx=\\
    &= \int_\mathbb{R} i\delta'(x)\varphi(x)\ dx\\
    &= i \delta'(\varphi).
\end{align}
\subsection{Fourier transform of the Dirac Comb}
The general case of the Dirac Comb as a distribution is
\begin{align}
    \Sha_T = \sum_{n \in \mathbb{Z}} \delta_{nT}.
\end{align}
The Fourier transform of the $\Sha_T$ distribution for $\varphi \in
\mathcal{S}(\mathbb{R})$ is
\begin{align}
    \widehat{\Sha_T}(\varphi)
    &= \sum_{n\in\mathbb{Z}} \hat{\delta}_{nT}(\varphi)\\
    &= \sum_{n\in\mathbb{Z}} \delta_{n\omega_0}(\varphi)\\
    &=\Sha_{\omega_0}(\varphi).
\end{align}
The Fourier transform, transforms the period of the combs.
\subsection{Shannon Sampling}
The Fourier transform of $1_{[-\frac{a}{2}, \frac{a}{2}]}(x)$ is
\begin{align}
    \mathcal{F}\left(1_{[-\frac{a}{2}, \frac{a}{2}]}\right)(\xi)
    &= \int_\mathbb{R} 1_{[-\frac{a}{2}, \frac{a}{2}]} e^{-2\pi i x \xi}\
    dx\\
    &= \int_{-\frac{a}{2}}^{\frac{a}{2}} e^{-2\pi i x\xi}\ dx\\
    &= \frac{-1}{2\pi i \xi} e^{-2\pi i x
        \xi}\bigg|_{-\frac{a}{2}}^{\frac{a}{2}}\\
    &= \frac{1}{\pi \xi} \frac{1}{2i}\left(
        e^{pi i a \xi} - e^{-\pi i a \xi}
    \right)\\
    &= \frac{\sin(\pi \xi a)}{\pi \xi}
\end{align}

%\printbibliography
\end{document}
