\documentclass[a4paper]{article}


\usepackage[T1]{fontenc}
\usepackage[utf8]{inputenc}
\usepackage{mlmodern}

%\usepackage{ngerman}	% Sprachanpassung Deutsch

\usepackage{graphicx}
\usepackage{geometry}
\geometry{a4paper, top=15mm}

\usepackage{subcaption}
\usepackage[shortlabels]{enumitem}
\usepackage{amssymb}
\usepackage{amsthm}
\usepackage{mathtools}
\usepackage{braket}
\usepackage{bbm}
\usepackage{graphicx}
\usepackage{float}
\usepackage{yhmath}
\usepackage{tikz}
\usetikzlibrary{patterns,decorations.pathmorphing,positioning}
\usetikzlibrary{calc,decorations.markings}

%\usepackage[backend=biber, sorting=none]{biblatex}
%\addbibresource{uni.bib}

\usepackage[framemethod=TikZ]{mdframed}

\tikzstyle{titlered} =
    [draw=black, thick, fill=white,%
        text=black, rectangle,
        right, minimum height=.7cm]


\usepackage[colorlinks=true,naturalnames=true,plainpages=false,pdfpagelabels=true]{hyperref}
\usepackage[parfill]{parskip}
\usepackage{lipsum}


\usepackage{tcolorbox}
\tcbuselibrary{skins,breakable}

\pagestyle{myheadings}

\markright{Popović\hfill Applied Analysis\hfill}


\title{University of Vienna\\ Faculty of Mathematics\\
\vspace{1cm}Applied Analysis Problems
}
\author{Milutin Popovic}

\begin{document}
\maketitle
\tableofcontents

\section{Sheet 4}

\subsection{Fourier Series}
The Fourier series of a $p$ periodic function $f$, integrable on
$[-\frac{p}{2}, \frac{p}{2}]$ is
\begin{align}
    f(x) = \frac{a_0}{2} + \sum_{n=1}^\infty \left(a_n \cos(\frac{2\pi n x}{p})
    b_n sin(\frac{2\pi n x}{p})\right).
\end{align}
The coefficients $a_n$ and $b_n$ are called the Fourier coefficients of $f$
and are given by
\begin{align}
    a_n &= \frac{2}{p} \int_{-\frac{p}{2}}^{\frac{p}{2}} f(x) \sin(\frac{2\pi
    n x}{p}) dx, \;\;\;\;\; n\geq 0 \\
    b_n &= \frac{2}{p} \int_{-\frac{p}{2}}^{\frac{p}{2}} f(x) \cos(\frac{2\pi
    n x}{p}) dx, \;\;\;\;\; n\geq 1
\end{align}
Let us compute the Fourier series of $f(t) = t$ for $t \in [-\frac{1}{2},
\frac{1}{2}]$. The Fourier coefficients are
\begin{align}
    a_n &= 2\int_{-\frac{1}{2}}^{\frac{1}{2}} t \cos(2\pi n t)\ dt = 0
    \;\;\;\;\; \text{(odd: g(-t) = -g(t))},\\
    \nonumber\\
    b_n &= 2\int_{-\frac{1}{2}}^{\frac{1}{2}} t \sin(2\pi n t)\ dt = \\
        &= 2 \left(-\frac{1}{2\pi n} \cos(2\pi n
        t)\bigg|_{-\frac{1}{2}}^{\frac{1}{2}}
        +\int_{\frac{1}{2}}^{\frac{1}{2}} \frac{1}{2 \pi n}\cos(2\pi n t)\ dt
            \right) =\\
        &= -\frac{1}{\pi n}\left( -\cos(\pi n) + \frac{1}{\pi n }\sin(\pi
            n)\right) =
        \frac{\sin(\pi n) - \pi n \cos(\pi n)}{(\pi n)^2}.
\end{align}
Thereby the Fourier series of $f(t) = t$ is
\begin{align}
    f(t) = \sum_{n=1}^\infty \left(\frac{\sin(\pi n) - \pi n \cos(\pi n)}{(\pi
    n)^2}\right) \sin(2\pi n t) = t
\end{align}
\subsection{Truncation Error}
The truncation error of the trigonometric polynomial $(Sf_N)$ of degree $N$ is
\begin{align}
    \sum_{|k| > N} |\hat{f}(k)|^2 = \lVert f - S_N\rVert_2^2 =
    \int_{-\frac{1}{2}}^{\frac{1}{2}} |E_N(t)|^2 dt.
\end{align}
Computations for $N = 3$ and $N = 9$ were done in python with a integration error of
around $10^{-15}$, resulting in the overall truncation errors of
\begin{align}
    \sum_{|k| > 3} |\hat{f}(k)|^2 = 0.0053,\\
    \sum_{|k| > 9} |\hat{f}(k)|^2 = 0.0143.
\end{align}
To achieve $\lVert E_N\rVert^2_2 < 0.1 \lVert f \rVert^2_2$, the number of
coefficients needed are about $61$. This was done using a while loop and
evaluating $\lVert E_N\rVert^2_2$ for $N$ until the above condition is met.

\subsection{Orthonormal Bases}
Here we will go through the most important properties of orthonormal bases.
So let $\{b_n\}_{n\in \mathbb{N}}$ be an ONB of a vector space $\mathcal{H}$,
then for every $x\in \mathcal{H}$ we may write
\begin{align}
    x = \sum_{b_n} \langle b_n, x\rangle b_n,
\end{align}
and
\begin{align}
    \lVert x \rVert^2 = \sum_{b_n} |\langle b_n, x\rangle|^2.
\end{align}
For any $x, y \in \mathcal{H}$ we can write the scalar product as
\begin{align}
    \langle x, y\rangle = \sum_{b_n} \langle b_n, x\rangle \langle b_n,
    y\rangle,
\end{align}
Furthermore there exists a linear projection $\Phi\ : \mathcal{H}
\rightarrow l^2(\{b_n\}_n)$ such that
\begin{align}
    \langle \Phi(x), \Phi(y)\rangle = \langle x, y \rangle\;\;\; \forall x, y
    \in \mathcal{H}.
\end{align}

An example of an orthonormal basis, which spans $L^2([-\frac{p}{2},
\frac{p}{2}])$ is $\mathcal{T}_p = \{e_n := \frac{e^{2\pi i
\frac{n}{p}x}}{\sqrt{p}}\}_{n\in\mathbb{Z}}$. The $e_n$'s are orthonormal in
$L^2$ which can be easily seen by using the scalar product of $L^2$, so for
$n, m \in \mathbb{Z}$
\begin{align}
    \langle e_n, e_m\rangle_{L^2([-\frac{p}{2}, \frac{p}{2})} &=
    \frac{1}{p}\int_{[-\frac{p}{2}, \frac{p}{2}]}e_n \cdot e_m^* \ dx=\\
    &=\frac{1}{p}\int_{[-\frac{p}{2}, \frac{p}{2}]} e^{2\pi i \frac{(n-m)}{p} x} \ dx=\\
    &=\frac{\sin(\pi (n-m))}{\pi(n-m)} =
    \begin{cases}
        0  \;\;\;\; n\neq m\\
        1 \;\;\;\;  n=m
    \end{cases}
\end{align}
\subsection{Dirichlet Kernel}
The function
\begin{align}
    D_t(x) := \sum_{\lVert k \rVert_\infty \leq t} e_k(x), \;\;\;\;\; x\in
    \mathbb{R}^d
\end{align}
is called the Dirichlet Kernel. For $0 < t \in \mathbb{N}$ we have
\begin{align}
    (S_tf)(x) = \int_{I^d} f(y) D_t(x-y) dy,
\end{align}
where $S_t$ represents the orthogonal projection onto the trigonometric
polynomials $\Pi_t$ of degree $t$, by
\begin{align}
    &S_t:\ L^1(\mathbb{T}^d) \rightarrow \Pi_t \\
    &f \mapsto \sum_{\lVert k \rVert \leq t} \langle f,
    e_k\rangle_{L^2(\mathbb{T}^d)} e_k \;\;\;\;\; k \in \mathbb{Z}^d
\end{align}
And furthermore the Dirichlet Kernel satisfies
\begin{align}
    D_t(x) = \prod_{i=1}^d \frac{e_{t+1}(x_i) - e_{-t}(x_i)}{e_1(x_i) - 1}
\end{align}
To show the convolution property, we start off by applying the orthogonal
projection into the trigonometric polynomials $S_t$ onto a function $f \in
L(\mathbb{T}^d)$
\begin{align}
    (S_tf) &= \sum_{\lvert k\rVert_\infty \leq t} \int_{I^d} f(y) e^{-2\pi i
    \langle k, y\rangle}\ dy\ e^{2\pi i\langle k, x\rangle} =\\
    &= \int_{I^d}f(y) \sum_{\lvert k\rVert_\infty \leq t} e^{2\pi i \langle
    k, (x- y)\rangle}\ dy =\\
    &= (f * D_t) (x) = \int_{I^d} f(y) D_t(x - y)\ dy.
\end{align}
To show the reformulation of the Dirichlet kernel, we need to simply
calculate it directly
\begin{align}
    \sum_{\lVert k \rVert_\infty \leq t} e^{2\pi i \langle k , x\rangle} &=
    \prod_{j=1}^d \sum_{k_j = -t}^t e^{2\pi i k_j x_j} =\\
    &= \prod_{j=1}^d e^{-2\pi i t x_j} \sum_{k_j = 0}^{2t} e^{2\pi i k_j
        x_j}=;\;\;\;\; \text{(trigonometric series)}\\
    &= \prod_{j=1}^d e^{-2\pi i t x_j} \frac{e^{2\pi i (2t + 1)x_j} -
    1}{e^{2\pi i x_j} - 1} =\\
    &= \prod_{j = 1} \frac{e_{t+1}(x_j) - e_{-t}(x_j)}{e_1(x_j) - 1}.
\end{align}
%\printbibliography
\end{document}
